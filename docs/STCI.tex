\documentclass[]{book}
\usepackage{lmodern}
\usepackage{amssymb,amsmath}
\usepackage{ifxetex,ifluatex}
\usepackage{fixltx2e} % provides \textsubscript
\ifnum 0\ifxetex 1\fi\ifluatex 1\fi=0 % if pdftex
  \usepackage[T1]{fontenc}
  \usepackage[utf8]{inputenc}
\else % if luatex or xelatex
  \ifxetex
    \usepackage{mathspec}
  \else
    \usepackage{fontspec}
  \fi
  \defaultfontfeatures{Ligatures=TeX,Scale=MatchLowercase}
\fi
% use upquote if available, for straight quotes in verbatim environments
\IfFileExists{upquote.sty}{\usepackage{upquote}}{}
% use microtype if available
\IfFileExists{microtype.sty}{%
\usepackage{microtype}
\UseMicrotypeSet[protrusion]{basicmath} % disable protrusion for tt fonts
}{}
\usepackage[margin=1in]{geometry}
\usepackage{hyperref}
\hypersetup{unicode=true,
            pdftitle={Statistical Tools for Causal Inference},
            pdfauthor={The SKY Community},
            pdfborder={0 0 0},
            breaklinks=true}
\urlstyle{same}  % don't use monospace font for urls
\usepackage{color}
\usepackage{fancyvrb}
\newcommand{\VerbBar}{|}
\newcommand{\VERB}{\Verb[commandchars=\\\{\}]}
\DefineVerbatimEnvironment{Highlighting}{Verbatim}{commandchars=\\\{\}}
% Add ',fontsize=\small' for more characters per line
\usepackage{framed}
\definecolor{shadecolor}{RGB}{248,248,248}
\newenvironment{Shaded}{\begin{snugshade}}{\end{snugshade}}
\newcommand{\KeywordTok}[1]{\textcolor[rgb]{0.13,0.29,0.53}{\textbf{#1}}}
\newcommand{\DataTypeTok}[1]{\textcolor[rgb]{0.13,0.29,0.53}{#1}}
\newcommand{\DecValTok}[1]{\textcolor[rgb]{0.00,0.00,0.81}{#1}}
\newcommand{\BaseNTok}[1]{\textcolor[rgb]{0.00,0.00,0.81}{#1}}
\newcommand{\FloatTok}[1]{\textcolor[rgb]{0.00,0.00,0.81}{#1}}
\newcommand{\ConstantTok}[1]{\textcolor[rgb]{0.00,0.00,0.00}{#1}}
\newcommand{\CharTok}[1]{\textcolor[rgb]{0.31,0.60,0.02}{#1}}
\newcommand{\SpecialCharTok}[1]{\textcolor[rgb]{0.00,0.00,0.00}{#1}}
\newcommand{\StringTok}[1]{\textcolor[rgb]{0.31,0.60,0.02}{#1}}
\newcommand{\VerbatimStringTok}[1]{\textcolor[rgb]{0.31,0.60,0.02}{#1}}
\newcommand{\SpecialStringTok}[1]{\textcolor[rgb]{0.31,0.60,0.02}{#1}}
\newcommand{\ImportTok}[1]{#1}
\newcommand{\CommentTok}[1]{\textcolor[rgb]{0.56,0.35,0.01}{\textit{#1}}}
\newcommand{\DocumentationTok}[1]{\textcolor[rgb]{0.56,0.35,0.01}{\textbf{\textit{#1}}}}
\newcommand{\AnnotationTok}[1]{\textcolor[rgb]{0.56,0.35,0.01}{\textbf{\textit{#1}}}}
\newcommand{\CommentVarTok}[1]{\textcolor[rgb]{0.56,0.35,0.01}{\textbf{\textit{#1}}}}
\newcommand{\OtherTok}[1]{\textcolor[rgb]{0.56,0.35,0.01}{#1}}
\newcommand{\FunctionTok}[1]{\textcolor[rgb]{0.00,0.00,0.00}{#1}}
\newcommand{\VariableTok}[1]{\textcolor[rgb]{0.00,0.00,0.00}{#1}}
\newcommand{\ControlFlowTok}[1]{\textcolor[rgb]{0.13,0.29,0.53}{\textbf{#1}}}
\newcommand{\OperatorTok}[1]{\textcolor[rgb]{0.81,0.36,0.00}{\textbf{#1}}}
\newcommand{\BuiltInTok}[1]{#1}
\newcommand{\ExtensionTok}[1]{#1}
\newcommand{\PreprocessorTok}[1]{\textcolor[rgb]{0.56,0.35,0.01}{\textit{#1}}}
\newcommand{\AttributeTok}[1]{\textcolor[rgb]{0.77,0.63,0.00}{#1}}
\newcommand{\RegionMarkerTok}[1]{#1}
\newcommand{\InformationTok}[1]{\textcolor[rgb]{0.56,0.35,0.01}{\textbf{\textit{#1}}}}
\newcommand{\WarningTok}[1]{\textcolor[rgb]{0.56,0.35,0.01}{\textbf{\textit{#1}}}}
\newcommand{\AlertTok}[1]{\textcolor[rgb]{0.94,0.16,0.16}{#1}}
\newcommand{\ErrorTok}[1]{\textcolor[rgb]{0.64,0.00,0.00}{\textbf{#1}}}
\newcommand{\NormalTok}[1]{#1}
\usepackage{longtable,booktabs}
\usepackage{graphicx,grffile}
\makeatletter
\def\maxwidth{\ifdim\Gin@nat@width>\linewidth\linewidth\else\Gin@nat@width\fi}
\def\maxheight{\ifdim\Gin@nat@height>\textheight\textheight\else\Gin@nat@height\fi}
\makeatother
% Scale images if necessary, so that they will not overflow the page
% margins by default, and it is still possible to overwrite the defaults
% using explicit options in \includegraphics[width, height, ...]{}
\setkeys{Gin}{width=\maxwidth,height=\maxheight,keepaspectratio}
\IfFileExists{parskip.sty}{%
\usepackage{parskip}
}{% else
\setlength{\parindent}{0pt}
\setlength{\parskip}{6pt plus 2pt minus 1pt}
}
\setlength{\emergencystretch}{3em}  % prevent overfull lines
\providecommand{\tightlist}{%
  \setlength{\itemsep}{0pt}\setlength{\parskip}{0pt}}
\setcounter{secnumdepth}{5}
% Redefines (sub)paragraphs to behave more like sections
\ifx\paragraph\undefined\else
\let\oldparagraph\paragraph
\renewcommand{\paragraph}[1]{\oldparagraph{#1}\mbox{}}
\fi
\ifx\subparagraph\undefined\else
\let\oldsubparagraph\subparagraph
\renewcommand{\subparagraph}[1]{\oldsubparagraph{#1}\mbox{}}
\fi

%%% Use protect on footnotes to avoid problems with footnotes in titles
\let\rmarkdownfootnote\footnote%
\def\footnote{\protect\rmarkdownfootnote}

%%% Change title format to be more compact
\usepackage{titling}

% Create subtitle command for use in maketitle
\newcommand{\subtitle}[1]{
  \posttitle{
    \begin{center}\large#1\end{center}
    }
}

\setlength{\droptitle}{-2em}

  \title{Statistical Tools for Causal Inference}
    \pretitle{\vspace{\droptitle}\centering\huge}
  \posttitle{\par}
    \author{The SKY Community}
    \preauthor{\centering\large\emph}
  \postauthor{\par}
      \predate{\centering\large\emph}
  \postdate{\par}
    \date{2019-08-23}

\usepackage{dsfont}
\newcommand{\uns}[1]{\mathds{1}[ #1 ]}
\usepackage{subfig}
\newcommand{\esp}[1]{\mathbb{E}[ #1 ]}
\newcommand\Ind{\protect\mathpalette{\protect\independenT}{\perp}}
\def\independenT#1#2{\mathrel{\setbox0\hbox{$#1#2$}\copy0\kern-\wd0\mkern4mu\box0}}
\newcommand{\var}[1]{\mathbb{V}[ #1 ]}

\usepackage{amsthm}
\newtheorem{theorem}{Theorem}[chapter]
\newtheorem{lemma}{Lemma}[chapter]
\newtheorem{corollary}{Corollary}[chapter]
\newtheorem{proposition}{Proposition}[chapter]
\newtheorem{conjecture}{Conjecture}[chapter]
\theoremstyle{definition}
\newtheorem{definition}{Definition}[chapter]
\theoremstyle{definition}
\newtheorem{example}{Example}[chapter]
\theoremstyle{definition}
\newtheorem{exercise}{Exercise}[chapter]
\theoremstyle{remark}
\newtheorem*{remark}{Remark}
\newtheorem*{solution}{Solution}
\let\BeginKnitrBlock\begin \let\EndKnitrBlock\end
\begin{document}
\maketitle

{
\setcounter{tocdepth}{1}
\tableofcontents
}
\chapter*{Introduction}\label{introduction}
\addcontentsline{toc}{chapter}{Introduction}

Tools of causal inference are the basic statistical building block
behind most scientific results. It is thus extremely useful to have an
open source collectively aggreed upon resource presenting and assessing
them, as well as listing the current unresolved issues. The content of
this book covers the basic theoretical knowledge and technical skills
required for implementing staistical methods of causal inference. This
means:

\begin{itemize}
\tightlist
\item
  Understanding of the basic language to encode causality,
\item
  Knowledge of the fundamental problems of inference and the biases of
  intuitive estimators,
\item
  Understanding of how econometric methods recover treatment effects,
\item
  Ability to compute these estimators along with an estimate of their
  precision using the statistical software R.
\end{itemize}

This book is geared for teaching causal inference to graduate students
that want to apply statistical tools of causal inference. The
demonstration of theoretical results are provided, but the final goal is
not to have students reproduce them, but mostly to enable them to grasp
a better understanding of the fundations for the tools that they will be
using. The focus is on understanding the issues and solutions more than
understanding the maths that are behind, even though the maths are there
and are used to convey the notions rigorously. All the notions and
estimators are introduced using a numerical example and simulations, so
that each notion is illustrated and appears more intuitive to the
students. The second version of this book will contain examples using
real applications. The third version will contain exercises.

This book is written in Rmarkdown using the bookdown package. It is
available both as a \href{https://chabefer.github.io/STCI/}{web-book}
and as a \href{https://chabefer.github.io/STCI/STCI.pdf}{pdf book}.

This book is a collaborative effort that is part of the
\href{https://chabefer.github.io/SKY/}{Social Science Knowledge
Accumulation Initiative (SKY)}. The code behind this book is publically
available on GitHub and you can propose corrections and updates. How to
make contributions to this book is explained on the
\href{https://chabefer.github.io/SKY/tutoSTCI.html}{SKY website}. Do not
hesitate to make suggestions, modifications and extensions. This way
this book will grow and become the living open source collaborative
reference for methodological work that it could be.

\part{The Two Fundamental Problems of
Inference}\label{part-the-two-fundamental-problems-of-inference}

\chapter*{Introduction: the Two Fundamental Problems of
Inference}\label{introduction-the-two-fundamental-problems-of-inference}
\addcontentsline{toc}{chapter}{Introduction: the Two Fundamental
Problems of Inference}

When trying to estimate the effect of a program on an outcome, we face
two very important and difficult problems: \href{FPCI.html}{the
Fundamental Problem of Causal Inference (FPCI)} and \href{FPSI.html}{the
Fundamental Problem of Statistical Inference (FPSI)}.

In its most basic form, the FPCI states that our causal parameter of
interest (\(TT\), short for Treatment on the Treated, that we will
define shortly) is fundamentally unobservable, even when the sample size
is infinite. The main reason for that is that one component of \(TT\),
the outcome of the treated had they not received the program, remains
unobservable. We call this outcome a counterfactual outcome. The FPCI is
a very dispiriting result, and is actually the basis for all of the
statistical methods of causal inference. All of these methods try to
find ways to estimate the counterfactual by using observable quantities
that hopefully approximate it as well as possible. Most people,
including us but also policymakers, generally rely on intuitive
quantities in order to generate the counterfactual (the individuals
without the program or the individuals before the program was
implemented). Unfortunately, these approximations are generally very
crude, and the resulting estimators of \(TT\) are generally biased,
sometimes severely.

The Fundamental Problem of Statistical Inference (FPSI) states that,
even if we have an estimator \(E\) that identifies \(TT\) in the
population, we cannot observe \(E\) because we only have access to a
finite sample of the population. The only thing that we can form from
the sample is a sample equivalent \(\hat{E}\) to the population quantity
\(E\), and \(\hat{E}\neq E\). Why is \(\hat{E}\neq E\)? Because a finite
sample is never perfectly representative of the population. What can we
do to deal with the FPSI? I am going to argue that there are mainly two
things that we might want to do: estimating the extent of sampling noise
and decreasing sampling noise.

\chapter{The Fundamental Problem of Causal
Inference}\label{the-fundamental-problem-of-causal-inference}

In order to state the FPCI, we are going to describe the basic language
to encode causality set up by Rubin, and named \href{RCM.html}{Rubin
Causal Model (RCM)}. RCM being about partly observed random variables,
it is hard to make these notions concrete with real data. That's why we
are going to use simulations from a simple model in order to make it
clear how these variables are generated. The second virtue of this model
is that it is going to make it clear the source of selection into the
treatment. This is going to be useful when understanding biases of
intuitive comparisons, but also to discuss the methods of causal
inference. A third virtue of this approach is that it makes clear the
connexion between the treatment effects literature and models. Finally,
a fourth reason that it is useful is that it is going to give us a
source of sampling variation that we are going to use to visualize and
explore the properties of our estimators.

I use \(X_i\) to denote random variable \(X\) all along the notes. I
assume that we have access to a sample of \(N\) observations indexed by
\(i\in\left\{1,\dots,N\right\}\). `'\(i\)'' will denote the basic
sampling units when we are in a sample, and a basic element of the
probability space when we are in populations. Introducing rigorous
measure-theoretic notations for the population is feasible but is not
necessary for comprehension.

When the sample size is infinite, we say that we have a population. A
population is a very useful fiction for two reasons. First, in a
population, there is no sampling noise: we observe an infinite amount of
observations, and our estimators are infinitely precise. This is useful
to study phenomena independently of sampling noise. For example, it is
in general easier to prove that an estimator is equal to \(TT\) under
some conditions in the population. Second, we are most of the time much
more interested in estimating the values of parameters in the population
rather than in the sample. The population parameter, independent of
sampling noise, gives a much better idea of the causal parameter for the
population of interest than the parameter in the sample. In general, the
estimator for both quantities will be the same, but the estimators for
the effetc of sampling noise on these estimators will differ. Sampling
noise for the population parameter will generally be larger, since it is
affected by another source of variability (sample choice).

\section{Rubin Causal Model}\label{rubin-causal-model}

The RCM is made of three distinct building blocks: a treatment
allocation rule, that decides who receives the treatment; potential
outcomes, that measure how each individual reacts to the treatment; the
switching equation that relates potential outcomes to observed outcomes
through the allocation rule.

\subsection{Treatment allocation rule}\label{treatment-allocation-rule}

The first building block of the RCM is the treatment allocation rule.
Throughout this class, we are going to be interested in inferring the
causal effect of only one treatment with respect to a control condition.
Extensions to multi-valued treatments are in general self-explanatory.

In the RCM, treatment allocation is captured by the variable \(D_i\).
\(D_i=1\) if unit \(i\) receives the treatment and \(D_i=0\) if unit
\(i\) does not receive the treatment and thus remains in the control
condition.

The treatment allocation rule is critical for several reasons. First,
because it switches the treatment on or off for each unit, it is going
to be at the source of the FPCI. Second, the specific properties of the
treatment allocatoin rule are going to matter for the feasibility and
bias of the various econometric methods that we are going to study.

Let's take a few examples of allocation rules. These allocation rules
are just examples. They do not cover the space of all possible
allocation rules. They are especially useful as concrete devices to
understand the sources of biases and the nature of the allocation rule.
In reality, there exists even more complex allocation rules (awareness,
eligibility, application, acceptance, active participation). Awareness
seems especially important for program participation and has only been
tackled recently by economists.

First, some notation. Let's imagine a treatment that is given to
individuals. Whether each individual receives the treatment partly
depends on the level of her outcome before receiving the treatment.
Let's denote this variable \(Y^B_i\), with \(B\) standing for
``Before''. It can be the health status assessed by a professional
before deciding to give a drug to a patient. It can be the poverty level
of a household used to assess its eligibilty to a cash transfer program.

\subsubsection{Sharp cutoff rule}\label{sharp-cutoff-rule}

The sharp cutoff rule means that everyone below some threshold
\(\bar{Y}\) is going to receive the treatment. Everyone whose outcome
before the treatment lies above \(\bar{Y}\) does not receive the
treatment. Such rules can be found in reality in a lot of situations.
They might be generated by administrative rules. One very simple way to
model this rule is as follows:

\begin{align}\label{eq:cutoff}
  D_i & = \uns{Y_i^B\leq\bar{Y}},
\end{align}

where \(\uns{A}\) is the indicator function, taking value \(1\) when
\(A\) is true and \(0\) otherwise.

\BeginKnitrBlock{example}[Sharp cutoff rule]
\protect\hypertarget{exm:unnamed-chunk-1}{}{\label{exm:unnamed-chunk-1}
\iffalse (Sharp cutoff rule) \fi{} }Imagine that \(Y_i^B=\exp(y_i^B)\),
with \(y_i^B=\mu_i+U_i^B\),
\(\mu_i\sim\mathcal{N}(\bar{\mu},\sigma^2_{\mu})\) and
\(U_i^B\sim\mathcal{N}(0,\sigma^2_{U})\). Now, let's choose some values
for these parameters so that we can generate a sample of individuals and
allocate the treatment among them. I'm going to switch to R for that.
\EndKnitrBlock{example}

\begin{Shaded}
\begin{Highlighting}[]
\NormalTok{param <-}\StringTok{ }\KeywordTok{c}\NormalTok{(}\DecValTok{8}\NormalTok{,.}\DecValTok{5}\NormalTok{,.}\DecValTok{28}\NormalTok{,}\DecValTok{1500}\NormalTok{)}
\KeywordTok{names}\NormalTok{(param) <-}\StringTok{ }\KeywordTok{c}\NormalTok{(}\StringTok{"barmu"}\NormalTok{,}\StringTok{"sigma2mu"}\NormalTok{,}\StringTok{"sigma2U"}\NormalTok{,}\StringTok{"barY"}\NormalTok{)}
\NormalTok{param}
\end{Highlighting}
\end{Shaded}

\begin{verbatim}
##    barmu sigma2mu  sigma2U     barY 
##     8.00     0.50     0.28  1500.00
\end{verbatim}

Now, I have choosen values for the parameters in my model. For example,
\(\bar{\mu}=\) 8 and \(\bar{Y}=\) 1500. What remains to be done is to
generate \(Y_i^B\) and then \(D_i\). For this, I have to choose a sample
size (\(N=1000\)) and then generate the shocks from a normal.

\begin{Shaded}
\begin{Highlighting}[]
\CommentTok{# for reproducibility, I choose a seed that will give me the same random sample each time I run the program}
\KeywordTok{set.seed}\NormalTok{(}\DecValTok{1234}\NormalTok{)}
\NormalTok{N <-}\DecValTok{1000}
\NormalTok{mu <-}\StringTok{ }\KeywordTok{rnorm}\NormalTok{(N,param[}\StringTok{"barmu"}\NormalTok{],}\KeywordTok{sqrt}\NormalTok{(param[}\StringTok{"sigma2mu"}\NormalTok{]))}
\NormalTok{UB <-}\StringTok{ }\KeywordTok{rnorm}\NormalTok{(N,}\DecValTok{0}\NormalTok{,}\KeywordTok{sqrt}\NormalTok{(param[}\StringTok{"sigma2U"}\NormalTok{]))}
\NormalTok{yB <-}\StringTok{ }\NormalTok{mu }\OperatorTok{+}\StringTok{ }\NormalTok{UB }
\NormalTok{YB <-}\StringTok{ }\KeywordTok{exp}\NormalTok{(yB)}
\NormalTok{Ds <-}\StringTok{ }\KeywordTok{ifelse}\NormalTok{(YB}\OperatorTok{<=}\NormalTok{param[}\StringTok{"barY"}\NormalTok{],}\DecValTok{1}\NormalTok{,}\DecValTok{0}\NormalTok{) }
\end{Highlighting}
\end{Shaded}

Let's now build a histogram of the data that we have just generated.

\begin{Shaded}
\begin{Highlighting}[]
\CommentTok{# building histogram of yB with cutoff point at ybar}
\CommentTok{# Number of steps}
\NormalTok{Nsteps.}\DecValTok{1}\NormalTok{ <-}\StringTok{ }\DecValTok{15}
\CommentTok{#step width}
\NormalTok{step.}\DecValTok{1}\NormalTok{ <-}\StringTok{ }\NormalTok{(}\KeywordTok{log}\NormalTok{(param[}\StringTok{"barY"}\NormalTok{])}\OperatorTok{-}\KeywordTok{min}\NormalTok{(yB[Ds}\OperatorTok{==}\DecValTok{1}\NormalTok{]))}\OperatorTok{/}\NormalTok{Nsteps.}\DecValTok{1}
\NormalTok{Nsteps.}\DecValTok{0}\NormalTok{ <-}\StringTok{ }\NormalTok{(}\OperatorTok{-}\KeywordTok{log}\NormalTok{(param[}\StringTok{"barY"}\NormalTok{])}\OperatorTok{+}\KeywordTok{max}\NormalTok{(yB[Ds}\OperatorTok{==}\DecValTok{0}\NormalTok{]))}\OperatorTok{/}\NormalTok{step.}\DecValTok{1}
\NormalTok{breaks <-}\StringTok{ }\KeywordTok{cumsum}\NormalTok{(}\KeywordTok{c}\NormalTok{(}\KeywordTok{min}\NormalTok{(yB[Ds}\OperatorTok{==}\DecValTok{1}\NormalTok{]),}\KeywordTok{c}\NormalTok{(}\KeywordTok{rep}\NormalTok{(step.}\DecValTok{1}\NormalTok{,Nsteps.}\DecValTok{1}\OperatorTok{+}\NormalTok{Nsteps.}\DecValTok{0}\OperatorTok{+}\DecValTok{1}\NormalTok{))))}
\KeywordTok{hist}\NormalTok{(yB,}\DataTypeTok{breaks=}\NormalTok{breaks,}\DataTypeTok{main=}\StringTok{""}\NormalTok{)}
\KeywordTok{abline}\NormalTok{(}\DataTypeTok{v=}\KeywordTok{log}\NormalTok{(param[}\StringTok{"barY"}\NormalTok{]),}\DataTypeTok{col=}\StringTok{"red"}\NormalTok{)}
\end{Highlighting}
\end{Shaded}

\begin{figure}

{\centering \includegraphics[width=0.6\linewidth]{STCI_files/figure-latex/histyb-1} 

}

\caption{Histogram of $y_B$}\label{fig:histyb}
\end{figure}

You can see on Figure \ref{fig:histyb} a histogram of \(y_i^B\) with the
red line indicating the cutoff point: \(\bar{y}=\ln(\bar{Y})=\) 7.3. All
the observations below the red line are treated according to the sharp
rule while all the one located above are not. In order to see how many
observations eventually receive the treatment with this allocation rule,
let's build a contingency table.

\begin{Shaded}
\begin{Highlighting}[]
\NormalTok{table.D.sharp <-}\StringTok{ }\KeywordTok{as.matrix}\NormalTok{(}\KeywordTok{table}\NormalTok{(Ds))}
\NormalTok{knitr}\OperatorTok{::}\KeywordTok{kable}\NormalTok{(table.D.sharp,}\DataTypeTok{caption=}\StringTok{'Treatment allocation with sharp cutoff rule'}\NormalTok{,}\DataTypeTok{booktabs=}\OtherTok{TRUE}\NormalTok{)}
\end{Highlighting}
\end{Shaded}

\begin{table}[t]

\caption{\label{tab:tableDsharp}Treatment allocation with sharp cutoff rule}
\centering
\begin{tabular}{lr}
\toprule
0 & 771\\
1 & 229\\
\bottomrule
\end{tabular}
\end{table}

We can see on Table \ref{tab:tableDsharp} that there are 229 treated
observations.

\subsubsection{Fuzzy cutoff rule}\label{fuzzy-cutoff-rule}

This rule is less sharp than the sharp cutoff rule. Here, other criteria
than \(Y_i^B\) enter into the decision to allocate the treatment. The
doctor might measure the health status of a patient following official
guidelines, but he might also measure other factors that will also
influence his decision of giving the drug to the patient. The officials
administering a program might measure the official income level of a
household, but they might also consider other features of the household
situation when deciding to enroll the household into the program or not.
If these additional criteria are unobserved to the econometrician, then
we have the fuzzy cutoff rule. A very simple way to model this rule is
as follows:

\begin{align}\label{eq:fuzzcutoff}
  D_i & = \uns{Y_i^B+V_i\leq\bar{Y}},
\end{align}

where \(V_i\) is a random variable unobserved to the econometrician and
standing for the other influences that might drive the allocation of the
treatment. \(V_i\) is distributed according to a, for the moment,
unspecified cumulative distribution function \(F_V\). When \(V_i\) is
degenerate (\textit{i.e.} it has only one point of support: it is a
constant), the fuzzy cutoff rule becomes the sharp cutoff rule.

\subsubsection{\texorpdfstring{Eligibility \(+\) self-selection
rule}{Eligibility + self-selection rule}}\label{eligibility-self-selection-rule}

It is also possible that households, once they have been made eligible
to the treatment, can decide whether they want to receive it or not. A
patient might be able to refuse the drug that the doctor suggests she
should take. A household might refuse to participate in a cash transfer
program to which it has been made eligible. Not all programs have this
feature, but most of them have some room for decisions by the agents
themselves of whether they want to receive the treatment or not. One
simple way to model this rule is as follows:

\begin{align}\label{eq:eligself}
  D_i & = \uns{D^*_i\geq0}E_i,
\end{align}

where \(D^*_i\) is individual \(i\)'s valuation of the treatment and
\(E_i\) is whether or not she is deemed eligible for the treatment.
\(E_i\) might be choosen according to the sharp cutoff rule of to the
fuzzy cutoff rule, or to any other eligibility rule. We will be more
explicit about \(D_i^*\) in what follows.

\textbf{SIMULATIONS ARE MISSING FOR THESE LAST TWO RULES}

\subsection{Potential outcomes}\label{potential-outcomes}

The second main building block of the RCM are potential outcomes. Let's
say that we are interested in the effect of a treatment on an outcome
\(Y\). Each unit \(i\) can thus be in two potential states: treated or
non treated. Before the allocation of the treatment is decided, both of
these states are feasible for each unit.

\BeginKnitrBlock{definition}[Potential outcomes]
\protect\hypertarget{def:unnamed-chunk-2}{}{\label{def:unnamed-chunk-2}
\iffalse (Potential outcomes) \fi{} }For each unit \(i\), we define two
potential outcomes:
\EndKnitrBlock{definition}

\begin{itemize}
\tightlist
\item
  \(Y_i^1\): the outcome that unit \(i\) is going to have if it receives
  the treatment,
\item
  \(Y_i^0\): the outcome that unit \(i\) is going to have if it
  \textbf{does not} receive the treatment.
\end{itemize}

\BeginKnitrBlock{example}
\protect\hypertarget{exm:unnamed-chunk-3}{}{\label{exm:unnamed-chunk-3}
}Let's choose functional forms for our potential outcomes. For
simplicity, all lower case letters will denote log outcomes.
\(y_i^0=\mu_i+\delta+U_i^0\), with \(\delta\) a time shock common to all
the observations and \(U_i^0=\rho U_i^B+\epsilon_i\), with \(|\rho|<1\).
In the absence of the treatment, part of the shocks \(U_i^B\) that the
individuals experienced in the previous period persist, while some part
vanish. \(y_i^1=y_i^0+\bar{\alpha}+\theta\mu_i+\eta_i\). In order to
generate the potential outcomes, one has to define the laws for the
shocks and to choose parameter values. Let's assume that
\(\epsilon_i\sim\mathcal{N}(0,\sigma^2_{\epsilon})\) and
\(\eta_i\sim\mathcal{N}(0,\sigma^2_{\eta})\). Now let's choose some
parameter values:
\EndKnitrBlock{example}

\begin{Shaded}
\begin{Highlighting}[]
\NormalTok{l <-}\StringTok{ }\KeywordTok{length}\NormalTok{(param)}
\NormalTok{param <-}\StringTok{ }\KeywordTok{c}\NormalTok{(param,}\FloatTok{0.9}\NormalTok{,}\FloatTok{0.01}\NormalTok{,}\FloatTok{0.05}\NormalTok{,}\FloatTok{0.05}\NormalTok{,}\FloatTok{0.05}\NormalTok{,}\FloatTok{0.1}\NormalTok{)}
\KeywordTok{names}\NormalTok{(param)[(l}\OperatorTok{+}\DecValTok{1}\NormalTok{)}\OperatorTok{:}\KeywordTok{length}\NormalTok{(param)] <-}\StringTok{ }\KeywordTok{c}\NormalTok{(}\StringTok{"rho"}\NormalTok{,}\StringTok{"theta"}\NormalTok{,}\StringTok{"sigma2epsilon"}\NormalTok{,}\StringTok{"sigma2eta"}\NormalTok{,}\StringTok{"delta"}\NormalTok{,}\StringTok{"baralpha"}\NormalTok{)}
\NormalTok{param}
\end{Highlighting}
\end{Shaded}

\begin{verbatim}
##         barmu      sigma2mu       sigma2U          barY           rho 
##          8.00          0.50          0.28       1500.00          0.90 
##         theta sigma2epsilon     sigma2eta         delta      baralpha 
##          0.01          0.05          0.05          0.05          0.10
\end{verbatim}

We can finally generate the potential outcomes;

\begin{Shaded}
\begin{Highlighting}[]
\NormalTok{epsilon <-}\StringTok{ }\KeywordTok{rnorm}\NormalTok{(N,}\DecValTok{0}\NormalTok{,}\KeywordTok{sqrt}\NormalTok{(param[}\StringTok{"sigma2epsilon"}\NormalTok{]))}
\NormalTok{eta<-}\StringTok{ }\KeywordTok{rnorm}\NormalTok{(N,}\DecValTok{0}\NormalTok{,}\KeywordTok{sqrt}\NormalTok{(param[}\StringTok{"sigma2eta"}\NormalTok{]))}
\NormalTok{U0 <-}\StringTok{ }\NormalTok{param[}\StringTok{"rho"}\NormalTok{]}\OperatorTok{*}\NormalTok{UB }\OperatorTok{+}\StringTok{ }\NormalTok{epsilon}
\NormalTok{y0 <-}\StringTok{ }\NormalTok{mu }\OperatorTok{+}\StringTok{  }\NormalTok{U0 }\OperatorTok{+}\StringTok{ }\NormalTok{param[}\StringTok{"delta"}\NormalTok{]}
\NormalTok{alpha <-}\StringTok{ }\NormalTok{param[}\StringTok{"baralpha"}\NormalTok{]}\OperatorTok{+}\StringTok{  }\NormalTok{param[}\StringTok{"theta"}\NormalTok{]}\OperatorTok{*}\NormalTok{mu }\OperatorTok{+}\StringTok{ }\NormalTok{eta}
\NormalTok{y1 <-}\StringTok{ }\NormalTok{y0}\OperatorTok{+}\NormalTok{alpha}
\NormalTok{Y0 <-}\StringTok{ }\KeywordTok{exp}\NormalTok{(y0)}
\NormalTok{Y1 <-}\StringTok{ }\KeywordTok{exp}\NormalTok{(y1)}
\end{Highlighting}
\end{Shaded}

Now, I would like to visualize my potential outcomes:

\begin{Shaded}
\begin{Highlighting}[]
\KeywordTok{plot}\NormalTok{(y0,y1)}
\end{Highlighting}
\end{Shaded}

\begin{figure}

{\centering \includegraphics[width=0.6\linewidth]{STCI_files/figure-latex/histpotout-1} 

}

\caption{Potential outcomes}\label{fig:histpotout}
\end{figure}

You can see on the resulting Figure \ref{fig:histpotout} that both
potential outcomes are positively correlated. Those with a large
potential outcome when untreated (\emph{e.g.} in good health without the
treatment) also have a positive health with the treatment. It is also
true that individuals with bad health in the absence of the treatment
also have bad health with the treatment.

\subsection{Switching equation}\label{switching-equation}

The last building block of the RCM is the switching equation. It links
the observed outcome to the potential outcomes through the allocation
rule:

\begin{align}
 \label{eq:switch}
  Y_i & = 
    \begin{cases}
    Y_i^1 & \text{if } D_i=1\\
    Y_i^0 & \text{if } D_i=0
    \end{cases} \\
    & = Y_i^1D_i + Y_i^0(1-D_i) \nonumber
\end{align}

\BeginKnitrBlock{example}
\protect\hypertarget{exm:unnamed-chunk-4}{}{\label{exm:unnamed-chunk-4} }In
order to generate observed outcomes in our numerical example, we simply
have to enforce the switching equation:
\EndKnitrBlock{example}

\begin{Shaded}
\begin{Highlighting}[]
\NormalTok{y <-}\StringTok{ }\NormalTok{y1}\OperatorTok{*}\NormalTok{Ds}\OperatorTok{+}\NormalTok{y0}\OperatorTok{*}\NormalTok{(}\DecValTok{1}\OperatorTok{-}\NormalTok{Ds)}
\NormalTok{Y <-}\StringTok{ }\NormalTok{Y1}\OperatorTok{*}\NormalTok{Ds}\OperatorTok{+}\NormalTok{Y0}\OperatorTok{*}\NormalTok{(}\DecValTok{1}\OperatorTok{-}\NormalTok{Ds)}
\end{Highlighting}
\end{Shaded}

What the switching equation \eqref{eq:switch} means is that, for each
individual \(i\), we get to observe only one of the two potential
outcomes. When individual \(i\) belongs to the treatment group
(\emph{i.e.} \(D_i=1\)), we get to observe \(Y_i^1\). When individual
\(i\) belongs to the control group (\emph{i.e.} \(D_i=0\)), we get to
observe \(Y_i^0\). Because the same individual cannot be at the same
time in both groups, we can NEVER see both potential outcomes for the
same individual at the same time.

For each of the individuals, one of the two potential outcomes is
unobserved. We say that it is a \textbf{counterfactual}. A
counterfactual quantity is a quantity that is, according to Hume's
definition, contrary to the observed facts. A counterfactual cannot be
observed, but it can be conceived by an effort of reason: it is the
consequence of what would have happened had some action not been taken.

\BeginKnitrBlock{remark}
\iffalse{} {Remark. } \fi{}One very nice way of visualising the
switching equation has been proposed by Jerzy Neyman in a 1923 prescient
paper. Neyman proposes to imagine two urns, each one filled with \(N\)
balls. One urn is the treatment urn and contains balls with the id of
the unit and the value of its potential outcome \(Y_i^1\). The other urn
is the control urn, and it contains balls with the value of the
potential outcome \(Y_i^0\) for each unit \(i\). Following the
allocation rule \(D_i\), we decide whether unit \(i\) is in the
treatment or control group. When unit \(i\) is in the treatment group,
we take the corresponding ball from the first urn and observe the
potential outcome on it. But, at the same time, the urns are connected
so that the corresponding ball with the potential outcome of unit \(i\)
in the control urn disappears as soon as we draw ball \(i\) from the
treatment urn.

The switching equation works a lot like Schrodinger's cat paradox.
Schrodinger's cat is placed in a sealed box and receives a dose of
poison when an atom emits a radiation. As long as the box is sealed,
there is no way we can know whether the cat is dead or alive. When we
open the box, we observe either a dead cat or a living cat, but we
cannot observe the cat both alive and dead at the same time. The
switching equation is like opening the box, it collapses the observed
outcome into one of the two potential ones.
\EndKnitrBlock{remark}

\BeginKnitrBlock{example}
\protect\hypertarget{exm:unnamed-chunk-6}{}{\label{exm:unnamed-chunk-6} }One
way to visualize the inner workings of the switching equation is to plot
the potential outcomes along with the criteria driving the allocation
rule. In our simple example, it simply amounts to plotting observed
(\(y_i\)) and potential outcomes (\(y_i^1\) and \(y_i^0\)) along
\(y_i^B\).
\EndKnitrBlock{example}

\begin{Shaded}
\begin{Highlighting}[]
\KeywordTok{plot}\NormalTok{(yB[Ds}\OperatorTok{==}\DecValTok{0}\NormalTok{],y0[Ds}\OperatorTok{==}\DecValTok{0}\NormalTok{],}\DataTypeTok{pch=}\DecValTok{1}\NormalTok{,}\DataTypeTok{xlim=}\KeywordTok{c}\NormalTok{(}\DecValTok{5}\NormalTok{,}\DecValTok{11}\NormalTok{),}\DataTypeTok{ylim=}\KeywordTok{c}\NormalTok{(}\DecValTok{5}\NormalTok{,}\DecValTok{11}\NormalTok{),}\DataTypeTok{xlab=}\StringTok{"yB"}\NormalTok{,}\DataTypeTok{ylab=}\StringTok{"Outcomes"}\NormalTok{)}
\KeywordTok{points}\NormalTok{(yB[Ds}\OperatorTok{==}\DecValTok{1}\NormalTok{],y1[Ds}\OperatorTok{==}\DecValTok{1}\NormalTok{],}\DataTypeTok{pch=}\DecValTok{3}\NormalTok{)}
\KeywordTok{points}\NormalTok{(yB[Ds}\OperatorTok{==}\DecValTok{0}\NormalTok{],y1[Ds}\OperatorTok{==}\DecValTok{0}\NormalTok{],}\DataTypeTok{pch=}\DecValTok{3}\NormalTok{,}\DataTypeTok{col=}\StringTok{'red'}\NormalTok{)}
\KeywordTok{points}\NormalTok{(yB[Ds}\OperatorTok{==}\DecValTok{1}\NormalTok{],y0[Ds}\OperatorTok{==}\DecValTok{1}\NormalTok{],}\DataTypeTok{pch=}\DecValTok{1}\NormalTok{,}\DataTypeTok{col=}\StringTok{'red'}\NormalTok{)}
\NormalTok{test <-}\StringTok{ }\FloatTok{5.8}
\NormalTok{i.test <-}\StringTok{ }\KeywordTok{which}\NormalTok{(}\KeywordTok{abs}\NormalTok{(yB}\OperatorTok{-}\NormalTok{test)}\OperatorTok{==}\KeywordTok{min}\NormalTok{(}\KeywordTok{abs}\NormalTok{(yB}\OperatorTok{-}\NormalTok{test)))}
\KeywordTok{points}\NormalTok{(yB[}\KeywordTok{abs}\NormalTok{(yB}\OperatorTok{-}\NormalTok{test)}\OperatorTok{==}\KeywordTok{min}\NormalTok{(}\KeywordTok{abs}\NormalTok{(yB}\OperatorTok{-}\NormalTok{test))],y1[}\KeywordTok{abs}\NormalTok{(yB}\OperatorTok{-}\NormalTok{test)}\OperatorTok{==}\KeywordTok{min}\NormalTok{(}\KeywordTok{abs}\NormalTok{(yB}\OperatorTok{-}\NormalTok{test))],}\DataTypeTok{col=}\StringTok{'green'}\NormalTok{,}\DataTypeTok{pch=}\DecValTok{3}\NormalTok{)}
\KeywordTok{points}\NormalTok{(yB[}\KeywordTok{abs}\NormalTok{(yB}\OperatorTok{-}\NormalTok{test)}\OperatorTok{==}\KeywordTok{min}\NormalTok{(}\KeywordTok{abs}\NormalTok{(yB}\OperatorTok{-}\NormalTok{test))],y0[}\KeywordTok{abs}\NormalTok{(yB}\OperatorTok{-}\NormalTok{test)}\OperatorTok{==}\KeywordTok{min}\NormalTok{(}\KeywordTok{abs}\NormalTok{(yB}\OperatorTok{-}\NormalTok{test))],}\DataTypeTok{col=}\StringTok{'green'}\NormalTok{)}
\KeywordTok{abline}\NormalTok{(}\DataTypeTok{v=}\KeywordTok{log}\NormalTok{(param[}\StringTok{"barY"}\NormalTok{]),}\DataTypeTok{col=}\StringTok{"red"}\NormalTok{)}
\KeywordTok{legend}\NormalTok{(}\DecValTok{5}\NormalTok{,}\DecValTok{11}\NormalTok{,}\KeywordTok{c}\NormalTok{(}\StringTok{'y0|D=0'}\NormalTok{,}\StringTok{'y1|D=1'}\NormalTok{,}\StringTok{'y0|D=1'}\NormalTok{,}\StringTok{'y1|D=0'}\NormalTok{,}\KeywordTok{paste}\NormalTok{(}\StringTok{'y0'}\NormalTok{,i.test,}\DataTypeTok{sep=}\StringTok{''}\NormalTok{),}\KeywordTok{paste}\NormalTok{(}\StringTok{'y1'}\NormalTok{,i.test,}\DataTypeTok{sep=}\StringTok{''}\NormalTok{)),}\DataTypeTok{pch=}\KeywordTok{c}\NormalTok{(}\DecValTok{1}\NormalTok{,}\DecValTok{3}\NormalTok{,}\DecValTok{1}\NormalTok{,}\DecValTok{3}\NormalTok{,}\DecValTok{1}\NormalTok{,}\DecValTok{3}\NormalTok{),}\DataTypeTok{col=}\KeywordTok{c}\NormalTok{(}\StringTok{'black'}\NormalTok{,}\StringTok{'black'}\NormalTok{,}\StringTok{'red'}\NormalTok{,}\StringTok{'red'}\NormalTok{,}\StringTok{'green'}\NormalTok{,}\StringTok{'green'}\NormalTok{),}\DataTypeTok{ncol=}\DecValTok{3}\NormalTok{)}
\end{Highlighting}
\end{Shaded}

\begin{figure}

{\centering \includegraphics[width=0.6\linewidth]{STCI_files/figure-latex/ploty1y0yB-1} 

}

\caption{Potential outcomes}\label{fig:ploty1y0yB}
\end{figure}

\begin{Shaded}
\begin{Highlighting}[]
\KeywordTok{plot}\NormalTok{(yB[Ds}\OperatorTok{==}\DecValTok{0}\NormalTok{],y0[Ds}\OperatorTok{==}\DecValTok{0}\NormalTok{],}\DataTypeTok{pch=}\DecValTok{1}\NormalTok{,}\DataTypeTok{xlim=}\KeywordTok{c}\NormalTok{(}\DecValTok{5}\NormalTok{,}\DecValTok{11}\NormalTok{),}\DataTypeTok{ylim=}\KeywordTok{c}\NormalTok{(}\DecValTok{5}\NormalTok{,}\DecValTok{11}\NormalTok{),}\DataTypeTok{xlab=}\StringTok{"yB"}\NormalTok{,}\DataTypeTok{ylab=}\StringTok{"Outcomes"}\NormalTok{)}
\KeywordTok{points}\NormalTok{(yB[Ds}\OperatorTok{==}\DecValTok{1}\NormalTok{],y1[Ds}\OperatorTok{==}\DecValTok{1}\NormalTok{],}\DataTypeTok{pch=}\DecValTok{3}\NormalTok{)}
\KeywordTok{legend}\NormalTok{(}\DecValTok{5}\NormalTok{,}\DecValTok{11}\NormalTok{,}\KeywordTok{c}\NormalTok{(}\StringTok{'y|D=0'}\NormalTok{,}\StringTok{'y|D=1'}\NormalTok{),}\DataTypeTok{pch=}\KeywordTok{c}\NormalTok{(}\DecValTok{1}\NormalTok{,}\DecValTok{3}\NormalTok{))}
\KeywordTok{abline}\NormalTok{(}\DataTypeTok{v=}\KeywordTok{log}\NormalTok{(param[}\StringTok{"barY"}\NormalTok{]),}\DataTypeTok{col=}\StringTok{"red"}\NormalTok{)}
\end{Highlighting}
\end{Shaded}

\begin{figure}

{\centering \includegraphics[width=0.6\linewidth]{STCI_files/figure-latex/plotyyB-1} 

}

\caption{Observed outcomes}\label{fig:plotyyB}
\end{figure}

Figure \ref{fig:ploty1y0yB} plots the observed outcomes \(y_i\) along
with the unobserved potential outcomes. Figure \ref{fig:ploty1y0yB}
shows that each individual in the sample is endowed with two potential
outcomes, represented by a circle and a cross. Figure \ref{fig:plotyyB}
plots the observed outcomes \(y_i\) that results from applying the
switching equation. Only one of the two potential outcomes is observed
(the cross for the treated group and the circle for the untreated group)
and the other is not. The observed sample in Figure \ref{fig:plotyyB}
only shows observed outcomes, and is thus silent on the values of the
missing potential outcomes.

\section{Treatment effects}\label{treatment-effects}

The RCM enables the definition of causal effects at the individual
level. In practice though, we generally focus on a summary measure: the
effect of the treatment on the treated.

\subsection{Individual level treatment
effects}\label{individual-level-treatment-effects}

Potential outcomes enable us to define the central notion of causal
inference: the causal effect, also labelled the treatment effect, which
is the difference between the two potential outcomes.

\BeginKnitrBlock{definition}[Individual level treatment effect]
\protect\hypertarget{def:causaleff}{}{\label{def:causaleff}
\iffalse (Individual level treatment effect) \fi{} }For each unit \(i\),
the causal effect of the treatment on outcome \(Y\) is:
\(\Delta^Y_i=Y_i^1-Y_i^0\).
\EndKnitrBlock{definition}

\BeginKnitrBlock{example}
\protect\hypertarget{exm:unnamed-chunk-7}{}{\label{exm:unnamed-chunk-7} }The
individual level causal effect in log terms is:
\(\Delta^y_i=\alpha_i=\bar{\alpha}+\theta\mu_i+\eta_i\). The effect is
the sum of a part common to all individuals, a part correlated with
\(\mu_i\): the treatment might have a larger or a smaller effect
depending on the unobserved permanent ability or health status of
individuals, and a random shock. It is possible to make the effect of
the treatment to depend on \(U_i^B\) also, but it would complicate the
model.
\EndKnitrBlock{example}

In Figure \ref{fig:ploty1y0yB}, the individual level treatment effects
are the differences between each cross and its corresponding circle. For
example, for observation 264, the two potential outcomes appear in green
in Figure \ref{fig:ploty1y0yB}. The effect of the treatment on unit 264
is equal to: \[
\Delta^y_{264}=y^1_{264}-y^0_{264}=6.98-6.64=0.34.
\]

Since observation 264 belongs to the treatment group, we can only
observe the potential outcome in the presence of the treatment,
\(y^1_{264}\).

The RCM allows for heterogeneity of treatment effects. The treatment has
a large effect on some units and a much smaller effect on other units.
We can even have some units that benefit from the treatment and some
units that are harmed by the treatment. The individual level effect of
the treatment is itself a random variable (and not a fixed parameter).
It has a distribution, \(F_{\Delta^Y}\).

Heterogeneity of treatment effects seems very natural: the treatment
might interact with individuals' different backgrounds. The effect of a
drug might depend on the genetic background of an individual. An
education program might only work for children that already have
sufficient non-cognitive skills, and thus might depend in turn on family
background. An environmental regulation or a behavioral intervention
might only trigger reactions by already environmentally aware
individuals. A CCT might have a larger effect when indiviuals are
credit-constrained or face shocks.

\BeginKnitrBlock{example}
\protect\hypertarget{exm:unnamed-chunk-8}{}{\label{exm:unnamed-chunk-8} }In
our numerical example, the distribution of \(\Delta^y_i=\alpha_i\) is a
normal:
\(\alpha_i\sim\mathcal{N}(\bar{\alpha}+\theta\bar{\mu},\theta^2\sigma^2_{\mu}+\sigma^2_{\eta})\).
We would like to visualize treatment effect heterogeneity. For that, we
can build a histogram of the individual level causal effect.
\EndKnitrBlock{example} On top of the histogram, we can also draw the
theoretical distribution of the treatment effect: a normal with mean
0.18 and variance 0.05.

\begin{Shaded}
\begin{Highlighting}[]
\KeywordTok{hist}\NormalTok{(alpha,}\DataTypeTok{main=}\StringTok{""}\NormalTok{,}\DataTypeTok{prob=}\OtherTok{TRUE}\NormalTok{)}
\KeywordTok{curve}\NormalTok{(}\KeywordTok{dnorm}\NormalTok{(x, }\DataTypeTok{mean=}\NormalTok{(param[}\StringTok{"baralpha"}\NormalTok{]}\OperatorTok{+}\NormalTok{param[}\StringTok{"theta"}\NormalTok{]}\OperatorTok{*}\NormalTok{param[}\StringTok{"barmu"}\NormalTok{]), }\DataTypeTok{sd=}\KeywordTok{sqrt}\NormalTok{(param[}\StringTok{"theta"}\NormalTok{]}\OperatorTok{^}\DecValTok{2}\OperatorTok{*}\NormalTok{param[}\StringTok{"sigma2mu"}\NormalTok{]}\OperatorTok{+}\NormalTok{param[}\StringTok{"sigma2eta"}\NormalTok{])), }\DataTypeTok{add=}\OtherTok{TRUE}\NormalTok{,}\DataTypeTok{col=}\StringTok{'red'}\NormalTok{)}
\end{Highlighting}
\end{Shaded}

\begin{figure}[htbp]

{\centering \includegraphics[width=0.6\linewidth]{STCI_files/figure-latex/histalpha-1} 

}

\caption{Histogram of $\Delta^y$}\label{fig:histalpha}
\end{figure}

The first thing that we can see on Figure \ref{fig:histalpha} is that
the theoretical and the empirical distributions nicely align with each
other. We also see that the majority of the observations lies to the
right of zero: most people experience a positive effect of the
treatment. But there are some individuals that do not benefit from the
treatment: the effect of the treatment on them is negative.

\subsection{Average treatment effect on the
treated}\label{average-treatment-effect-on-the-treated}

We do not generally estimate individual-level treatment effects. We
generally look for summary statistics of the effect of the treatment. By
far the most widely reported causal parameter is the Treatment on the
Treated parameter (TT). It can be defined in the sample at hand or in
the population.

\BeginKnitrBlock{definition}[Average and expected treatment effects on the treated]
\protect\hypertarget{def:TT}{}{\label{def:TT} \iffalse (Average and expected
treatment effects on the treated) \fi{} }The Treatment on the Treated
parameters for outcome \(Y\) are:
\EndKnitrBlock{definition}

\begin{itemize}
\tightlist
\item
  The average Treatment effect on the Treated in the sample:
\end{itemize}

\begin{align*}
  \Delta^Y_{TT_s} & = \frac{1}{\sum_{i=1}^ND_i}\sum_{i=1}^N(Y_i^1-Y_i^0)D_i,
  \end{align*}

\begin{itemize}
\tightlist
\item
  The expected Treatment effect on the Treated in the population:
\end{itemize}

\begin{align*}
  \Delta^Y_{TT} & = \esp{Y_i^1-Y_i^0|D_i=1}.
  \end{align*}

The TT parameters measure the average effect of the treatment on those
who actually take it, either in the sample at hand or in the popluation.
It is generally considered to be the most policy-relevant parameter
since it measures the effect of the treatment as it has actually been
allocated. For example, the expected causal effect on the overall
population is only relevant if policymakers are considering implementing
the treatment even on those who have not been selected to receive it.
For a drug or an anti-poverty program, it would mean giving the
treatment to healthy or rich people, which would make little sense.

TT does not say anything about how the effect of the treatment is
distributed in the population or in the sample. TT does not account for
the heterogneity of treatment effects. In Lecture 7, we will look at
other parameters of interest that look more closely into how the effect
of the treatment is distributed.

\BeginKnitrBlock{example}
\protect\hypertarget{exm:unnamed-chunk-9}{}{\label{exm:unnamed-chunk-9} }The
value of TT in our sample is:
\EndKnitrBlock{example} \[
\Delta^y_{TT_s}=0.168.
\]

Computing the population value of \(TT\) is slightly more involved: we
have to use the formula for the conditional expectation of a censored
bivariate normal random variable:

\begin{align*}
\Delta^y_{TT} & = \esp{\alpha_i|D_i=1}\\
              & = \bar{\alpha}+\theta\esp{\mu_i|\mu_i+U_i^B\leq\bar{y}}\\
              & = \bar{\alpha}+\theta\left(\bar{\mu} - \frac{\sigma^2_{\mu}}{\sqrt{\sigma^2_{\mu}+\sigma^2_{U}}}\frac{\phi\left(\frac{\bar{y}-\bar{\mu}}{\sqrt{\sigma^2_{\mu}+\sigma^2_{U}}}\right)}{\Phi\left(\frac{\bar{y}-\bar{\mu}}{\sqrt{\sigma^2_{\mu}+\sigma^2_{U}}}\right)}\right)\\
              & = \bar{\alpha}+\theta\bar{\mu}-\theta\left(\frac{\sigma^2_{\mu}}{\sqrt{\sigma^2_{\mu}+\sigma^2_{U}}}\frac{\phi\left(\frac{\bar{y}-\bar{\mu}}{\sqrt{\sigma^2_{\mu}+\sigma^2_{U}}}\right)}{\Phi\left(\frac{\bar{y}-\bar{\mu}}{\sqrt{\sigma^2_{\mu}+\sigma^2_{U}}}\right)}\right),
\end{align*}

where \(\phi\) and \(\Phi\) are respectively the density and the
cumulative distribution functions of the standard normal. The second
equality follows from the definition of \(\alpha_i\) and \(D_i\) and
from the fact that \(\eta_i\) is independent from \(\mu_i\) and
\(U_i^B\). The third equality comes from the formula for the expectation
of a censored bivariate normal random variable. In order to compute the
population value of TT easily for different sets of parameter values,
let's write a function in R:

\begin{Shaded}
\begin{Highlighting}[]
\NormalTok{delta.y.tt <-}\StringTok{ }\ControlFlowTok{function}\NormalTok{(param)\{}\KeywordTok{return}\NormalTok{(param[}\StringTok{"baralpha"}\NormalTok{]}\OperatorTok{+}\NormalTok{param[}\StringTok{"theta"}\NormalTok{]}\OperatorTok{*}\NormalTok{param[}\StringTok{"barmu"}\NormalTok{]}
                                     \OperatorTok{-}\NormalTok{param[}\StringTok{"theta"}\NormalTok{]}\OperatorTok{*}\NormalTok{((param[}\StringTok{"sigma2mu"}\NormalTok{]}\OperatorTok{*}\KeywordTok{dnorm}\NormalTok{((}\KeywordTok{log}\NormalTok{(param[}\StringTok{"barY"}\NormalTok{])}\OperatorTok{-}\NormalTok{param[}\StringTok{"barmu"}\NormalTok{])}\OperatorTok{/}\NormalTok{(}\KeywordTok{sqrt}\NormalTok{(param[}\StringTok{"sigma2mu"}\NormalTok{]}\OperatorTok{+}\NormalTok{param[}\StringTok{"sigma2U"}\NormalTok{]))))}
                                                      \OperatorTok{/}\NormalTok{(}\KeywordTok{sqrt}\NormalTok{(param[}\StringTok{"sigma2mu"}\NormalTok{]}\OperatorTok{+}\NormalTok{param[}\StringTok{"sigma2U"}\NormalTok{])}
                                                        \OperatorTok{*}\KeywordTok{pnorm}\NormalTok{((}\KeywordTok{log}\NormalTok{(param[}\StringTok{"barY"}\NormalTok{])}\OperatorTok{-}\NormalTok{param[}\StringTok{"barmu"}\NormalTok{])}\OperatorTok{/}\NormalTok{(}\KeywordTok{sqrt}\NormalTok{(param[}\StringTok{"sigma2mu"}\NormalTok{]}\OperatorTok{+}\NormalTok{param[}\StringTok{"sigma2U"}\NormalTok{]))))))\}}
\end{Highlighting}
\end{Shaded}

The population value of TT computed using this function is:
\(\Delta^y_{TT}=\) 0.172. We can see that the values of TT in the sample
and in the population differ slightly. This is because of sampling
noise: the units in the sample are not perfectly representative of the
units in the population.

\section{Fundamental problem of causal
inference}\label{fundamental-problem-of-causal-inference}

At least in this lecture, causal inference is about trying to infer TT,
either in the sample or in the population. The FPCI states that it is
impossible to directly observe TT because one part of it remains
fundamentally unobserved.

\BeginKnitrBlock{theorem}[Fundamental problem of causal inference]
\protect\hypertarget{thm:FPCI}{}{\label{thm:FPCI} \iffalse (Fundamental
problem of causal inference) \fi{} }It is impossible to observe TT,
either in the population or in the sample.
\EndKnitrBlock{theorem}

\BeginKnitrBlock{proof}
\iffalse{} {Proof. } \fi{}The proof of the FPCI is rather
straightforward. Let me start with the sample TT:

\begin{align*}
  \Delta^Y_{TT_s} & = \frac{1}{\sum_{i=1}^ND_i}\sum_{i=1}^N(Y_i^1-Y_i^0)D_i \\
                  & = \frac{1}{\sum_{i=1}^ND_i}\sum_{i=1}^NY_i^1D_i- \frac{1}{\sum_{i=1}^ND_i}\sum_{i=1}^NY_i^0D_i \\
                  & = \frac{1}{\sum_{i=1}^ND_i}\sum_{i=1}^NY_iD_i- \frac{1}{\sum_{i=1}^ND_i}\sum_{i=1}^NY_i^0D_i.
\end{align*}

Since \(Y_i^0\) is unobserved whenever \(D_i=1\),
\(\frac{1}{\sum_{i=1}^ND_i}\sum_{i=1}^NY_i^0D_i\) is unobserved, and so
is \(\Delta^Y_{TT_s}\). The same is true for the population TT:

\begin{align*}
  \Delta^Y_{TT} & = \esp{Y_i^1-Y_i^0|D_i=1} \\
                & = \esp{Y_i^1|D_i=1}-\esp{Y_i^0|D_i=1}\\
                & = \esp{Y_i|D_i=1}-\esp{Y_i^0|D_i=1}.
\end{align*}

\(\esp{Y_i^0|D_i=1}\) is unobserved, and so is \(\Delta^Y_{TT}\).
\EndKnitrBlock{proof}

The key insight in order to understand the FPCI is to see that the
outcomes of the treated units had they not been treated are
unobservable, and so is their average or expectation. We say that they
are counterfactual, contrary to what has happened.

\BeginKnitrBlock{definition}[Couterfactual]
\protect\hypertarget{def:counter}{}{\label{def:counter}
\iffalse (Couterfactual) \fi{} }Both
\(\frac{1}{\sum_{i=1}^ND_i}\sum_{i=1}^NY_i^0D_i\) and
\(\esp{Y_i^0|D_i=1}\) are counterfactual quantities that we will never
get to observe.
\EndKnitrBlock{definition}

\BeginKnitrBlock{example}
\protect\hypertarget{exm:unnamed-chunk-11}{}{\label{exm:unnamed-chunk-11}
}The average counterfactual outcome of the treated is the mean of the
red circles in the \(y\) axis on Figure \ref{fig:ploty1y0yB}:
\EndKnitrBlock{example} \[
\frac{1}{\sum_{i=1}^ND_i}\sum_{i=1}^Ny_i^0D_i= 6.91.
\] Remember that we can estimate this quantity only because we have
generated the data ourselves. In real life, this quantity is hopelessly
unobserved.

\(\esp{y_i^0|D_i=1}\) can be computed using the formula for the
expectation of a censored normal random variable:

\begin{align*}
\esp{y_i^0|D_i=1} & = \esp{\mu_i+\delta+U_i^0|D_i=1}\\
                  & = \esp{\mu_i+\delta+\rho U_i^B+\epsilon_i|D_i=1}\\
                  & = \delta + \esp{\mu_i+\rho U_i^B|y_i^B\leq\bar{y}}\\
                  & = \delta + \bar{\mu} - \frac{\sigma^2_{\mu}+\rho\sigma^2_U}{\sqrt{\sigma^2_{\mu}+\sigma^2_{U}}}\frac{\phi\left(\frac{\bar{y}-\bar{\mu}}{\sqrt{\sigma^2_{\mu}+\sigma^2_{U}}}\right)}{\Phi\left(\frac{\bar{y}-\bar{\mu}}{\sqrt{\sigma^2_{\mu}+\sigma^2_{U}}}\right)}.
\end{align*}

We can write a function in R to compute this value:

\begin{Shaded}
\begin{Highlighting}[]
\NormalTok{esp.y0.D1 <-}\StringTok{ }\ControlFlowTok{function}\NormalTok{(param)\{}
  \KeywordTok{return}\NormalTok{(param[}\StringTok{"delta"}\NormalTok{]}\OperatorTok{+}\NormalTok{param[}\StringTok{"barmu"}\NormalTok{]}
         \OperatorTok{-}\NormalTok{((param[}\StringTok{"sigma2mu"}\NormalTok{]}\OperatorTok{+}\NormalTok{param[}\StringTok{"rho"}\NormalTok{]}\OperatorTok{*}\NormalTok{param[}\StringTok{"sigma2U"}\NormalTok{])}
           \OperatorTok{*}\KeywordTok{dnorm}\NormalTok{((}\KeywordTok{log}\NormalTok{(param[}\StringTok{"barY"}\NormalTok{])}\OperatorTok{-}\NormalTok{param[}\StringTok{"barmu"}\NormalTok{])}\OperatorTok{/}\NormalTok{(}\KeywordTok{sqrt}\NormalTok{(param[}\StringTok{"sigma2mu"}\NormalTok{]}\OperatorTok{+}\NormalTok{param[}\StringTok{"sigma2U"}\NormalTok{]))))}
         \OperatorTok{/}\NormalTok{(}\KeywordTok{sqrt}\NormalTok{(param[}\StringTok{"sigma2mu"}\NormalTok{]}\OperatorTok{+}\NormalTok{param[}\StringTok{"sigma2U"}\NormalTok{])}\OperatorTok{*}\KeywordTok{pnorm}\NormalTok{((}\KeywordTok{log}\NormalTok{(param[}\StringTok{"barY"}\NormalTok{])}\OperatorTok{-}\NormalTok{param[}\StringTok{"barmu"}\NormalTok{])}
                                                          \OperatorTok{/}\NormalTok{(}\KeywordTok{sqrt}\NormalTok{(param[}\StringTok{"sigma2mu"}\NormalTok{]}\OperatorTok{+}\NormalTok{param[}\StringTok{"sigma2U"}\NormalTok{])))))}
\NormalTok{\}}
\end{Highlighting}
\end{Shaded}

The population value of TT computed using this function is:
\(\esp{y_i^0|D_i=1}=\) 6.9.

\section{Intuitive estimators, confounding factors and selection
bias}\label{intuitive-estimators-confounding-factors-and-selection-bias}

In this section, we are going to examine the properties of two intuitive
comparisons that laypeople, policymakers but also ourselves make in
order to estimate causal effects: the with/wihtout comparison (\(WW\))
and the before/after comparison (\(BA\)). \(WW\) compares the average
outcomes of the treated individuals with those of the untreated
individuals. \(BA\) compares the average outcomes of the treated after
taking the treatment to their average outcomes before they took the
treatment. These comparisons try to proxy for the expected
counterfactual outcome in the treated group by using an observed
quantity. \(WW\) uses the expected outcome of the untreated individuals
as a proxy. \(BA\) uses the expected outcome of the treated before they
take the treatment as a proxy.

Unfortunately, both of these proxies are generally poor and provide
biased estimates of \(TT\). The reason that these proxies are poor is
that the treatment is not the only factor that differentiates the
treated group from the groups used to form the proxy. The intuitive
comparisons are biased because factors, other than the treatment, are
correlated to its allocation. The factors that bias the intuitive
comparisons are generally called confouding factors or confounders.

The treatment effect measures the effect of a ceteris paribus change in
treatment status, while the intuitive comparisons capture both the
effect of this change and that of other correlated changes that
spuriously contaminate the comparison. Intuitive comparisons measure
correlations while treatment effects measure causality. The old motto
``correlation is not causation'' applies vehemently here.

\BeginKnitrBlock{remark}
\iffalse{} {Remark. } \fi{}A funny anecdote about this expression
``correlation is not causation''. This expression is due to Karl
Pearson, the father of modern statistics. He coined the phrase in his
famous book ``The Grammar of Science.'' Pearson is famous for inventing
the correlation coefficient. He actually thought that correlation was a
much superior, much more rigorous term, than causation. In his book, he
actually used the sentence to argue in favor of abandoning causation
altogether and focusing on the much better-defined and measurable
concept of correlation. Interesting turn of events that his sentence is
now used to mean that correlation is weaker than causation, totally
reverting the original intended meaning.
\EndKnitrBlock{remark}

In this section, we are going to define both comparisons, study their
biases and state the conditions under which they identify \(TT\). This
will prove to be a very useful introduction to the notion of
identification. It is also very important to be able to understand the
sources of bias of comparisons that we use every day and that come very
naturally to policy makers and lay people.

\BeginKnitrBlock{remark}
\iffalse{} {Remark. } \fi{}In this section, we state the definitions and
formulae in the population. This is for two reasons. First, it is
simpler, and lighter in terms of notation. Second, it emphasizes that
the problems with intuitive comparisons are independent of sampling
noise. Most of the results stated here for the population extend to the
sample, replacing the expectation operator by the average operator. I
will nevertheless give examples in the sample, since it is so much
simpler to compute. I will denote sample equivalents of population
estimators with a hat.
\EndKnitrBlock{remark}

\subsection{With/Without comparison, selection bias and cross-sectional
confounders}\label{withwithout-comparison-selection-bias-and-cross-sectional-confounders}

The with/without comparison (\(WW\)) is very intuitive: just compare the
outcomes of the treated and untreated individuals in order to estimate
the causal effect. This approach is nevertheless generally biased. We
call the bias of \(WW\) selection bias (\(SB\)). Selection bias is due
to unobserved confounders that are distributed differently in the
treatment and control group and that generate differences in outcomes
even in the absence of the treatment. In this section, I define the
\(WW\) estimator, derives its bias, introduces the confounders and
states conditions under which it is unbiased.

\subsubsection{With/Without comparison}\label{withwithout-comparison}

The with/without comparison (\(WW\)) is very intuitive: just compare the
outcomes of the treated and untreated individuals in order to estimate
the causal effect.

\BeginKnitrBlock{definition}[With/without comparison]
\protect\hypertarget{def:unnamed-chunk-14}{}{\label{def:unnamed-chunk-14}
\iffalse (With/without comparison) \fi{} }The with/without comparison is
the difference between the expected outcomes of the treated and the
expected outcomes of the untreated:

\begin{align*}
\Delta^Y_{WW} & =  \esp{Y_i|D_i=1}-\esp{Y_i|D_i=0}.
\end{align*}
\EndKnitrBlock{definition}

\BeginKnitrBlock{example}
\protect\hypertarget{exm:unnamed-chunk-15}{}{\label{exm:unnamed-chunk-15}
}In the population, \(WW\) can be computed using the traditional formula
for the expectation of a truncated normal distribution:

\begin{align*}
\Delta^y_{WW} & = \esp{y_i|D_i=1}-\esp{y_i|D_i=0} \\
              & = \esp{y_i^1|D_i=1}-\esp{y^0_i|D_i=0} \\
              & = \esp{\alpha_i|D_i=1}+\esp{\mu_i+\rho U_i^B|\mu_i+U_i^B\leq\bar{y}}-\esp{\mu_i+\rho U_i^B|\mu_i+U_i^B>\bar{y}} \\
              & = \bar{\alpha}+\theta\left(\bar{\mu}-\frac{\sigma^2_{\mu}}{\sqrt{\sigma^2_{\mu}+\sigma^2_{U}}}\frac{\phi\left(\frac{\bar{y}-\bar{\mu}}{\sqrt{\sigma^2_{\mu}+\sigma^2_{U}}}\right)}{\Phi\left(\frac{\bar{y}-\bar{\mu}}{\sqrt{\sigma^2_{\mu}+\sigma^2_{U}}}\right)}\right)  
              -\frac{\sigma^2_{\mu}+\rho\sigma^2_{U}}{\sqrt{\sigma^2_{\mu}+\sigma^2_{U}}}\left(\frac{\phi\left(\frac{\bar{y}-\bar{\mu}}{\sqrt{\sigma^2_{\mu}+\sigma^2_{U}}}\right)}{\Phi\left(\frac{\bar{y}-\bar{\mu}}{\sqrt{\sigma^2_{\mu}+\sigma^2_{U}}}\right)}+\frac{\phi\left(\frac{\bar{y}-\bar{\mu}}{\sqrt{\sigma^2_{\mu}+\sigma^2_{U}}}\right)}{1-\Phi\left(\frac{\bar{y}-\bar{\mu}}{\sqrt{\sigma^2_{\mu}+\sigma^2_{U}}}\right)}\right).
\end{align*}
\EndKnitrBlock{example} In order to compute this parameter, we are going
to set up a R function. For reasons that will become clearer later, we
will define two separate functions to compute the first and second part
of the formula. In the first part, you should have recognised \(TT\),
that we have already computed in Lecture 1. We are going to call the
second part \(SB\), for reasons that will become explicit in a bit.

\begin{Shaded}
\begin{Highlighting}[]
\NormalTok{delta.y.tt <-}\StringTok{ }\ControlFlowTok{function}\NormalTok{(param)\{}
  \KeywordTok{return}\NormalTok{(param[}\StringTok{"baralpha"}\NormalTok{]}\OperatorTok{+}\NormalTok{param[}\StringTok{"theta"}\NormalTok{]}\OperatorTok{*}\NormalTok{param[}\StringTok{"barmu"}\NormalTok{]}\OperatorTok{-}\NormalTok{param[}\StringTok{"theta"}\NormalTok{]}
         \OperatorTok{*}\NormalTok{((param[}\StringTok{"sigma2mu"}\NormalTok{]}\OperatorTok{*}\KeywordTok{dnorm}\NormalTok{((}\KeywordTok{log}\NormalTok{(param[}\StringTok{"barY"}\NormalTok{])}\OperatorTok{-}\NormalTok{param[}\StringTok{"barmu"}\NormalTok{])}
                                    \OperatorTok{/}\NormalTok{(}\KeywordTok{sqrt}\NormalTok{(param[}\StringTok{"sigma2mu"}\NormalTok{]}\OperatorTok{+}\NormalTok{param[}\StringTok{"sigma2U"}\NormalTok{]))))}
           \OperatorTok{/}\NormalTok{(}\KeywordTok{sqrt}\NormalTok{(param[}\StringTok{"sigma2mu"}\NormalTok{]}\OperatorTok{+}\NormalTok{param[}\StringTok{"sigma2U"}\NormalTok{])}\OperatorTok{*}\KeywordTok{pnorm}\NormalTok{((}\KeywordTok{log}\NormalTok{(param[}\StringTok{"barY"}\NormalTok{])}\OperatorTok{-}\NormalTok{param[}\StringTok{"barmu"}\NormalTok{])}
                                                            \OperatorTok{/}\NormalTok{(}\KeywordTok{sqrt}\NormalTok{(param[}\StringTok{"sigma2mu"}\NormalTok{]}\OperatorTok{+}\NormalTok{param[}\StringTok{"sigma2U"}\NormalTok{]))))))}
\NormalTok{\}}
\NormalTok{delta.y.sb <-}\StringTok{ }\ControlFlowTok{function}\NormalTok{(param)\{}
  \KeywordTok{return}\NormalTok{(}\OperatorTok{-}\NormalTok{(param[}\StringTok{"sigma2mu"}\NormalTok{]}\OperatorTok{+}\NormalTok{param[}\StringTok{"rho"}\NormalTok{]}\OperatorTok{*}\NormalTok{param[}\StringTok{"sigma2U"}\NormalTok{])}\OperatorTok{/}\KeywordTok{sqrt}\NormalTok{(param[}\StringTok{"sigma2mu"}\NormalTok{]}\OperatorTok{+}\NormalTok{param[}\StringTok{"sigma2U"}\NormalTok{])}
         \OperatorTok{*}\KeywordTok{dnorm}\NormalTok{((}\KeywordTok{log}\NormalTok{(param[}\StringTok{"barY"}\NormalTok{])}\OperatorTok{-}\NormalTok{param[}\StringTok{"barmu"}\NormalTok{])}\OperatorTok{/}\NormalTok{(}\KeywordTok{sqrt}\NormalTok{(param[}\StringTok{"sigma2mu"}\NormalTok{]}\OperatorTok{+}\NormalTok{param[}\StringTok{"sigma2U"}\NormalTok{])))}
         \OperatorTok{*}\NormalTok{(}\DecValTok{1}\OperatorTok{/}\KeywordTok{pnorm}\NormalTok{((}\KeywordTok{log}\NormalTok{(param[}\StringTok{"barY"}\NormalTok{])}\OperatorTok{-}\NormalTok{param[}\StringTok{"barmu"}\NormalTok{])}\OperatorTok{/}\NormalTok{(}\KeywordTok{sqrt}\NormalTok{(param[}\StringTok{"sigma2mu"}\NormalTok{]}\OperatorTok{+}\NormalTok{param[}\StringTok{"sigma2U"}\NormalTok{])))}
           \OperatorTok{+}\DecValTok{1}\OperatorTok{/}\NormalTok{(}\DecValTok{1}\OperatorTok{-}\KeywordTok{pnorm}\NormalTok{((}\KeywordTok{log}\NormalTok{(param[}\StringTok{"barY"}\NormalTok{])}\OperatorTok{-}\NormalTok{param[}\StringTok{"barmu"}\NormalTok{])}\OperatorTok{/}\NormalTok{(}\KeywordTok{sqrt}\NormalTok{(param[}\StringTok{"sigma2mu"}\NormalTok{]}\OperatorTok{+}\NormalTok{param[}\StringTok{"sigma2U"}\NormalTok{]))))))}
\NormalTok{\}}
\NormalTok{delta.y.ww <-}\StringTok{ }\ControlFlowTok{function}\NormalTok{(param)\{}
  \KeywordTok{return}\NormalTok{(}\KeywordTok{delta.y.tt}\NormalTok{(param)}\OperatorTok{+}\KeywordTok{delta.y.sb}\NormalTok{(param))}
\NormalTok{\}}
\end{Highlighting}
\end{Shaded}

As a conclusion of all these derivations, \(WW\) in the population is
equal to -1.298. Remember that the value of \(TT\) in the population is
0.172.

In order to compute the \(WW\) estimator in a sample, I'm going to
generate a brand new sample and I'm going to choose a seed for the
pseudo-random number generator so that we obtain the same result each
time we run the code. I use \texttt{set.seed(1234)} in the code chunk
below.

\begin{Shaded}
\begin{Highlighting}[]
\NormalTok{param <-}\StringTok{ }\KeywordTok{c}\NormalTok{(}\DecValTok{8}\NormalTok{,.}\DecValTok{5}\NormalTok{,.}\DecValTok{28}\NormalTok{,}\DecValTok{1500}\NormalTok{)}
\KeywordTok{names}\NormalTok{(param) <-}\StringTok{ }\KeywordTok{c}\NormalTok{(}\StringTok{"barmu"}\NormalTok{,}\StringTok{"sigma2mu"}\NormalTok{,}\StringTok{"sigma2U"}\NormalTok{,}\StringTok{"barY"}\NormalTok{)}
\KeywordTok{set.seed}\NormalTok{(}\DecValTok{1234}\NormalTok{)}
\NormalTok{N <-}\DecValTok{1000}
\NormalTok{mu <-}\StringTok{ }\KeywordTok{rnorm}\NormalTok{(N,param[}\StringTok{"barmu"}\NormalTok{],}\KeywordTok{sqrt}\NormalTok{(param[}\StringTok{"sigma2mu"}\NormalTok{]))}
\NormalTok{UB <-}\StringTok{ }\KeywordTok{rnorm}\NormalTok{(N,}\DecValTok{0}\NormalTok{,}\KeywordTok{sqrt}\NormalTok{(param[}\StringTok{"sigma2U"}\NormalTok{]))}
\NormalTok{yB <-}\StringTok{ }\NormalTok{mu }\OperatorTok{+}\StringTok{ }\NormalTok{UB}
\NormalTok{YB <-}\StringTok{ }\KeywordTok{exp}\NormalTok{(yB)}
\NormalTok{Ds <-}\StringTok{ }\KeywordTok{rep}\NormalTok{(}\DecValTok{0}\NormalTok{,N)}
\NormalTok{Ds[YB}\OperatorTok{<=}\NormalTok{param[}\StringTok{"barY"}\NormalTok{]] <-}\StringTok{ }\DecValTok{1}
\NormalTok{l <-}\StringTok{ }\KeywordTok{length}\NormalTok{(param)}
\NormalTok{param <-}\StringTok{ }\KeywordTok{c}\NormalTok{(param,}\FloatTok{0.9}\NormalTok{,}\FloatTok{0.01}\NormalTok{,}\FloatTok{0.05}\NormalTok{,}\FloatTok{0.05}\NormalTok{,}\FloatTok{0.05}\NormalTok{,}\FloatTok{0.1}\NormalTok{)}
\KeywordTok{names}\NormalTok{(param)[(l}\OperatorTok{+}\DecValTok{1}\NormalTok{)}\OperatorTok{:}\KeywordTok{length}\NormalTok{(param)] <-}\StringTok{ }\KeywordTok{c}\NormalTok{(}\StringTok{"rho"}\NormalTok{,}\StringTok{"theta"}\NormalTok{,}\StringTok{"sigma2epsilon"}\NormalTok{,}\StringTok{"sigma2eta"}\NormalTok{,}\StringTok{"delta"}\NormalTok{,}\StringTok{"baralpha"}\NormalTok{)}
\NormalTok{epsilon <-}\StringTok{ }\KeywordTok{rnorm}\NormalTok{(N,}\DecValTok{0}\NormalTok{,}\KeywordTok{sqrt}\NormalTok{(param[}\StringTok{"sigma2epsilon"}\NormalTok{]))}
\NormalTok{eta<-}\StringTok{ }\KeywordTok{rnorm}\NormalTok{(N,}\DecValTok{0}\NormalTok{,}\KeywordTok{sqrt}\NormalTok{(param[}\StringTok{"sigma2eta"}\NormalTok{]))}
\NormalTok{U0 <-}\StringTok{ }\NormalTok{param[}\StringTok{"rho"}\NormalTok{]}\OperatorTok{*}\NormalTok{UB }\OperatorTok{+}\StringTok{ }\NormalTok{epsilon}
\NormalTok{y0 <-}\StringTok{ }\NormalTok{mu }\OperatorTok{+}\StringTok{  }\NormalTok{U0 }\OperatorTok{+}\StringTok{ }\NormalTok{param[}\StringTok{"delta"}\NormalTok{]}
\NormalTok{alpha <-}\StringTok{ }\NormalTok{param[}\StringTok{"baralpha"}\NormalTok{]}\OperatorTok{+}\StringTok{  }\NormalTok{param[}\StringTok{"theta"}\NormalTok{]}\OperatorTok{*}\NormalTok{mu }\OperatorTok{+}\StringTok{ }\NormalTok{eta}
\NormalTok{y1 <-}\StringTok{ }\NormalTok{y0}\OperatorTok{+}\NormalTok{alpha}
\NormalTok{Y0 <-}\StringTok{ }\KeywordTok{exp}\NormalTok{(y0)}
\NormalTok{Y1 <-}\StringTok{ }\KeywordTok{exp}\NormalTok{(y1)}
\NormalTok{y <-}\StringTok{ }\NormalTok{y1}\OperatorTok{*}\NormalTok{Ds}\OperatorTok{+}\NormalTok{y0}\OperatorTok{*}\NormalTok{(}\DecValTok{1}\OperatorTok{-}\NormalTok{Ds)}
\NormalTok{Y <-}\StringTok{ }\NormalTok{Y1}\OperatorTok{*}\NormalTok{Ds}\OperatorTok{+}\NormalTok{Y0}\OperatorTok{*}\NormalTok{(}\DecValTok{1}\OperatorTok{-}\NormalTok{Ds)}
\end{Highlighting}
\end{Shaded}

In this sample, the average outcome of the treated in the presence of
the treatment is \[
\frac{1}{\sum_{i=1}^ND_i}\sum_{i=1}^ND_iy_i= 7.074.
\] It is materialized by a circle on Figure \ref{fig:ploty0y1yBD}. The
average outcome of the untreated is \[
\frac{1}{\sum_{i=1}^N(1-D_i)}\sum_{i=1}^N(1-D_i)y_i= 8.383.
\] It is materialized by a plus sign on Figure \ref{fig:ploty0y1yBD}.

\begin{figure}

{\centering \includegraphics[width=0.6\linewidth]{STCI_files/figure-latex/ploty0y1yBD-1} 

}

\caption{Evolution of average outcomes in the treated and control group before (Time =1) and after (Time=2) the treatment}\label{fig:ploty0y1yBD}
\end{figure}

The estimate of the \(WW\) comparison in the sample is thus:

\begin{align*}
\hat{\Delta^Y_{WW}} & = \frac{1}{\sum_{i=1}^N D_i}\sum_{i=1}^N Y_iD_i-\frac{1}{\sum_{i=1}^N (1-D_i)}\sum_{i=1}^N Y_i(1-D_i).
\end{align*}

We have \(\hat{\Delta^y_{WW}}=\) -1.308. Remember that the value of
\(TT\) in the sample is \(\Delta^y_{TT_s}=\) 0.168.

Overall, \(WW\) severely underestimates the effect of the treatment in
our example. \(WW\) suggests that the treatment has a negative effect on
outcomes whereas we know by construction that it has a positive one.

\subsubsection{Selection bias}\label{selection-bias}

When we form the with/without comparison, we do not recover the \(TT\)
parameter. Instead, we recover \(TT\) plus a bias term, called
\textbf{selection bias}:

\begin{align*}
\Delta^Y_{WW} & =\Delta^Y_{TT}+\Delta^Y_{SB}.
\end{align*}

\BeginKnitrBlock{definition}[Selection bias]
\protect\hypertarget{def:SB}{}{\label{def:SB} \iffalse (Selection bias)
\fi{} }Selection bias is the difference between the with/without
comparison and the treatment on the treated parameter:

\begin{align*}
\Delta^Y_{SB} & = \Delta^Y_{WW}-\Delta^Y_{TT}.
\end{align*}
\EndKnitrBlock{definition}

\(WW\) tries to approximate the counterfactual expected outcome in the
treated group by using \(\esp{Y_i^0|D_i=0}\), the expected outcome in
the untreated group . Selection bias appears because this proxy is
generally poor. It is very easy to see that selection bias is indeed
directly due to this bad proxy problem:

\BeginKnitrBlock{theorem}[Selection bias and counterfactual]
\protect\hypertarget{thm:SBth}{}{\label{thm:SBth} \iffalse (Selection bias
and counterfactual) \fi{} }Selection bias is the difference between the
counterfactual expected potential outcome in the absence of the
treatment among the treated and the expected potential outcome in the
absence of the treatment among the untreated.

\begin{align*}
\Delta^Y_{SB} & = \esp{Y_i^0|D_i=1}-\esp{Y_i^0|D_i=0}.
\end{align*}
\EndKnitrBlock{theorem}

\BeginKnitrBlock{proof}
\iffalse{} {Proof. } \fi{}

\begin{align*}
\Delta^Y_{SB} & = \Delta^Y_{WW}-\Delta^Y_{TT} \\
              & = \esp{Y_i|D_i=1}-\esp{Y_i|D_i=0}-\esp{Y_i^1-Y_i^0|D_i=1}\\
              & = \esp{Y_i^0|D_i=1}-\esp{Y_i^0|D_i=0}.
\end{align*}

The first and second equalities stem only from the definition of both
parameters. The third equality stems from using the switching equation:
\(Y_i=Y_i^1D_i+Y_i^0(1-D_i)\), so that
\(\esp{Y_i|D_i=1}=\esp{Y^1_i|D_i=1}\) and
\(\esp{Y_i|D_i=0}=\esp{Y_i^0|D_i=0}\).
\EndKnitrBlock{proof}

\BeginKnitrBlock{example}
\protect\hypertarget{exm:unnamed-chunk-17}{}{\label{exm:unnamed-chunk-17}
}In the population, \(SB\) is equal to
\EndKnitrBlock{example}

\begin{align*}
  \Delta^y_{SB} & = \Delta^y_{WW}-\Delta^y_{TT} \\
                & = -1.298 - 0.172 \\
                & = -1.471
\end{align*}

We could have computed \(SB\) directly using the formula from Theorem
\ref{thm:SBth}:

\begin{align*}
\Delta^y_{SB} & = \esp{y_i^0|D_i=1}-\esp{y_i^0|D_i=0}\\
              & = -\frac{\sigma^2_{\mu}+\rho\sigma^2_{U}}{\sqrt{\sigma^2_{\mu}+\sigma^2_{U}}}\left(\frac{\phi\left(\frac{\bar{y}-\bar{\mu}}{\sqrt{\sigma^2_{\mu}+\sigma^2_{U}}}\right)}{\Phi\left(\frac{\bar{y}-\bar{\mu}}{\sqrt{\sigma^2_{\mu}+\sigma^2_{U}}}\right)}+\frac{\phi\left(\frac{\bar{y}-\bar{\mu}}{\sqrt{\sigma^2_{\mu}+\sigma^2_{U}}}\right)}{1-\Phi\left(\frac{\bar{y}-\bar{\mu}}{\sqrt{\sigma^2_{\mu}+\sigma^2_{U}}}\right)}\right).
\end{align*}

When using the R function for \(SB\) that we have defined earlier, we
indeed find: \(\Delta^y_{SB}=\) -1.471.

In the sample, \(\hat{\Delta^y_{SB}}=\)-1.308-0.168 \(=\) -1.476.
Selection bias emerges because we are using a bad proxy for the
counterfactual. The average outcome for the untreated is equal to
\(\frac{1}{\sum_{i=1}^N(1-D_i)}\sum_{i=1}^N(1-D_i)y_i=\) 8.383 while the
counterfactual average outcome for the treated is
\(\frac{1}{\sum_{i=1}^ND_i}\sum_{i=1}^ND_iy^0_i=\) 6.906. Their
difference is as expected equal to \(SB\): \(\hat{\Delta^y_{SB}}=\)
6.906 \(-\) 8.383 \(=\) -1.476. The counterfactual average outcome of
the treated is much smaller than the average outcome of the untreated.
On Figure \ref{fig:ploty0y1yBD}, this is materialized by the fact that
the plus sign is located much above the triangle.

\BeginKnitrBlock{remark}
\iffalse{} {Remark. } \fi{}The concept of selection bias is related to
but different from the concept of sample selection bias. With sample
selection bias, we worry that selection into the sample might bias the
estimated effect of a treatment on outcomes. With selection bias, we
worry that selection into the treatment itself might bias the effect of
the treatment on outcomes. Both biases are due to unbserved covariates,
but they do not play out in the same way.

For example, estimating the effect of education on women's wages raises
both selection bias and sample selection bias issues. Selection bias
stems from the fact that more educated women are more likely to be more
dynamic and thus to have higher earnings even when less educated.
Selection bias would be positive in that case, overestimating the effect
of education on earnings.

Sample selection bias stems from the fact that we can only use a sample
of working women in order to estimate the effect of education on wages,
since we do not observe the wages on non working women. But, selection
into the labor force might generate sample selection bias. More educated
women participate more in the labor market, while less educated women
participate less. As a consequence, less educated women that work are
different from the overall sample of less educated women. They might be
more dynamic and work-focused. As a consequence, their wages are higher
than the average wages of the less educated women. Comparing the wages
of less educated women that work to those of more educated women that
work might understate the effect of education on earnings. Sample
selection bias would generate a negative bias on the education
coefficient.
\EndKnitrBlock{remark}

\subsubsection{Confounding factors}\label{confounding-factors}

Confounding factors are the factors that generate differences between
treated and untreated individuals even in the absence of the treatment.
The confounding factors are thus responsible for selection bias. In
general, the mere fact of being selected for receiving the treatment
means that you have a host of characteristics that would differentiate
you from the unselected individuals, even if you were not to receive the
treatment eventually.

For example, if a drug is given to initially sicker individuals, then,
we expect that they will be sicker that the untreated in the absence of
the treatment. Comparing sick individuals to healthy ones is not a sound
way to estimate the effect of a treatment. Obviously, even if our
treatment performs well, healthier individuals will be healthier after
the treatment has been allocated to the sicker patients. The best we can
expect is that the treated patients have recovered, and that their
health after the treatment is comparable to that of the untreated
patients. In that case, the with/without comparison is going to be null,
whereas the true effect of the treatment is positive. Selection bias is
negative in that case: in the absence of the treatment, the average
health status of the treated individuals would have been smaller than
that of the untreated individuals. The confounding factor is the health
status of individuals when the decision to allocate the drug has been
taken. It is correlated to both the allocation of the treatment
(negatively) and to health in the absence of the treatment (positively).

\BeginKnitrBlock{example}
\protect\hypertarget{exm:unnamed-chunk-19}{}{\label{exm:unnamed-chunk-19}
}In our example, \(\mu_i\) and \(U_i^B\) are the confounding factors.
Because the treatment is only given to individuals with pre-treament
outcomes smaller than a threshold (\(y_i^B\leq\bar{y}\)), participants
tend to have smaller \(\mu_i\) and \(U_i^B\) than non participants, as
we can see on Figure \ref{fig:histmuD}.
\EndKnitrBlock{example}

\begin{figure}[htbp]

{\centering \includegraphics[width=0.6\linewidth]{STCI_files/figure-latex/histmuD-1} 

}

\caption{Distribution of confounders in the treated and control group}\label{fig:histmuD}
\end{figure}

Since confounding factors are persistent, they affect the outcomes of
participants and non participants after the treatment date. \(\mu_i\)
persists entirely over time, and \(U_i^B\) persists at a rate \(\rho\).
As a consequence, even in the absence of the treatment, participants
have lower outcomes than non participants, as we can see on Figure
\ref{fig:histmuD}.

We can derive the contributions of both confouding factors to overall
SB:

\begin{align*}
\esp{Y_i^0|D_i=1} & = \esp{\mu_i+\delta+U_i^0|\mu_i+U_i^B\leq\bar{y}}\\
                  & = \delta + \esp{\mu_i|\mu_i+U_i^B\leq\bar{y}} + \rho\esp{U_i^B|\mu_i+U_i^B\leq\bar{y}}\\
\Delta^y_{SB}     & = \esp{\mu_i|\mu_i+U_i^B\leq\bar{y}}-\esp{\mu_i|\mu_i+U_i^B>\bar{y}} \\
                  & \phantom{=} + \rho\left(\esp{U_i^B|\mu_i+U_i^B\leq\bar{y}}-\esp{U_i^B|\mu_i+U_i^B>\bar{y}}\right)\\
                  & = -\frac{\sigma^2_{\mu}}{\sqrt{\sigma^2_{\mu}+\sigma^2_{U}}}\left(\frac{\phi\left(\frac{\bar{y}-\bar{\mu}}{\sqrt{\sigma^2_{\mu}+\sigma^2_{U}}}\right)}{\Phi\left(\frac{\bar{y}-\bar{\mu}}{\sqrt{\sigma^2_{\mu}+\sigma^2_{U}}}\right)}+\frac{\phi\left(\frac{\bar{y}-\bar{\mu}}{\sqrt{\sigma^2_{\mu}+\sigma^2_{U}}}\right)}{1-\Phi\left(\frac{\bar{y}-\bar{\mu}}{\sqrt{\sigma^2_{\mu}+\sigma^2_{U}}}\right)}\right) \\
                  & \phantom{=} -\frac{\rho\sigma^2_{U}}{\sqrt{\sigma^2_{\mu}+\sigma^2_{U}}}\left(\frac{\phi\left(\frac{\bar{y}-\bar{\mu}}{\sqrt{\sigma^2_{\mu}+\sigma^2_{U}}}\right)}{\Phi\left(\frac{\bar{y}-\bar{\mu}}{\sqrt{\sigma^2_{\mu}+\sigma^2_{U}}}\right)}+\frac{\phi\left(\frac{\bar{y}-\bar{\mu}}{\sqrt{\sigma^2_{\mu}+\sigma^2_{U}}}\right)}{1-\Phi\left(\frac{\bar{y}-\bar{\mu}}{\sqrt{\sigma^2_{\mu}+\sigma^2_{U}}}\right)}\right)
\end{align*}

In order to evaluate these quantities, let's build two R functions:

\begin{Shaded}
\begin{Highlighting}[]
\NormalTok{delta.y.sb.mu <-}\StringTok{ }\ControlFlowTok{function}\NormalTok{(param)\{}
  \KeywordTok{return}\NormalTok{(}\OperatorTok{-}\NormalTok{(param[}\StringTok{"sigma2mu"}\NormalTok{])}\OperatorTok{/}\KeywordTok{sqrt}\NormalTok{(param[}\StringTok{"sigma2mu"}\NormalTok{]}\OperatorTok{+}\NormalTok{param[}\StringTok{"sigma2U"}\NormalTok{])}
         \OperatorTok{*}\KeywordTok{dnorm}\NormalTok{((}\KeywordTok{log}\NormalTok{(param[}\StringTok{"barY"}\NormalTok{])}\OperatorTok{-}\NormalTok{param[}\StringTok{"barmu"}\NormalTok{])}\OperatorTok{/}\NormalTok{(}\KeywordTok{sqrt}\NormalTok{(param[}\StringTok{"sigma2mu"}\NormalTok{]}\OperatorTok{+}\NormalTok{param[}\StringTok{"sigma2U"}\NormalTok{])))}
         \OperatorTok{*}\NormalTok{(}\DecValTok{1}\OperatorTok{/}\KeywordTok{pnorm}\NormalTok{((}\KeywordTok{log}\NormalTok{(param[}\StringTok{"barY"}\NormalTok{])}\OperatorTok{-}\NormalTok{param[}\StringTok{"barmu"}\NormalTok{])}\OperatorTok{/}\NormalTok{(}\KeywordTok{sqrt}\NormalTok{(param[}\StringTok{"sigma2mu"}\NormalTok{]}\OperatorTok{+}\NormalTok{param[}\StringTok{"sigma2U"}\NormalTok{])))}
           \OperatorTok{+}\DecValTok{1}\OperatorTok{/}\NormalTok{(}\DecValTok{1}\OperatorTok{-}\KeywordTok{pnorm}\NormalTok{((}\KeywordTok{log}\NormalTok{(param[}\StringTok{"barY"}\NormalTok{])}\OperatorTok{-}\NormalTok{param[}\StringTok{"barmu"}\NormalTok{])}\OperatorTok{/}\NormalTok{(}\KeywordTok{sqrt}\NormalTok{(param[}\StringTok{"sigma2mu"}\NormalTok{]}\OperatorTok{+}\NormalTok{param[}\StringTok{"sigma2U"}\NormalTok{]))))))}
\NormalTok{\}}
\NormalTok{delta.y.sb.U <-}\StringTok{ }\ControlFlowTok{function}\NormalTok{(param)\{}
  \KeywordTok{return}\NormalTok{(}\OperatorTok{-}\NormalTok{(param[}\StringTok{"rho"}\NormalTok{]}\OperatorTok{*}\NormalTok{param[}\StringTok{"sigma2U"}\NormalTok{])}\OperatorTok{/}\KeywordTok{sqrt}\NormalTok{(param[}\StringTok{"sigma2mu"}\NormalTok{]}\OperatorTok{+}\NormalTok{param[}\StringTok{"sigma2U"}\NormalTok{])}
         \OperatorTok{*}\KeywordTok{dnorm}\NormalTok{((}\KeywordTok{log}\NormalTok{(param[}\StringTok{"barY"}\NormalTok{])}\OperatorTok{-}\NormalTok{param[}\StringTok{"barmu"}\NormalTok{])}\OperatorTok{/}\NormalTok{(}\KeywordTok{sqrt}\NormalTok{(param[}\StringTok{"sigma2mu"}\NormalTok{]}\OperatorTok{+}\NormalTok{param[}\StringTok{"sigma2U"}\NormalTok{])))}
         \OperatorTok{*}\NormalTok{(}\DecValTok{1}\OperatorTok{/}\KeywordTok{pnorm}\NormalTok{((}\KeywordTok{log}\NormalTok{(param[}\StringTok{"barY"}\NormalTok{])}\OperatorTok{-}\NormalTok{param[}\StringTok{"barmu"}\NormalTok{])}\OperatorTok{/}\NormalTok{(}\KeywordTok{sqrt}\NormalTok{(param[}\StringTok{"sigma2mu"}\NormalTok{]}\OperatorTok{+}\NormalTok{param[}\StringTok{"sigma2U"}\NormalTok{])))}
           \OperatorTok{+}\DecValTok{1}\OperatorTok{/}\NormalTok{(}\DecValTok{1}\OperatorTok{-}\KeywordTok{pnorm}\NormalTok{((}\KeywordTok{log}\NormalTok{(param[}\StringTok{"barY"}\NormalTok{])}\OperatorTok{-}\NormalTok{param[}\StringTok{"barmu"}\NormalTok{])}\OperatorTok{/}\NormalTok{(}\KeywordTok{sqrt}\NormalTok{(param[}\StringTok{"sigma2mu"}\NormalTok{]}\OperatorTok{+}\NormalTok{param[}\StringTok{"sigma2U"}\NormalTok{]))))))}
\NormalTok{\}}
\end{Highlighting}
\end{Shaded}

The contribution of \(\mu_i\) to selection bias is -0.978 while that of
\(U_i^0\) is of -0.493.

\subsubsection{\texorpdfstring{When does \(WW\) identify
\(TT\)?}{When does WW identify TT?}}\label{when-does-ww-identify-tt}

Are there conditions under which \(WW\) identify \(TT\)? The answer is
yes: when there is no selection bias, the proxy used by \(WW\) for the
counterfactual quantity is actually valid. Formally, \(WW\) identifies
\(TT\) when the following assumption holds:

\BeginKnitrBlock{definition}[No selection bias]
\protect\hypertarget{def:noselb}{}{\label{def:noselb} \iffalse (No selection
bias) \fi{} }We assume the following:

\begin{align*}
\esp{Y_i^0|D_i=1} & = \esp{Y_i^0|D_i=0}.
\end{align*}
\EndKnitrBlock{definition} Under Assumption \ref{def:noselb}, the
expected counterfactual outcome of the treated is equal to the expected
potential outcome of the untreated in the absence of the treatment. This
yields to the following result:

\BeginKnitrBlock{theorem}
\protect\hypertarget{thm:wwtt}{}{\label{thm:wwtt} }Under Assumption
\ref{def:noselb}, \(WW\) identifies the \(TT\) parameter:

\begin{align*}
\Delta^Y_{WW} & = \Delta^Y_{TT}.
\end{align*}
\EndKnitrBlock{theorem}

\BeginKnitrBlock{proof}
\iffalse{} {Proof. } \fi{}

\begin{align*}
\Delta^Y_{WW} & = \esp{Y_i|D_i=1}-\esp{Y_i|D_i=0}\\
              & = \esp{Y_i^1|D_i=1}-\esp{Y_i^0|D_i=0}\\
              & = \esp{Y_i^1|D_i=1}-\esp{Y_i^0|D_i=1} \\
              & = \Delta^Y_{TT},
\end{align*}

where the second equation uses the switching equation and the third uses
Assumption \ref{def:noselb}.
\EndKnitrBlock{proof}

So, under Assumption \ref{def:noselb}, the \(WW\) comparison actually
identifies the \(TT\) parameter. We say that Assumption \ref{def:noselb}
is an \textbf{identification assumption}: it serves to identify the
parameter of interest using observed data. The intuition for this result
is simply that, under Assumption \ref{def:noselb}, there are no
confounding factors and thus no selection bias. under Assumption
\ref{def:noselb}, the factors that yield individuals to receive or not
the treatment are mean-independent of the potential outcomes in the
absence of the treatment. In this case, the expected outcome in the
untreated group actually is a perfect proxy for the counterfactual
expected outcome of the treated group.

Obviously, Assumption \ref{def:noselb} is extremely unlikely to hold in
real life. For Assumption \ref{def:noselb} to hold, it has to be that
\textbf{all} the determinants of \(D_i\) are actually unrelated to
\(Y_i^0\). One way to enforce Assumption \ref{def:noselb} is to
randomize treatment intake. We will see this in the Lecture on RCTs. It
might also be possible that Assumption \ref{def:noselb} holds in the
data in the absence of an RCT. But this is not very likely, and should
be checked by every mean possible.

One way to test for the validity of Assumption \ref{def:noselb} is to
compare the values of observed covariates in the treated and untreated
group. For Assumption \ref{def:noselb} to be credible, observed
covariates should be distributed in the same way.

Another nice way to test for the validity of Assumption \ref{def:noselb}
with observed data is to implement a \textbf{placebo test}. A placebo
test looks for an effect where there should be none, if we believe the
identification assumptions. For example, under Assumption
\ref{def:noselb} it should be (even though it is not rigorously implied)
that outcomes before the treatment are also mean-independent of the
treatment allocation. And actually, since a future treatment cannot have
an effect today (unless people anticipate the treatment, which we assume
away here), the \(WW\) comparison before the treatment should be null,
therefore giving a zero effect of the placebo treatment ``will receive
the treatment in the future.''

\BeginKnitrBlock{example}
\protect\hypertarget{exm:unnamed-chunk-21}{}{\label{exm:unnamed-chunk-21}
}When the allocation rule defining \(D_i\) is the eligibility rule that
we have used so far, we have already seen that Assumption
\ref{def:noselb} does not hold and the placebo test should not pass
either.
\EndKnitrBlock{example} One way of generating Assumption
\ref{def:noselb} from the eligibility rule that we are using is to mute
the persistence in outcome dynamics. For example, one could set
\(\rho=0\) and \(\sigma^2_{\mu}=0\).

\begin{Shaded}
\begin{Highlighting}[]
\NormalTok{param <-}\StringTok{ }\KeywordTok{c}\NormalTok{(}\DecValTok{8}\NormalTok{,}\DecValTok{0}\NormalTok{,.}\DecValTok{28}\NormalTok{,}\DecValTok{1500}\NormalTok{,}\DecValTok{0}\NormalTok{,}\FloatTok{0.01}\NormalTok{,}\FloatTok{0.05}\NormalTok{,}\FloatTok{0.05}\NormalTok{,}\FloatTok{0.05}\NormalTok{,}\FloatTok{0.1}\NormalTok{)}
\KeywordTok{names}\NormalTok{(param) <-}\StringTok{ }\KeywordTok{c}\NormalTok{(}\StringTok{"barmu"}\NormalTok{,}\StringTok{"sigma2mu"}\NormalTok{,}\StringTok{"sigma2U"}\NormalTok{,}\StringTok{"barY"}\NormalTok{,}\StringTok{"rho"}\NormalTok{,}\StringTok{"theta"}\NormalTok{,}\StringTok{"sigma2epsilon"}\NormalTok{,}\StringTok{"sigma2eta"}\NormalTok{,}\StringTok{"delta"}\NormalTok{,}\StringTok{"baralpha"}\NormalTok{)}
\NormalTok{param}
\end{Highlighting}
\end{Shaded}

\begin{verbatim}
##         barmu      sigma2mu       sigma2U          barY           rho 
##          8.00          0.00          0.28       1500.00          0.00 
##         theta sigma2epsilon     sigma2eta         delta      baralpha 
##          0.01          0.05          0.05          0.05          0.10
\end{verbatim}

In that case, outcomes are not persistent and Assumption
\ref{def:noselb} holds:

\begin{align*}
\esp{y_i^0|D_i=1} & = \esp{\mu_i+\delta+U_i^0|y_i^B\leq\bar{y}}\\
                  & = \esp{\bar{\mu}+\delta+\epsilon_i|\bar{\mu}+U_i^B\leq\bar{y}}\\
                  & = \bar{\mu} + \delta + \esp{\epsilon_i|\bar{\mu}+U_i^B\leq\bar{y}}\\
                  & = \bar{\mu} + \delta + \esp{\epsilon_i|\bar{\mu}+U_i^B>\bar{y}}\\
                  & = \esp{\mu_i+\delta+U_i^0|y_i^B>\bar{y}}\\
                  & = \esp{y_i^0|D_i=0},                  
\end{align*}

where the second equality follows from \(\sigma^2_{\mu}=0\) and
\(\rho=0\) and the fourth from \(\epsilon_i \Ind U_i^B\). Another direct
way to see this is to use the formula for selection bias that we have
derived above. It is easy to see that with \(\rho=0\) and
\(\sigma^2_{\mu}=0\), \(\Delta^y_{SB}=0\). To be sure, we can compute
\(\Delta^y_{SB}\) with the new parameter values: \(\Delta^y_{SB}=\) 0.
As a consequence, \(\Delta^y_{TT}=\) 0.18 \(=\) 0.18 \(=\Delta^y_{WW}\).

\BeginKnitrBlock{remark}
\iffalse{} {Remark. } \fi{}You might have noticed that the value of
\(\Delta^y_{TT}\) is different than before. It is normal, since it
depends on the values of parameters, and especially on
\(\sigma_{\mu}^2\) and \(\rho\).
\EndKnitrBlock{remark}

Let's see how these quantities behave in the sample.

\begin{Shaded}
\begin{Highlighting}[]
\KeywordTok{set.seed}\NormalTok{(}\DecValTok{1234}\NormalTok{)}
\NormalTok{mu <-}\StringTok{ }\KeywordTok{rnorm}\NormalTok{(N,param[}\StringTok{"barmu"}\NormalTok{],}\KeywordTok{sqrt}\NormalTok{(param[}\StringTok{"sigma2mu"}\NormalTok{]))}
\NormalTok{UB <-}\StringTok{ }\KeywordTok{rnorm}\NormalTok{(N,}\DecValTok{0}\NormalTok{,}\KeywordTok{sqrt}\NormalTok{(param[}\StringTok{"sigma2U"}\NormalTok{]))}
\NormalTok{yB <-}\StringTok{ }\NormalTok{mu }\OperatorTok{+}\StringTok{ }\NormalTok{UB }
\NormalTok{YB <-}\StringTok{ }\KeywordTok{exp}\NormalTok{(yB)}
\NormalTok{Ds <-}\StringTok{ }\KeywordTok{rep}\NormalTok{(}\DecValTok{0}\NormalTok{,N)}
\NormalTok{Ds[YB}\OperatorTok{<=}\NormalTok{param[}\StringTok{"barY"}\NormalTok{]] <-}\StringTok{ }\DecValTok{1} 
\NormalTok{epsilon <-}\StringTok{ }\KeywordTok{rnorm}\NormalTok{(N,}\DecValTok{0}\NormalTok{,}\KeywordTok{sqrt}\NormalTok{(param[}\StringTok{"sigma2epsilon"}\NormalTok{]))}
\NormalTok{eta<-}\StringTok{ }\KeywordTok{rnorm}\NormalTok{(N,}\DecValTok{0}\NormalTok{,}\KeywordTok{sqrt}\NormalTok{(param[}\StringTok{"sigma2eta"}\NormalTok{]))}
\NormalTok{U0 <-}\StringTok{ }\NormalTok{param[}\StringTok{"rho"}\NormalTok{]}\OperatorTok{*}\NormalTok{UB }\OperatorTok{+}\StringTok{ }\NormalTok{epsilon}
\NormalTok{y0 <-}\StringTok{ }\NormalTok{mu }\OperatorTok{+}\StringTok{  }\NormalTok{U0 }\OperatorTok{+}\StringTok{ }\NormalTok{param[}\StringTok{"delta"}\NormalTok{]}
\NormalTok{alpha <-}\StringTok{ }\NormalTok{param[}\StringTok{"baralpha"}\NormalTok{]}\OperatorTok{+}\StringTok{  }\NormalTok{param[}\StringTok{"theta"}\NormalTok{]}\OperatorTok{*}\NormalTok{mu }\OperatorTok{+}\StringTok{ }\NormalTok{eta}
\NormalTok{y1 <-}\StringTok{ }\NormalTok{y0}\OperatorTok{+}\NormalTok{alpha}
\NormalTok{Y0 <-}\StringTok{ }\KeywordTok{exp}\NormalTok{(y0)}
\NormalTok{Y1 <-}\StringTok{ }\KeywordTok{exp}\NormalTok{(y1)}
\NormalTok{y <-}\StringTok{ }\NormalTok{y1}\OperatorTok{*}\NormalTok{Ds}\OperatorTok{+}\NormalTok{y0}\OperatorTok{*}\NormalTok{(}\DecValTok{1}\OperatorTok{-}\NormalTok{Ds)}
\NormalTok{Y <-}\StringTok{ }\NormalTok{Y1}\OperatorTok{*}\NormalTok{Ds}\OperatorTok{+}\NormalTok{Y0}\OperatorTok{*}\NormalTok{(}\DecValTok{1}\OperatorTok{-}\NormalTok{Ds)}
\end{Highlighting}
\end{Shaded}

We can see that \(\hat{\esp{Y_i^0|D_i=1}}=\) 8.038 \(\approx\) 8.055
\(=\hat{\esp{Y_i^0|D_i=0}}\). This means that \(WW\) should be close to
\(TT\): \(\hat{\Delta^y_{TT}}=\) 0.198 \(\approx\) 0.182
\(=\hat{\Delta^y_{WW}}\). Note that \(\hat{WW}\) in the sample is not
exactly, but only approximately, equal to \(TT\) in the population and
in the sample. This is an instance of the Fundamental Problem of
Statistical Inference that we will study in the next chapter.

Under these restrictions, the placebo test would unfortunately conclude
against Assumption \ref{def:noselb} even though it is valid:

\begin{align*}
\esp{y_i^B|D_i=1} & = \esp{\mu_i+ U_i^B|y_i^B\leq\bar{y}}\\
                  & = \esp{\bar{\mu}+U_i^B|\bar{\mu}+U_i^B\leq\bar{y}}\\
                  & = \bar{\mu}  + \esp{U_i^B|\bar{\mu}+U_i^B\leq\bar{y}}\\
                  & \neq \bar{\mu}  + \esp{U_i^B|\bar{\mu}+U_i^B>\bar{y}}\\
                  & = \esp{\mu_i+U_i^0|y_i^B>\bar{y}}\\
                  & = \esp{y_i^B|D_i=0}.                  
\end{align*}

In the sample, we indeed have that \(\hat{\esp{Y_i^B|D_i=1}}=\) 7.004
\(\neq\) 8.072 \(=\hat{\esp{Y_i^B|D_i=0}}\). The reason for the failure
of the placebo test to conclude that \(Ww\) is actually correct is that
the \(U_i^B\) shock enters both into the selection equation and the
outcome equation for \(y_i^B\), generating a wage at period \(B\)
between the outcomes of the treated and of the untreated. Since it is
not persistent, this wedge does not generate selection bias. This wedge
would not be detected if we could perform it further back in time,
before the selection period.

Another way to make Assumption \ref{def:noselb} work is to generate a
new allocation rule where all the determinants of treatment intake are
indeed orthogonal to potential outcomes and to outcomes before the
treatment. Let's assume for example that \(D_i=\uns{V_i\leq\bar{y}}\),
with \(V_i\sim\mathcal{N}(\bar{\mu},\sigma^2_{\mu}+\sigma^2_{U})\) and
\(V_i\Ind(Y_i^0,Y_i^1,Y_i^B,\mu_i,\eta_i)\). In that case, Assumption
\ref{def:noselb} holds and the placebo test does work. Indeed, we have:

\begin{align*}
\Delta^y_{TT} & = \esp{Y_i^1-Y_i^0|D_i=1} \\
              & = \esp{\alpha_i|D_i=1} \\
              & = \esp{\bar{\alpha}+\theta\mu_i+\eta_i|V_i\leq\bar{y}}\\
              & = \bar{\alpha}+\theta\bar{\mu} \\
              & = \Delta^y_{ATE} \\
\Delta^y_{WW} & = \esp{Y_i|D_i=1} - \esp{Y_i|D_i=0} \\
              & = \esp{Y^1_i|D_i=1} - \esp{Y^0_i|D_i=0} \\
              & = \esp{Y^1_i|V_i\leq\bar{y}} - \esp{Y^0_i|V_i>\bar{y}} \\
              & = \esp{Y^1_i} - \esp{Y^0_i} \\
              & = \Delta^y_{ATE}
\end{align*}

\(ATE\) is the Average Treatment Effect in the population. It is the
expected effect of the treatment on all the members of the population,
not only on the treated. When the treatment is randomly allocated, both
\(TT\) and \(ATE\) are equal, since the treated are a random subset of
the overall population. I prefer to use \(ATE\) for my definition of the
\(R\) function in order not to erase the definition of the \(TT\)
function:

\begin{Shaded}
\begin{Highlighting}[]
\NormalTok{delta.y.ate <-}\StringTok{ }\ControlFlowTok{function}\NormalTok{(param)\{}
  \KeywordTok{return}\NormalTok{(param[}\StringTok{"baralpha"}\NormalTok{]}\OperatorTok{+}\NormalTok{param[}\StringTok{"theta"}\NormalTok{]}\OperatorTok{*}\NormalTok{param[}\StringTok{"barmu"}\NormalTok{])}
\NormalTok{\}}
\end{Highlighting}
\end{Shaded}

In the population, \(WW\) identifies \(TT\): \(\Delta^y_{TT}=\) 0.18
\(=\Delta^y_{WW}\). Let's see how these quantities behave in the sample:

\begin{Shaded}
\begin{Highlighting}[]
\KeywordTok{set.seed}\NormalTok{(}\DecValTok{1234}\NormalTok{)}
\NormalTok{N <-}\DecValTok{1000}
\NormalTok{mu <-}\StringTok{ }\KeywordTok{rnorm}\NormalTok{(N,param[}\StringTok{"barmu"}\NormalTok{],}\KeywordTok{sqrt}\NormalTok{(param[}\StringTok{"sigma2mu"}\NormalTok{]))}
\NormalTok{UB <-}\StringTok{ }\KeywordTok{rnorm}\NormalTok{(N,}\DecValTok{0}\NormalTok{,}\KeywordTok{sqrt}\NormalTok{(param[}\StringTok{"sigma2U"}\NormalTok{]))}
\NormalTok{yB <-}\StringTok{ }\NormalTok{mu }\OperatorTok{+}\StringTok{ }\NormalTok{UB }
\NormalTok{YB <-}\StringTok{ }\KeywordTok{exp}\NormalTok{(yB)}
\NormalTok{Ds <-}\StringTok{ }\KeywordTok{rep}\NormalTok{(}\DecValTok{0}\NormalTok{,N)}
\NormalTok{V <-}\StringTok{ }\KeywordTok{rnorm}\NormalTok{(N,param[}\StringTok{"barmu"}\NormalTok{],}\KeywordTok{sqrt}\NormalTok{(param[}\StringTok{"sigma2mu"}\NormalTok{]}\OperatorTok{+}\NormalTok{param[}\StringTok{"sigma2U"}\NormalTok{]))}
\NormalTok{Ds[V}\OperatorTok{<=}\KeywordTok{log}\NormalTok{(param[}\StringTok{"barY"}\NormalTok{])] <-}\StringTok{ }\DecValTok{1} 
\NormalTok{epsilon <-}\StringTok{ }\KeywordTok{rnorm}\NormalTok{(N,}\DecValTok{0}\NormalTok{,}\KeywordTok{sqrt}\NormalTok{(param[}\StringTok{"sigma2epsilon"}\NormalTok{]))}
\NormalTok{eta<-}\StringTok{ }\KeywordTok{rnorm}\NormalTok{(N,}\DecValTok{0}\NormalTok{,}\KeywordTok{sqrt}\NormalTok{(param[}\StringTok{"sigma2eta"}\NormalTok{]))}
\NormalTok{U0 <-}\StringTok{ }\NormalTok{param[}\StringTok{"rho"}\NormalTok{]}\OperatorTok{*}\NormalTok{UB }\OperatorTok{+}\StringTok{ }\NormalTok{epsilon}
\NormalTok{y0 <-}\StringTok{ }\NormalTok{mu }\OperatorTok{+}\StringTok{  }\NormalTok{U0 }\OperatorTok{+}\StringTok{ }\NormalTok{param[}\StringTok{"delta"}\NormalTok{]}
\NormalTok{alpha <-}\StringTok{ }\NormalTok{param[}\StringTok{"baralpha"}\NormalTok{]}\OperatorTok{+}\StringTok{  }\NormalTok{param[}\StringTok{"theta"}\NormalTok{]}\OperatorTok{*}\NormalTok{mu }\OperatorTok{+}\StringTok{ }\NormalTok{eta}
\NormalTok{y1 <-}\StringTok{ }\NormalTok{y0}\OperatorTok{+}\NormalTok{alpha}
\NormalTok{Y0 <-}\StringTok{ }\KeywordTok{exp}\NormalTok{(y0)}
\NormalTok{Y1 <-}\StringTok{ }\KeywordTok{exp}\NormalTok{(y1)}
\NormalTok{y <-}\StringTok{ }\NormalTok{y1}\OperatorTok{*}\NormalTok{Ds}\OperatorTok{+}\NormalTok{y0}\OperatorTok{*}\NormalTok{(}\DecValTok{1}\OperatorTok{-}\NormalTok{Ds)}
\NormalTok{Y <-}\StringTok{ }\NormalTok{Y1}\OperatorTok{*}\NormalTok{Ds}\OperatorTok{+}\NormalTok{Y0}\OperatorTok{*}\NormalTok{(}\DecValTok{1}\OperatorTok{-}\NormalTok{Ds)}
\end{Highlighting}
\end{Shaded}

In the sample, the counterfactual is well approximated by the outcomes
of the untreated: \(\hat{\esp{Y_i^0|D_i=1}}=\) 8.085 \(\approx\) 8.054
\(=\hat{\esp{Y_i^0|D_i=0}}\). As a consequence, \(WW\) should be close
to \(TT\): \(\hat{\Delta^y_{TT}}=\) 0.168 \(\approx\) 0.199
\(=\hat{\Delta^y_{WW}}\). The placebo test is also valid in that case:
\(\hat{\esp{Y_i^B|D_i=1}}=\) 7.95 \(\approx\) 7.99
\(=\hat{\esp{Y_i^B|D_i=0}}\).

\subsection{The before/after comparison, temporal confounders and time
trend
bias}\label{the-beforeafter-comparison-temporal-confounders-and-time-trend-bias}

The before/after comparison (\(BA\)) is also very intuitive: it consists
in looking at how the outcomes of the treated have changed over time and
to attribute this change to the effect of the treatment. The problem is
that other changes might have affected outcomes in the absence of the
treatment, thereby biasing \(BA\). The bias of \(BA\) is called
time-trend bias. It is due to confounders that affect the outcomes of
the treated over time. This section defines the \(BA\) estimator,
derives its bias, describes the role of the confounders and states
conditions under which \(BA\) identifies \(TT\).

\BeginKnitrBlock{example}
\protect\hypertarget{exm:unnamed-chunk-23}{}{\label{exm:unnamed-chunk-23}
}Before computing any estimates, we need to reset all our parameter
values and generated sample it their usual values:
\EndKnitrBlock{example}

\begin{Shaded}
\begin{Highlighting}[]
\NormalTok{param <-}\StringTok{ }\KeywordTok{c}\NormalTok{(}\DecValTok{8}\NormalTok{,.}\DecValTok{5}\NormalTok{,.}\DecValTok{28}\NormalTok{,}\DecValTok{1500}\NormalTok{)}
\KeywordTok{names}\NormalTok{(param) <-}\StringTok{ }\KeywordTok{c}\NormalTok{(}\StringTok{"barmu"}\NormalTok{,}\StringTok{"sigma2mu"}\NormalTok{,}\StringTok{"sigma2U"}\NormalTok{,}\StringTok{"barY"}\NormalTok{)}
\KeywordTok{set.seed}\NormalTok{(}\DecValTok{1234}\NormalTok{)}
\NormalTok{N <-}\DecValTok{1000}
\NormalTok{mu <-}\StringTok{ }\KeywordTok{rnorm}\NormalTok{(N,param[}\StringTok{"barmu"}\NormalTok{],}\KeywordTok{sqrt}\NormalTok{(param[}\StringTok{"sigma2mu"}\NormalTok{]))}
\NormalTok{UB <-}\StringTok{ }\KeywordTok{rnorm}\NormalTok{(N,}\DecValTok{0}\NormalTok{,}\KeywordTok{sqrt}\NormalTok{(param[}\StringTok{"sigma2U"}\NormalTok{]))}
\NormalTok{yB <-}\StringTok{ }\NormalTok{mu }\OperatorTok{+}\StringTok{ }\NormalTok{UB }
\NormalTok{YB <-}\StringTok{ }\KeywordTok{exp}\NormalTok{(yB)}
\NormalTok{Ds <-}\StringTok{ }\KeywordTok{rep}\NormalTok{(}\DecValTok{0}\NormalTok{,N)}
\NormalTok{Ds[YB}\OperatorTok{<=}\NormalTok{param[}\StringTok{"barY"}\NormalTok{]] <-}\StringTok{ }\DecValTok{1} 
\NormalTok{l <-}\StringTok{ }\KeywordTok{length}\NormalTok{(param)}
\NormalTok{param <-}\StringTok{ }\KeywordTok{c}\NormalTok{(param,}\FloatTok{0.9}\NormalTok{,}\FloatTok{0.01}\NormalTok{,}\FloatTok{0.05}\NormalTok{,}\FloatTok{0.05}\NormalTok{,}\FloatTok{0.05}\NormalTok{,}\FloatTok{0.1}\NormalTok{)}
\KeywordTok{names}\NormalTok{(param)[(l}\OperatorTok{+}\DecValTok{1}\NormalTok{)}\OperatorTok{:}\KeywordTok{length}\NormalTok{(param)] <-}\StringTok{ }\KeywordTok{c}\NormalTok{(}\StringTok{"rho"}\NormalTok{,}\StringTok{"theta"}\NormalTok{,}\StringTok{"sigma2epsilon"}\NormalTok{,}\StringTok{"sigma2eta"}\NormalTok{,}\StringTok{"delta"}\NormalTok{,}\StringTok{"baralpha"}\NormalTok{)}
\NormalTok{epsilon <-}\StringTok{ }\KeywordTok{rnorm}\NormalTok{(N,}\DecValTok{0}\NormalTok{,}\KeywordTok{sqrt}\NormalTok{(param[}\StringTok{"sigma2epsilon"}\NormalTok{]))}
\NormalTok{eta<-}\StringTok{ }\KeywordTok{rnorm}\NormalTok{(N,}\DecValTok{0}\NormalTok{,}\KeywordTok{sqrt}\NormalTok{(param[}\StringTok{"sigma2eta"}\NormalTok{]))}
\NormalTok{U0 <-}\StringTok{ }\NormalTok{param[}\StringTok{"rho"}\NormalTok{]}\OperatorTok{*}\NormalTok{UB }\OperatorTok{+}\StringTok{ }\NormalTok{epsilon}
\NormalTok{y0 <-}\StringTok{ }\NormalTok{mu }\OperatorTok{+}\StringTok{  }\NormalTok{U0 }\OperatorTok{+}\StringTok{ }\NormalTok{param[}\StringTok{"delta"}\NormalTok{]}
\NormalTok{alpha <-}\StringTok{ }\NormalTok{param[}\StringTok{"baralpha"}\NormalTok{]}\OperatorTok{+}\StringTok{  }\NormalTok{param[}\StringTok{"theta"}\NormalTok{]}\OperatorTok{*}\NormalTok{mu }\OperatorTok{+}\StringTok{ }\NormalTok{eta}
\NormalTok{y1 <-}\StringTok{ }\NormalTok{y0}\OperatorTok{+}\NormalTok{alpha}
\NormalTok{Y0 <-}\StringTok{ }\KeywordTok{exp}\NormalTok{(y0)}
\NormalTok{Y1 <-}\StringTok{ }\KeywordTok{exp}\NormalTok{(y1)}
\NormalTok{y <-}\StringTok{ }\NormalTok{y1}\OperatorTok{*}\NormalTok{Ds}\OperatorTok{+}\NormalTok{y0}\OperatorTok{*}\NormalTok{(}\DecValTok{1}\OperatorTok{-}\NormalTok{Ds)}
\NormalTok{Y <-}\StringTok{ }\NormalTok{Y1}\OperatorTok{*}\NormalTok{Ds}\OperatorTok{+}\NormalTok{Y0}\OperatorTok{*}\NormalTok{(}\DecValTok{1}\OperatorTok{-}\NormalTok{Ds)}
\end{Highlighting}
\end{Shaded}

\subsubsection{The before/after
comparison}\label{the-beforeafter-comparison}

The before/after estimator (\(BA\)) compares the outcomes of the treated
after taking the treatment to the outcomes of the treated before taking
the treatment. It is also sometimes called a ``pre-post comparison.''

\BeginKnitrBlock{definition}[Before/after comparison]
\protect\hypertarget{def:unnamed-chunk-24}{}{\label{def:unnamed-chunk-24}
\iffalse (Before/after comparison) \fi{} }The before/after comparison is
the difference between the expected outcomes in the treated group after
the treatment and the expected outcomes in the same group before the
treatment:

\begin{align*}
\Delta^Y_{BA} & =  \esp{Y_i|D_i=1}-\esp{Y^B_i|D_i=1}.
\end{align*}
\EndKnitrBlock{definition}

\BeginKnitrBlock{example}
\protect\hypertarget{exm:unnamed-chunk-25}{}{\label{exm:unnamed-chunk-25}
}In the population, the \(BA\) estimator has the following shape:

\begin{align*}
  \Delta^y_{BA} & = \esp{y_i|D_i=1}-\esp{y^B_i|D_i=1}\\
                & = \esp{y^1_i-y^B_i|D_i=1}\\
                & = \esp{\alpha_i|D_i=1} + \delta + (\rho-1)\esp{U_i^B|\mu_i+U_i^B\leq\bar{y}}\\
                & = \Delta^y_{TT} + \delta + (1-\rho)\left(\frac{\sigma^2_{U}}{\sqrt{\sigma^2_{\mu}+\sigma^2_{U}}}\frac{\phi\left(\frac{\bar{y}-\bar{\mu}}{\sqrt{\sigma^2_{\mu}+\sigma^2_{U}}}\right)}{\Phi\left(\frac{\bar{y}-\bar{\mu}}{\sqrt{\sigma^2_{\mu}+\sigma^2_{U}}}\right)}\right).
\end{align*}
\EndKnitrBlock{example}

In order to compute \(BA\) in the population, we can again use a R
function, combining the value of \(TT\) and that of the second part of
the formula, that we are going to denote \(TB\) for reasons that are
going to become clear in a bit.

\begin{Shaded}
\begin{Highlighting}[]
\NormalTok{delta.y.tb <-}\StringTok{ }\ControlFlowTok{function}\NormalTok{(param)\{}
  \KeywordTok{return}\NormalTok{(param[}\StringTok{"delta"}\NormalTok{]}
          \OperatorTok{+}\NormalTok{(}\DecValTok{1}\OperatorTok{-}\NormalTok{param[}\StringTok{"rho"}\NormalTok{])}\OperatorTok{*}\NormalTok{((param[}\StringTok{"sigma2U"}\NormalTok{])}\OperatorTok{/}\KeywordTok{sqrt}\NormalTok{(param[}\StringTok{"sigma2mu"}\NormalTok{]}\OperatorTok{+}\NormalTok{param[}\StringTok{"sigma2U"}\NormalTok{]))}
         \OperatorTok{*}\KeywordTok{dnorm}\NormalTok{((}\KeywordTok{log}\NormalTok{(param[}\StringTok{"barY"}\NormalTok{])}\OperatorTok{-}\NormalTok{param[}\StringTok{"barmu"}\NormalTok{])}\OperatorTok{/}\NormalTok{(}\KeywordTok{sqrt}\NormalTok{(param[}\StringTok{"sigma2mu"}\NormalTok{]}\OperatorTok{+}\NormalTok{param[}\StringTok{"sigma2U"}\NormalTok{])))}
         \OperatorTok{/}\KeywordTok{pnorm}\NormalTok{((}\KeywordTok{log}\NormalTok{(param[}\StringTok{"barY"}\NormalTok{])}\OperatorTok{-}\NormalTok{param[}\StringTok{"barmu"}\NormalTok{])}\OperatorTok{/}\NormalTok{(}\KeywordTok{sqrt}\NormalTok{(param[}\StringTok{"sigma2mu"}\NormalTok{]}\OperatorTok{+}\NormalTok{param[}\StringTok{"sigma2U"}\NormalTok{]))))}
\NormalTok{\}}
\NormalTok{delta.y.ba <-}\StringTok{ }\ControlFlowTok{function}\NormalTok{(param)\{}
  \KeywordTok{return}\NormalTok{(}\KeywordTok{delta.y.tt}\NormalTok{(param)}\OperatorTok{+}\StringTok{ }\KeywordTok{delta.y.tb}\NormalTok{(param))}
\NormalTok{\}}
\end{Highlighting}
\end{Shaded}

The value of \(BA\) in the population is thus \(\Delta^y_{BA}=\) 0.265.
Remember that the true value of \(TT\) in the population is 0.172. In
the sample, the value of \(BA\) is \(\hat{\Delta^y_{BA}}=\) 0.267.
Remember that the value of \(TT\) in the sample is \(\Delta^y_{TT_s}=\)
0.168.

\subsubsection{Time trend bias}\label{time-trend-bias}

When we form the before/after comparison, we do not recover the \(TT\)
parameter. Instead, we recover \(TT\) plus a bias term, called
\textbf{time trend bias}:

\begin{align*}
\Delta^Y_{BA} & =\Delta^Y_{TT}+\Delta^Y_{TB}.
\end{align*}

\BeginKnitrBlock{definition}[Time trend bias]
\protect\hypertarget{def:unnamed-chunk-26}{}{\label{def:unnamed-chunk-26}
\iffalse (Time trend bias) \fi{} }Time trend bias is the difference
between the before/after comparison and the treatment on the treated
parameter:

\begin{align*}
\Delta^Y_{TB} & = \Delta^Y_{BA}-\Delta^Y_{TT} .
\end{align*}
\EndKnitrBlock{definition}

\(BA\) uses the expected outcome in the treated group before the
treatment as a proxy for the expected counterfactual outcome in the
absence of the treatment in the same group. \(TB\) is due to the fact
that \(BA\) uses an imperfect proxy for the counterfactual expected
outcome of the treated:

\BeginKnitrBlock{theorem}
\protect\hypertarget{thm:TB}{}{\label{thm:TB} }Time trend bias is the
difference between the counterfactual expected potential outcome in the
absence of the treatment among the treated and the expected outcome
before the treatment in the same group.

\begin{align*}
\Delta^Y_{TB} & = \esp{Y_i^0|D_i=1}-\esp{Y_i^B|D_i=1}.
\end{align*}
\EndKnitrBlock{theorem}

\BeginKnitrBlock{proof}
\iffalse{} {Proof. } \fi{}

\begin{align*}
\Delta^Y_{TB} & = \Delta^Y_{BA}-\Delta^Y_{TT} \\
              & = \esp{Y_i|D_i=1}-\esp{Y^B_i|D_i=1}-\esp{Y_i^1-Y_i^0|D_i=1}\\
              & = \esp{Y_i^0|D_i=1}-\esp{Y_i^B|D_i=1}.
\end{align*}

The first and second equalities stem from the definition of both
parameters. The third equality stems from using the switching equation:
\(Y_i=Y_i^1D_i+Y_i^0(1-D_i)\), so that
\(\esp{Y_i|D_i=1}=\esp{Y^1_i|D_i=1}\).
\EndKnitrBlock{proof}

\BeginKnitrBlock{example}
\protect\hypertarget{exm:unnamed-chunk-28}{}{\label{exm:unnamed-chunk-28}
}In the population, \(TB\) is equal to
\EndKnitrBlock{example} \(\Delta^y_{TB}=\Delta^y_{BA}-\Delta^y_{TT}=\)
0.265 \(-\) 0.172 \(=\) 0.093. We could have computed this result using
Theorem\ref{thm:TB}:

\begin{align*}
\Delta^y_{TB} & = \esp{y_i^0|D_i=1}-\esp{y_i^B|D_i=1} \\
              & = \delta + (1-\rho)\left(\frac{\sigma^2_{U}}{\sqrt{\sigma^2_{\mu}+\sigma^2_{U}}}\frac{\phi\left(\frac{\bar{y}-\bar{\mu}}{\sqrt{\sigma^2_{\mu}+\sigma^2_{U}}}\right)}{\Phi\left(\frac{\bar{y}-\bar{\mu}}{\sqrt{\sigma^2_{\mu}+\sigma^2_{U}}}\right)}\right).
\end{align*}

Using the R function that we have defined previously, this approach
gives \(\Delta^y_{TB}=\) 0.093.

In the sample \(\hat{\Delta^y_{BA}}=\) 0.267 while
\(\hat{\Delta^y_{TT}}=\) 0.168, so that \(\hat{\Delta^y_{TB}}=\) 0.099.
Time trend bias emerges because we are using a bad proxy for the
counterfactual average outcomes of the treated. The average outcome of
the treated before the treatment takes place is
\(\hat{\esp{y_i^B|D_i=1}}=\) 6.807 while the true counterfactual average
outcome for the treated after the treatment is
\(\hat{\esp{y_i^0|D_i=1}}=\) 6.906. Outcomes would have increased in the
treatment group even in the absence of the treatment. As a consequence,
the \(BA\) comparison overestimates the true effect of the treatment.
\(TB\) estimated using Theorem \ref{thm:TB} is equal to:
\(\hat{\Delta^y_{TB}}=\) 6.906 \(-\) 6.807 \(=\) 0.099. This can be seen
on Figure \ref{fig:ploty0y1yBD}: the triangle in period 2 is higher than
in period 1.

\subsubsection{Temporal confounders}\label{temporal-confounders}

Temporal confounders make the outcomes of the treated change at the same
time as the treatment status changes, thereby confounding the effect of
the treatment. Temporal confounders are responsible for time trend bias.

Over time, there are other things that change than the treatment status.
For example, maybe sick individuals naturally recover, and thus their
counterfactual health status is better than ther health status before
taking the treatment. As a results, \(BA\) might overstimate the effect
of the treatment. It might also be that the overall level of health in
the country has increased, because of increasing GDP, for example.

\BeginKnitrBlock{example}
\protect\hypertarget{exm:unnamed-chunk-29}{}{\label{exm:unnamed-chunk-29}
}In our example, \(\delta\) and \(U_i^B\) are the confounders.
\(\delta\) captures the overall changes in outcomes over time (business
cycle, general improvement of health status). \(U^B_i\) captures the
fact that transitorily sicker individuals tend at the same time to
receive the treatment and also to recover naturally. The \(BA\)
comparison incorrectly attributes both of these changes to the effect of
the treatment. We can compute the relative contributions of both sources
to the overall time-trend bias in the population.
\EndKnitrBlock{example} \(\delta\) contributes for 0.05 while \(U^B_i\)
contributes for 0.043.

\subsubsection{\texorpdfstring{When does \(BA\) identify
\(TT\)?}{When does BA identify TT?}}\label{when-does-ba-identify-tt}

Are there conditions under which \(BA\) actually identifies \(TT\)? The
answer is yes, when there are no temporal confounders. When that is the
case, the variation of outcomes over time is only due to the treatment
and it identifies the treatment effect.

More formally, we make the following assumption:

\BeginKnitrBlock{definition}[No time trend bias]
\protect\hypertarget{def:notb}{}{\label{def:notb} \iffalse (No time trend
bias) \fi{} }We assume the following:

\begin{align*}
\esp{Y_i^0|D_i=1} & = \esp{Y_i^B|D_i=1}.
\end{align*}
\EndKnitrBlock{definition}

Under Assumption \ref{def:notb}, the expected counterfactual outcome of
the treated is equal to the expected potential outcome of the untreated
in the absence of the treatment. This yields to the following result:

\BeginKnitrBlock{theorem}
\protect\hypertarget{thm:batt}{}{\label{thm:batt} }Under Assumption
\ref{def:notb}, \(BA\) identifies the \(TT\) parameter:

\begin{align*}
\Delta^Y_{BA} & = \Delta^Y_{TT}.
\end{align*}
\EndKnitrBlock{theorem}

\BeginKnitrBlock{proof}
\iffalse{} {Proof. } \fi{}

\begin{align*}
\Delta^Y_{BA} & = \esp{Y_i|D_i=1}-\esp{Y_i^B|D_i=1}\\
              & = \esp{Y_i^1|D_i=1}-\esp{Y_i^B|D_i=1}\\
              & = \esp{Y_i^1|D_i=1}-\esp{Y_i^0|D_i=1} \\
              & = \Delta^Y_{TT},
\end{align*}

where the second equation uses the switching equation and the third uses
Assumption \ref{def:notb}.
\EndKnitrBlock{proof}

Under Assumption \ref{def:notb} the outcomes of the treated before the
treatment takes place are a good proxy for the counterfactual. As a
consequence, \(BA\) identifies \(TT\). Under Assumption \ref{def:notb}is
very unlikely to hold in real life. Indeed, it requires tha nothing
happens to the outcomes of the treated in the absence of the treatment.
Assumption \ref{def:notb} rules out economy-wide shocks but also
mean-reversion, such as when sick people naturally recover from an
illness.

A good way to test for the validity of Assumption \ref{def:notb} is to
perform a placebo test. Two of these tests are possible. One placebo
test would be to apply the \(BA\) estimator between two pre-treatment
periods where nothing should happen since the treatment status does not
vary and, by assumption, nothing else should vary. A second placebo test
would be to apply the \(BA\) estimator to a group that does not receive
the treatment. The untreated group is a perfectly suitable candidate for
that. Assumption \ref{def:notb} does not imply that there should be no
change in the untreated outcomes, but detecting such a change would cast
a serious doubt on the validity of Assumption \ref{def:notb}.

\BeginKnitrBlock{example}
\protect\hypertarget{exm:unnamed-chunk-31}{}{\label{exm:unnamed-chunk-31}
}One way to generate a population in which Assumption \ref{def:notb}
holds is to shut down the two sources of confounders in our original
model by setting both \(\delta=0\) and \(\rho=1\).
\EndKnitrBlock{example}

\begin{Shaded}
\begin{Highlighting}[]
\NormalTok{param <-}\StringTok{ }\KeywordTok{c}\NormalTok{(}\DecValTok{8}\NormalTok{,}\FloatTok{0.5}\NormalTok{,.}\DecValTok{28}\NormalTok{,}\DecValTok{1500}\NormalTok{,}\DecValTok{1}\NormalTok{,}\FloatTok{0.01}\NormalTok{,}\FloatTok{0.05}\NormalTok{,}\FloatTok{0.05}\NormalTok{,}\DecValTok{0}\NormalTok{,}\FloatTok{0.1}\NormalTok{)}
\KeywordTok{names}\NormalTok{(param) <-}\StringTok{ }\KeywordTok{c}\NormalTok{(}\StringTok{"barmu"}\NormalTok{,}\StringTok{"sigma2mu"}\NormalTok{,}\StringTok{"sigma2U"}\NormalTok{,}\StringTok{"barY"}\NormalTok{,}\StringTok{"rho"}\NormalTok{,}\StringTok{"theta"}\NormalTok{,}\StringTok{"sigma2epsilon"}\NormalTok{,}\StringTok{"sigma2eta"}\NormalTok{,}\StringTok{"delta"}\NormalTok{,}\StringTok{"baralpha"}\NormalTok{)}
\NormalTok{param}
\end{Highlighting}
\end{Shaded}

\begin{verbatim}
##         barmu      sigma2mu       sigma2U          barY           rho 
##          8.00          0.50          0.28       1500.00          1.00 
##         theta sigma2epsilon     sigma2eta         delta      baralpha 
##          0.01          0.05          0.05          0.00          0.10
\end{verbatim}

In that case, according to the formula we have derived for \(TB\), we
have: \(\Delta^y_{TB}=\) 0. Let's see how these quantities behave in the
sample:

\begin{Shaded}
\begin{Highlighting}[]
\KeywordTok{set.seed}\NormalTok{(}\DecValTok{1234}\NormalTok{)}
\NormalTok{mu <-}\StringTok{ }\KeywordTok{rnorm}\NormalTok{(N,param[}\StringTok{"barmu"}\NormalTok{],}\KeywordTok{sqrt}\NormalTok{(param[}\StringTok{"sigma2mu"}\NormalTok{]))}
\NormalTok{UB <-}\StringTok{ }\KeywordTok{rnorm}\NormalTok{(N,}\DecValTok{0}\NormalTok{,}\KeywordTok{sqrt}\NormalTok{(param[}\StringTok{"sigma2U"}\NormalTok{]))}
\NormalTok{yB <-}\StringTok{ }\NormalTok{mu }\OperatorTok{+}\StringTok{ }\NormalTok{UB }
\NormalTok{YB <-}\StringTok{ }\KeywordTok{exp}\NormalTok{(yB)}
\NormalTok{Ds <-}\StringTok{ }\KeywordTok{rep}\NormalTok{(}\DecValTok{0}\NormalTok{,N)}
\NormalTok{Ds[YB}\OperatorTok{<=}\NormalTok{param[}\StringTok{"barY"}\NormalTok{]] <-}\StringTok{ }\DecValTok{1} 
\NormalTok{epsilon <-}\StringTok{ }\KeywordTok{rnorm}\NormalTok{(N,}\DecValTok{0}\NormalTok{,}\KeywordTok{sqrt}\NormalTok{(param[}\StringTok{"sigma2epsilon"}\NormalTok{]))}
\NormalTok{eta<-}\StringTok{ }\KeywordTok{rnorm}\NormalTok{(N,}\DecValTok{0}\NormalTok{,}\KeywordTok{sqrt}\NormalTok{(param[}\StringTok{"sigma2eta"}\NormalTok{]))}
\NormalTok{U0 <-}\StringTok{ }\NormalTok{param[}\StringTok{"rho"}\NormalTok{]}\OperatorTok{*}\NormalTok{UB }\OperatorTok{+}\StringTok{ }\NormalTok{epsilon}
\NormalTok{y0 <-}\StringTok{ }\NormalTok{mu }\OperatorTok{+}\StringTok{  }\NormalTok{U0 }\OperatorTok{+}\StringTok{ }\NormalTok{param[}\StringTok{"delta"}\NormalTok{]}
\NormalTok{alpha <-}\StringTok{ }\NormalTok{param[}\StringTok{"baralpha"}\NormalTok{]}\OperatorTok{+}\StringTok{  }\NormalTok{param[}\StringTok{"theta"}\NormalTok{]}\OperatorTok{*}\NormalTok{mu }\OperatorTok{+}\StringTok{ }\NormalTok{eta}
\NormalTok{y1 <-}\StringTok{ }\NormalTok{y0}\OperatorTok{+}\NormalTok{alpha}
\NormalTok{Y0 <-}\StringTok{ }\KeywordTok{exp}\NormalTok{(y0)}
\NormalTok{Y1 <-}\StringTok{ }\KeywordTok{exp}\NormalTok{(y1)}
\NormalTok{y <-}\StringTok{ }\NormalTok{y1}\OperatorTok{*}\NormalTok{Ds}\OperatorTok{+}\NormalTok{y0}\OperatorTok{*}\NormalTok{(}\DecValTok{1}\OperatorTok{-}\NormalTok{Ds)}
\NormalTok{Y <-}\StringTok{ }\NormalTok{Y1}\OperatorTok{*}\NormalTok{Ds}\OperatorTok{+}\NormalTok{Y0}\OperatorTok{*}\NormalTok{(}\DecValTok{1}\OperatorTok{-}\NormalTok{Ds)}
\end{Highlighting}
\end{Shaded}

In the sample, the value of \(BA\) is \(\hat{\Delta^y_{BA}}=\) 0.173
while the value of \(TT\) in the sample is \(\Delta^y_{TT_s}=\) 0.168.
We cannot perform a placebo test using two periods of pre-treatment
outcomes for the treated since we have generated only one period of
pre-treatment outcome. We will be able to perform this test later in the
DID lecture. We can perfom the placebo test that applies the \(BA\)
estimator to the untreated. The value of \(BA\) for the untreated is
\(\hat{\Delta^y_{BA|D=0}}=\) 0.007, which is reasonably close to zero.

\chapter{The Fundamental Problem of Statistical
Inference}\label{the-fundamental-problem-of-statistical-inference}

The Fundamental Problem of Statistical Inference (FPSI) states that,
even if we have an estimator \(E\) that identifies \(TT\) in the
population, we cannot observe \(E\) because we only have access to a
finite sample of the population. The only thing that we can form from
the sample is a sample equivalent \(\hat{E}\) to the population quantity
\(E\), and \(\hat{E}\neq E\). For example, the sample analog to \(WW\)
is the difference in means between treated and untreated units
\(\hat{WW}\). As we saw in the last lecture, \(\hat{WW}\) is never
exactly equal to \(WW\).

Why is \(\hat{E}\neq E\)? Because a finite sample is never perfectly
representative of the population. In a sample, even in a random sample,
the distribution of the observed and unobserved covariates deviates from
the true population one. As a consequence, the sample value of the
estimator is never precisely equal to the population value, but
fluctuates around it with sampling noise. The main problem with the FPSI
is that if we find an effect of our treatment, be it small or large, we
cannot know whether we should attribute it to the treatment or to the
bad or good luck of sampling noise.

What can we do to deal with the FPSI? I am going to argue that there are
mainly two things that we might want to do: estimating the extent of
sampling noise and decreasing sampling noise.

Estimating sampling noise means measuring how much variability there is
in our estimate \(\hat{E}\) due to the sampling procedure. This is very
useful because it enables us to form a confidence interval that gauges
how far from \(\hat{E}\) the true value \(E\) might be. It is a measure
of the precision of our estimation and of the extent to which sampling
noise might drive our results. Estimating sampling noise is very hard
because we have only access to one sample and we would like to know the
behavior of our estimator over repeated samples. We are going to learn
four ways to estimate the extent of sampling noise using data from one
sample.

Because sampling noise is such a nuisance and makes our estimates
imprecise, we would like to be able to make it as small as possible. We
are going to study three ways of decreasing sampling noise, two that
take place before collecting the data (increasing sample size,
stratifying) and one that takes place after (conditioning).

Maybe you are surprised not to find statistical significance tests as an
important answer to the FPSI. I argue in this lecture that statistical
tests are misleading tools that make us overestimate the confidence in
our results and underestimate the scope of sampling noise. Statistical
tests are not meant to be used for scientific research, but were
originally designed to make decisions in industrial settings where the
concept of successive sampling made actual sense. Statistical tests also
generate collective behaviors such as publication bias and specification
search that undermine the very foundations of science. A general
movement in the social sciences, but also in physics, is starting to ban
the reporting of p-values.

\section{What is sampling noise? Definition and
illustration}\label{sec:sampnoise}

In this section, I am going to define sampling noise and illustrate it
with a numerical example. In Section \ref{sec:definitionnoise}, I define
sampling noise. In section \ref{sec:illusnoisepop}, I illustrate how
sampling noise varies when one is interested in the population treatment
effect. In section \ref{sec:illusnoisesamp}, I illustrate how sampling
noise varies when one is interesetd in the sample treatment effect.
Finally, in section \ref{sec:confinterv}, I show how confidence
intervals can be built from an estimate of sampling noise.

\subsection{Sampling noise, a definition}\label{sec:definitionnoise}

Sampling noise measures how much sampling variability moves the sample
estimator \(\hat{E}\) around. One way to define it more rigorously is to
make it equal to the width of a confidence interval:

\BeginKnitrBlock{definition}[Sampling noise]
\protect\hypertarget{def:sampnoise}{}{\label{def:sampnoise}
\iffalse (Sampling noise) \fi{} }Sampling noise is the width of the
symmetric interval around TT within which \(\delta*100\)\% of the sample
estimators fall, where \(\delta\) is the confidence level. As a
consequence, sampling noise is equal to \(2\epsilon\) where \(\epsilon\)
is such that:

\begin{align*}
\Pr(|\hat{E}-TT|\leq\epsilon) &= \delta.
\end{align*}
\EndKnitrBlock{definition}

This definition tries to capture the properties of the distribution of
\(\hat{E}\) using only one number. As every simplification, it leaves
room for dissatisfaction, exactly as a 2D map is a convenient albeit
arbitrary betrayal of a 3D phenomenon. For example, there is nothing
sacred about the symmetry of the interval. It is just extremely
convenient. One might prefer an interval that is symmetric in tail
probabilities instead. Feel free to explore with different concepts if
you like.

A related concept to that of sampling noise is that of precision: the
smaller the sampling noise, the higher the precision. Precision can be
defined for example as the inverse of sampling noise
\(\frac{1}{2\epsilon}\).

Finally, a very useful concept is that of signal to noise ratio. It is
not used in economics, but physicists use this concept all the time. The
signal to noise ratio measures the treatment effect in multiple of the
sampling noise. If they are of the same order of magnitude, we have a
lot of noise and little confidence in our estimates. If the signal is
much larger than the noise, we tend to have a lot of confidence in our
parameter estimates. The signal to noise ratio can be computed as
follows: \(\frac{E}{2\epsilon}\) or \(\frac{\hat{E}}{2\epsilon}\).

\BeginKnitrBlock{remark}
\iffalse{} {Remark. } \fi{}A very striking result is that the signal to
noise ratio of a result that is marginally significant at the 5\% level
is very small, around one half, meaning that the noise is generally
double the signal in these results. We will derive this result after
studying how to estimate sampling noise with real data.
\EndKnitrBlock{remark}

There are two distinct ways of understanding sampling noise, depending
on whether we are after the population treatment effect
(\(\Delta^Y_{TT}\)) or the sample treatment effect
(\(\Delta^Y_{TT_s}\)). Sampling noise for the population treatment
effect stems from the fact that the sample is not perfectly
representative of the population. The sample differs from the population
and thus the sample estimates differs from the population estimate.
Sampling noise for the sample parameter stems from the fact that the
control group is not a perfect embodiment of the counterfactual.
Discrepancies between treated and control samples are going to generate
differences between the \(WW\) estimate and the \(TT\) effect in the
sample.

\subsection{Sampling noise for the population treatment
effect}\label{sec:illusnoisepop}

Sampling noise for the population treatment effect stems from the fact
that the sample is not perfectly representative of the population.

\BeginKnitrBlock{example}
\protect\hypertarget{exm:unnamed-chunk-33}{}{\label{exm:unnamed-chunk-33}
}In order to assess the scope of sampling noise for our population
treatment effect estimate, let's first draw a sample. In order to be
able to do that, I first have to define the parameter values:
\EndKnitrBlock{example}

\begin{Shaded}
\begin{Highlighting}[]
\NormalTok{param <-}\StringTok{ }\KeywordTok{c}\NormalTok{(}\DecValTok{8}\NormalTok{,.}\DecValTok{5}\NormalTok{,.}\DecValTok{28}\NormalTok{,}\DecValTok{1500}\NormalTok{,}\FloatTok{0.9}\NormalTok{,}\FloatTok{0.01}\NormalTok{,}\FloatTok{0.05}\NormalTok{,}\FloatTok{0.05}\NormalTok{,}\FloatTok{0.05}\NormalTok{,}\FloatTok{0.1}\NormalTok{)}
\KeywordTok{names}\NormalTok{(param) <-}\StringTok{ }\KeywordTok{c}\NormalTok{(}\StringTok{"barmu"}\NormalTok{,}\StringTok{"sigma2mu"}\NormalTok{,}\StringTok{"sigma2U"}\NormalTok{,}\StringTok{"barY"}\NormalTok{,}\StringTok{"rho"}\NormalTok{,}\StringTok{"theta"}\NormalTok{,}\StringTok{"sigma2epsilon"}\NormalTok{,}\StringTok{"sigma2eta"}\NormalTok{,}\StringTok{"delta"}\NormalTok{,}\StringTok{"baralpha"}\NormalTok{)}
\NormalTok{param}
\end{Highlighting}
\end{Shaded}

\begin{verbatim}
##         barmu      sigma2mu       sigma2U          barY           rho 
##          8.00          0.50          0.28       1500.00          0.90 
##         theta sigma2epsilon     sigma2eta         delta      baralpha 
##          0.01          0.05          0.05          0.05          0.10
\end{verbatim}

\begin{Shaded}
\begin{Highlighting}[]
\KeywordTok{set.seed}\NormalTok{(}\DecValTok{1234}\NormalTok{)}
\NormalTok{N <-}\DecValTok{1000}
\NormalTok{mu <-}\StringTok{ }\KeywordTok{rnorm}\NormalTok{(N,param[}\StringTok{"barmu"}\NormalTok{],}\KeywordTok{sqrt}\NormalTok{(param[}\StringTok{"sigma2mu"}\NormalTok{]))}
\NormalTok{UB <-}\StringTok{ }\KeywordTok{rnorm}\NormalTok{(N,}\DecValTok{0}\NormalTok{,}\KeywordTok{sqrt}\NormalTok{(param[}\StringTok{"sigma2U"}\NormalTok{]))}
\NormalTok{yB <-}\StringTok{ }\NormalTok{mu }\OperatorTok{+}\StringTok{ }\NormalTok{UB }
\NormalTok{YB <-}\StringTok{ }\KeywordTok{exp}\NormalTok{(yB)}
\NormalTok{Ds <-}\StringTok{ }\KeywordTok{rep}\NormalTok{(}\DecValTok{0}\NormalTok{,N)}
\NormalTok{V <-}\StringTok{ }\KeywordTok{rnorm}\NormalTok{(N,param[}\StringTok{"barmu"}\NormalTok{],}\KeywordTok{sqrt}\NormalTok{(param[}\StringTok{"sigma2mu"}\NormalTok{]}\OperatorTok{+}\NormalTok{param[}\StringTok{"sigma2U"}\NormalTok{]))}
\NormalTok{Ds[V}\OperatorTok{<=}\KeywordTok{log}\NormalTok{(param[}\StringTok{"barY"}\NormalTok{])] <-}\StringTok{ }\DecValTok{1} 
\NormalTok{epsilon <-}\StringTok{ }\KeywordTok{rnorm}\NormalTok{(N,}\DecValTok{0}\NormalTok{,}\KeywordTok{sqrt}\NormalTok{(param[}\StringTok{"sigma2epsilon"}\NormalTok{]))}
\NormalTok{eta<-}\StringTok{ }\KeywordTok{rnorm}\NormalTok{(N,}\DecValTok{0}\NormalTok{,}\KeywordTok{sqrt}\NormalTok{(param[}\StringTok{"sigma2eta"}\NormalTok{]))}
\NormalTok{U0 <-}\StringTok{ }\NormalTok{param[}\StringTok{"rho"}\NormalTok{]}\OperatorTok{*}\NormalTok{UB }\OperatorTok{+}\StringTok{ }\NormalTok{epsilon}
\NormalTok{y0 <-}\StringTok{ }\NormalTok{mu }\OperatorTok{+}\StringTok{  }\NormalTok{U0 }\OperatorTok{+}\StringTok{ }\NormalTok{param[}\StringTok{"delta"}\NormalTok{]}
\NormalTok{alpha <-}\StringTok{ }\NormalTok{param[}\StringTok{"baralpha"}\NormalTok{]}\OperatorTok{+}\StringTok{  }\NormalTok{param[}\StringTok{"theta"}\NormalTok{]}\OperatorTok{*}\NormalTok{mu }\OperatorTok{+}\StringTok{ }\NormalTok{eta}
\NormalTok{y1 <-}\StringTok{ }\NormalTok{y0}\OperatorTok{+}\NormalTok{alpha}
\NormalTok{Y0 <-}\StringTok{ }\KeywordTok{exp}\NormalTok{(y0)}
\NormalTok{Y1 <-}\StringTok{ }\KeywordTok{exp}\NormalTok{(y1)}
\NormalTok{y <-}\StringTok{ }\NormalTok{y1}\OperatorTok{*}\NormalTok{Ds}\OperatorTok{+}\NormalTok{y0}\OperatorTok{*}\NormalTok{(}\DecValTok{1}\OperatorTok{-}\NormalTok{Ds)}
\NormalTok{Y <-}\StringTok{ }\NormalTok{Y1}\OperatorTok{*}\NormalTok{Ds}\OperatorTok{+}\NormalTok{Y0}\OperatorTok{*}\NormalTok{(}\DecValTok{1}\OperatorTok{-}\NormalTok{Ds)}

\NormalTok{delta.y.ate <-}\StringTok{ }\ControlFlowTok{function}\NormalTok{(param)\{}
  \KeywordTok{return}\NormalTok{(param[}\StringTok{"baralpha"}\NormalTok{]}\OperatorTok{+}\NormalTok{param[}\StringTok{"theta"}\NormalTok{]}\OperatorTok{*}\NormalTok{param[}\StringTok{"barmu"}\NormalTok{])}
\NormalTok{\}}
\end{Highlighting}
\end{Shaded}

In this sample, the \(WW\) estimator yields an estimate of
\(\hat{\Delta^y_{WW}}=\) 0.133. Despite random assignment, we have
\(\hat{\Delta^y_{WW}}\neq\Delta^y_{TT}=\) 0.18, an instance of the FPSI.

In order to see how sampling noise varies, let's draw another sample. In
order to do so, I am going to choose a different seed to initialize the
pseudo-random number generator in R.

\begin{Shaded}
\begin{Highlighting}[]
\KeywordTok{set.seed}\NormalTok{(}\DecValTok{12345}\NormalTok{)}
\NormalTok{N <-}\DecValTok{1000}
\NormalTok{mu <-}\StringTok{ }\KeywordTok{rnorm}\NormalTok{(N,param[}\StringTok{"barmu"}\NormalTok{],}\KeywordTok{sqrt}\NormalTok{(param[}\StringTok{"sigma2mu"}\NormalTok{]))}
\NormalTok{UB <-}\StringTok{ }\KeywordTok{rnorm}\NormalTok{(N,}\DecValTok{0}\NormalTok{,}\KeywordTok{sqrt}\NormalTok{(param[}\StringTok{"sigma2U"}\NormalTok{]))}
\NormalTok{yB <-}\StringTok{ }\NormalTok{mu }\OperatorTok{+}\StringTok{ }\NormalTok{UB }
\NormalTok{YB <-}\StringTok{ }\KeywordTok{exp}\NormalTok{(yB)}
\NormalTok{Ds <-}\StringTok{ }\KeywordTok{rep}\NormalTok{(}\DecValTok{0}\NormalTok{,N)}
\NormalTok{V <-}\StringTok{ }\KeywordTok{rnorm}\NormalTok{(N,param[}\StringTok{"barmu"}\NormalTok{],}\KeywordTok{sqrt}\NormalTok{(param[}\StringTok{"sigma2mu"}\NormalTok{]}\OperatorTok{+}\NormalTok{param[}\StringTok{"sigma2U"}\NormalTok{]))}
\NormalTok{Ds[V}\OperatorTok{<=}\KeywordTok{log}\NormalTok{(param[}\StringTok{"barY"}\NormalTok{])] <-}\StringTok{ }\DecValTok{1} 
\NormalTok{epsilon <-}\StringTok{ }\KeywordTok{rnorm}\NormalTok{(N,}\DecValTok{0}\NormalTok{,}\KeywordTok{sqrt}\NormalTok{(param[}\StringTok{"sigma2epsilon"}\NormalTok{]))}
\NormalTok{eta<-}\StringTok{ }\KeywordTok{rnorm}\NormalTok{(N,}\DecValTok{0}\NormalTok{,}\KeywordTok{sqrt}\NormalTok{(param[}\StringTok{"sigma2eta"}\NormalTok{]))}
\NormalTok{U0 <-}\StringTok{ }\NormalTok{param[}\StringTok{"rho"}\NormalTok{]}\OperatorTok{*}\NormalTok{UB }\OperatorTok{+}\StringTok{ }\NormalTok{epsilon}
\NormalTok{y0 <-}\StringTok{ }\NormalTok{mu }\OperatorTok{+}\StringTok{  }\NormalTok{U0 }\OperatorTok{+}\StringTok{ }\NormalTok{param[}\StringTok{"delta"}\NormalTok{]}
\NormalTok{alpha <-}\StringTok{ }\NormalTok{param[}\StringTok{"baralpha"}\NormalTok{]}\OperatorTok{+}\StringTok{  }\NormalTok{param[}\StringTok{"theta"}\NormalTok{]}\OperatorTok{*}\NormalTok{mu }\OperatorTok{+}\StringTok{ }\NormalTok{eta}
\NormalTok{y1 <-}\StringTok{ }\NormalTok{y0}\OperatorTok{+}\NormalTok{alpha}
\NormalTok{Y0 <-}\StringTok{ }\KeywordTok{exp}\NormalTok{(y0)}
\NormalTok{Y1 <-}\StringTok{ }\KeywordTok{exp}\NormalTok{(y1)}
\NormalTok{y <-}\StringTok{ }\NormalTok{y1}\OperatorTok{*}\NormalTok{Ds}\OperatorTok{+}\NormalTok{y0}\OperatorTok{*}\NormalTok{(}\DecValTok{1}\OperatorTok{-}\NormalTok{Ds)}
\NormalTok{Y <-}\StringTok{ }\NormalTok{Y1}\OperatorTok{*}\NormalTok{Ds}\OperatorTok{+}\NormalTok{Y0}\OperatorTok{*}\NormalTok{(}\DecValTok{1}\OperatorTok{-}\NormalTok{Ds)}
\end{Highlighting}
\end{Shaded}

In this sample, the \(WW\) estimator yields an estimate of
\(\hat{\Delta^y_{WW}}=\) 0.179. Again, despite random assignment, we
have \(\hat{\Delta^y_{WW}}\neq\Delta^y_{TT}=\) 0.18, an instance of the
FPSI. Furthermore, the estimate of the population treatment effect in
this sample differs from the previous one, a consequence of sampling
noise.

Let's now visualize the extent of sampling noise by repeating the
procedure multiple times with various sample sizes. This is called Monte
Carlo replications: in each replication, I choose a sample size, draw
one sample from the population and compute the \(\hat{WW}\) estimator.
At each replication, the sample I'm using is different, reflecting the
actual sampling process and enabling me to gauge the extent of sampling
noise. In order to focus on sampling noise alone, I am running the
replications in the model in which selection into the treatment is
independent on potential outcomes, so that \(WW=TT\) in the population.
In order to speed up the process, I am using parallelized computing: I
send each sample to a different core in my computer so that several
samples can be run at the same time. You might want to adapt the program
below to the number of cores you actually have using the \texttt{ncpus}
variable in the beginning of the \texttt{.Rmd} file that generates this
page.. In order to parallelize computations, I use the Snowfall package
in R, that gives very simple and intuitive parallelization commands. In
order to save time when generating the graph, I use the wonderful
``cache'' option of knitr: it stores the estimates from the code chunk
and will not rerun it as long as the code inside the chunk has not been
altered nor the code of the chunks that it depends on (parameter values,
for example).

\begin{Shaded}
\begin{Highlighting}[]
\NormalTok{monte.carlo.ww <-}\StringTok{ }\ControlFlowTok{function}\NormalTok{(s,N,param)\{}
  \KeywordTok{set.seed}\NormalTok{(s)}
\NormalTok{  mu <-}\StringTok{ }\KeywordTok{rnorm}\NormalTok{(N,param[}\StringTok{"barmu"}\NormalTok{],}\KeywordTok{sqrt}\NormalTok{(param[}\StringTok{"sigma2mu"}\NormalTok{]))}
\NormalTok{  UB <-}\StringTok{ }\KeywordTok{rnorm}\NormalTok{(N,}\DecValTok{0}\NormalTok{,}\KeywordTok{sqrt}\NormalTok{(param[}\StringTok{"sigma2U"}\NormalTok{]))}
\NormalTok{  yB <-}\StringTok{ }\NormalTok{mu }\OperatorTok{+}\StringTok{ }\NormalTok{UB }
\NormalTok{  YB <-}\StringTok{ }\KeywordTok{exp}\NormalTok{(yB)}
\NormalTok{  Ds <-}\StringTok{ }\KeywordTok{rep}\NormalTok{(}\DecValTok{0}\NormalTok{,N)}
\NormalTok{  V <-}\StringTok{ }\KeywordTok{rnorm}\NormalTok{(N,param[}\StringTok{"barmu"}\NormalTok{],}\KeywordTok{sqrt}\NormalTok{(param[}\StringTok{"sigma2mu"}\NormalTok{]}\OperatorTok{+}\NormalTok{param[}\StringTok{"sigma2U"}\NormalTok{]))}
\NormalTok{  Ds[V}\OperatorTok{<=}\KeywordTok{log}\NormalTok{(param[}\StringTok{"barY"}\NormalTok{])] <-}\StringTok{ }\DecValTok{1} 
\NormalTok{  epsilon <-}\StringTok{ }\KeywordTok{rnorm}\NormalTok{(N,}\DecValTok{0}\NormalTok{,}\KeywordTok{sqrt}\NormalTok{(param[}\StringTok{"sigma2epsilon"}\NormalTok{]))}
\NormalTok{  eta<-}\StringTok{ }\KeywordTok{rnorm}\NormalTok{(N,}\DecValTok{0}\NormalTok{,}\KeywordTok{sqrt}\NormalTok{(param[}\StringTok{"sigma2eta"}\NormalTok{]))}
\NormalTok{  U0 <-}\StringTok{ }\NormalTok{param[}\StringTok{"rho"}\NormalTok{]}\OperatorTok{*}\NormalTok{UB }\OperatorTok{+}\StringTok{ }\NormalTok{epsilon}
\NormalTok{  y0 <-}\StringTok{ }\NormalTok{mu }\OperatorTok{+}\StringTok{  }\NormalTok{U0 }\OperatorTok{+}\StringTok{ }\NormalTok{param[}\StringTok{"delta"}\NormalTok{]}
\NormalTok{  alpha <-}\StringTok{ }\NormalTok{param[}\StringTok{"baralpha"}\NormalTok{]}\OperatorTok{+}\StringTok{  }\NormalTok{param[}\StringTok{"theta"}\NormalTok{]}\OperatorTok{*}\NormalTok{mu }\OperatorTok{+}\StringTok{ }\NormalTok{eta}
\NormalTok{  y1 <-}\StringTok{ }\NormalTok{y0}\OperatorTok{+}\NormalTok{alpha}
\NormalTok{  Y0 <-}\StringTok{ }\KeywordTok{exp}\NormalTok{(y0)}
\NormalTok{  Y1 <-}\StringTok{ }\KeywordTok{exp}\NormalTok{(y1)}
\NormalTok{  y <-}\StringTok{ }\NormalTok{y1}\OperatorTok{*}\NormalTok{Ds}\OperatorTok{+}\NormalTok{y0}\OperatorTok{*}\NormalTok{(}\DecValTok{1}\OperatorTok{-}\NormalTok{Ds)}
\NormalTok{  Y <-}\StringTok{ }\NormalTok{Y1}\OperatorTok{*}\NormalTok{Ds}\OperatorTok{+}\NormalTok{Y0}\OperatorTok{*}\NormalTok{(}\DecValTok{1}\OperatorTok{-}\NormalTok{Ds)}
  \KeywordTok{return}\NormalTok{(}\KeywordTok{c}\NormalTok{((}\DecValTok{1}\OperatorTok{/}\KeywordTok{sum}\NormalTok{(Ds))}\OperatorTok{*}\KeywordTok{sum}\NormalTok{(y}\OperatorTok{*}\NormalTok{Ds)}\OperatorTok{-}\NormalTok{(}\DecValTok{1}\OperatorTok{/}\KeywordTok{sum}\NormalTok{(}\DecValTok{1}\OperatorTok{-}\NormalTok{Ds))}\OperatorTok{*}\KeywordTok{sum}\NormalTok{(y}\OperatorTok{*}\NormalTok{(}\DecValTok{1}\OperatorTok{-}\NormalTok{Ds)),}\KeywordTok{var}\NormalTok{(y[Ds}\OperatorTok{==}\DecValTok{1}\NormalTok{]),}\KeywordTok{var}\NormalTok{(y[Ds}\OperatorTok{==}\DecValTok{0}\NormalTok{]),}\KeywordTok{mean}\NormalTok{(Ds)))}
\NormalTok{\}}

\NormalTok{simuls.ww.N <-}\StringTok{ }\ControlFlowTok{function}\NormalTok{(N,Nsim,param)\{}
\NormalTok{  simuls.ww <-}\StringTok{ }\KeywordTok{as.data.frame}\NormalTok{(}\KeywordTok{matrix}\NormalTok{(}\KeywordTok{unlist}\NormalTok{(}\KeywordTok{lapply}\NormalTok{(}\DecValTok{1}\OperatorTok{:}\NormalTok{Nsim,monte.carlo.ww,}\DataTypeTok{N=}\NormalTok{N,}\DataTypeTok{param=}\NormalTok{param)),}\DataTypeTok{nrow=}\NormalTok{Nsim,}\DataTypeTok{ncol=}\DecValTok{4}\NormalTok{,}\DataTypeTok{byrow=}\OtherTok{TRUE}\NormalTok{))}
  \KeywordTok{colnames}\NormalTok{(simuls.ww) <-}\StringTok{ }\KeywordTok{c}\NormalTok{(}\StringTok{'WW'}\NormalTok{,}\StringTok{'V1'}\NormalTok{,}\StringTok{'V0'}\NormalTok{,}\StringTok{'p'}\NormalTok{)}
  \KeywordTok{return}\NormalTok{(simuls.ww)}
\NormalTok{\}}

\NormalTok{sf.simuls.ww.N <-}\StringTok{ }\ControlFlowTok{function}\NormalTok{(N,Nsim,param)\{}
  \KeywordTok{sfInit}\NormalTok{(}\DataTypeTok{parallel=}\OtherTok{TRUE}\NormalTok{,}\DataTypeTok{cpus=}\NormalTok{ncpus)}
\NormalTok{  sim <-}\StringTok{ }\KeywordTok{as.data.frame}\NormalTok{(}\KeywordTok{matrix}\NormalTok{(}\KeywordTok{unlist}\NormalTok{(}\KeywordTok{sfLapply}\NormalTok{(}\DecValTok{1}\OperatorTok{:}\NormalTok{Nsim,monte.carlo.ww,}\DataTypeTok{N=}\NormalTok{N,}\DataTypeTok{param=}\NormalTok{param)),}\DataTypeTok{nrow=}\NormalTok{Nsim,}\DataTypeTok{ncol=}\DecValTok{4}\NormalTok{,}\DataTypeTok{byrow=}\OtherTok{TRUE}\NormalTok{))}
  \KeywordTok{sfStop}\NormalTok{()}
  \KeywordTok{colnames}\NormalTok{(sim) <-}\StringTok{ }\KeywordTok{c}\NormalTok{(}\StringTok{'WW'}\NormalTok{,}\StringTok{'V1'}\NormalTok{,}\StringTok{'V0'}\NormalTok{,}\StringTok{'p'}\NormalTok{)}
  \KeywordTok{return}\NormalTok{(sim)}
\NormalTok{\}}

\NormalTok{simuls.ww <-}\StringTok{ }\KeywordTok{lapply}\NormalTok{(N.sample,sf.simuls.ww.N,}\DataTypeTok{Nsim=}\NormalTok{Nsim,}\DataTypeTok{param=}\NormalTok{param)}

\KeywordTok{par}\NormalTok{(}\DataTypeTok{mfrow=}\KeywordTok{c}\NormalTok{(}\DecValTok{2}\NormalTok{,}\DecValTok{2}\NormalTok{))}
\ControlFlowTok{for}\NormalTok{ (i }\ControlFlowTok{in} \DecValTok{1}\OperatorTok{:}\DecValTok{4}\NormalTok{)\{}
  \KeywordTok{hist}\NormalTok{(simuls.ww[[i]][,}\StringTok{'WW'}\NormalTok{],}\DataTypeTok{main=}\KeywordTok{paste}\NormalTok{(}\StringTok{'N='}\NormalTok{,}\KeywordTok{as.character}\NormalTok{(N.sample[i])),}\DataTypeTok{xlab=}\KeywordTok{expression}\NormalTok{(}\KeywordTok{hat}\NormalTok{(Delta}\OperatorTok{^}\NormalTok{yWW)),}\DataTypeTok{xlim=}\KeywordTok{c}\NormalTok{(}\OperatorTok{-}\FloatTok{0.15}\NormalTok{,}\FloatTok{0.55}\NormalTok{))}
  \KeywordTok{abline}\NormalTok{(}\DataTypeTok{v=}\KeywordTok{delta.y.ate}\NormalTok{(param),}\DataTypeTok{col=}\StringTok{"red"}\NormalTok{)}
\NormalTok{\}}
\end{Highlighting}
\end{Shaded}

\begin{figure}[htbp]

{\centering \includegraphics[width=0.6\linewidth]{STCI_files/figure-latex/montecarlo-1} 

}

\caption{Distribution of the $WW$ estimator over replications of samples of different sizes}\label{fig:montecarlo}
\end{figure}

Figure \ref{fig:montecarlo} is essential to understanding statistical
inference and the properties of our estimators. We can see on Figure
\ref{fig:montecarlo} that the estimates indeed move around at each
sample replication. We can also see that the estimates seem to be
concentrated around the truth. We also see that the estimates are more
and more concentrated around the truth as sample size grows larger and
larger.

How big is sampling noise in all of these examples? We can compute it by
using the replications as approximations to the true distribution of the
estimator after an infinite number of samples has been drawn. Let's
first choose a confidence level and then compute the empirical
equivalent to the formula in Definition \ref{def:sampnoise}.

\begin{Shaded}
\begin{Highlighting}[]
\NormalTok{delta<-}\StringTok{ }\FloatTok{0.99}
\NormalTok{delta.}\DecValTok{2}\NormalTok{ <-}\StringTok{ }\FloatTok{0.95}
\NormalTok{samp.noise <-}\StringTok{ }\ControlFlowTok{function}\NormalTok{(estim,delta)\{}
  \KeywordTok{return}\NormalTok{(}\DecValTok{2}\OperatorTok{*}\KeywordTok{quantile}\NormalTok{(}\KeywordTok{abs}\NormalTok{(}\KeywordTok{delta.y.ate}\NormalTok{(param)}\OperatorTok{-}\NormalTok{estim),}\DataTypeTok{prob=}\NormalTok{delta))}
\NormalTok{\}}
\NormalTok{samp.noise.ww <-}\StringTok{ }\KeywordTok{sapply}\NormalTok{(}\KeywordTok{lapply}\NormalTok{(simuls.ww,}\StringTok{`}\DataTypeTok{[}\StringTok{`}\NormalTok{,,}\DecValTok{1}\NormalTok{),samp.noise,}\DataTypeTok{delta=}\NormalTok{delta)}
\KeywordTok{names}\NormalTok{(samp.noise.ww) <-}\StringTok{ }\NormalTok{N.sample}
\NormalTok{samp.noise.ww}
\end{Highlighting}
\end{Shaded}

\begin{verbatim}
##        100       1000      10000      1e+05 
## 1.09916429 0.39083801 0.11582492 0.03527744
\end{verbatim}

Let's also compute precision and the signal to noise ratio and put all
of these results together in a nice table.

\begin{Shaded}
\begin{Highlighting}[]
\NormalTok{precision <-}\StringTok{ }\ControlFlowTok{function}\NormalTok{(estim,delta)\{}
  \KeywordTok{return}\NormalTok{(}\DecValTok{1}\OperatorTok{/}\KeywordTok{samp.noise}\NormalTok{(estim,delta))}
\NormalTok{\}}
\NormalTok{signal.to.noise <-}\StringTok{ }\ControlFlowTok{function}\NormalTok{(estim,delta,param)\{}
  \KeywordTok{return}\NormalTok{(}\KeywordTok{delta.y.ate}\NormalTok{(param)}\OperatorTok{/}\KeywordTok{samp.noise}\NormalTok{(estim,delta))}
\NormalTok{\}}
\NormalTok{precision.ww <-}\StringTok{ }\KeywordTok{sapply}\NormalTok{(}\KeywordTok{lapply}\NormalTok{(simuls.ww,}\StringTok{`}\DataTypeTok{[}\StringTok{`}\NormalTok{,,}\DecValTok{1}\NormalTok{),precision,}\DataTypeTok{delta=}\NormalTok{delta)}
\KeywordTok{names}\NormalTok{(precision.ww) <-}\StringTok{ }\NormalTok{N.sample}
\NormalTok{signal.to.noise.ww <-}\StringTok{ }\KeywordTok{sapply}\NormalTok{(}\KeywordTok{lapply}\NormalTok{(simuls.ww,}\StringTok{`}\DataTypeTok{[}\StringTok{`}\NormalTok{,,}\DecValTok{1}\NormalTok{),signal.to.noise,}\DataTypeTok{delta=}\NormalTok{delta,}\DataTypeTok{param=}\NormalTok{param)}
\KeywordTok{names}\NormalTok{(signal.to.noise.ww) <-}\StringTok{ }\NormalTok{N.sample}
\NormalTok{table.noise <-}\StringTok{ }\KeywordTok{cbind}\NormalTok{(samp.noise.ww,precision.ww,signal.to.noise.ww)}
\KeywordTok{colnames}\NormalTok{(table.noise) <-}\StringTok{ }\KeywordTok{c}\NormalTok{(}\StringTok{'Sampling noise'}\NormalTok{, }\StringTok{'Precision'}\NormalTok{, }\StringTok{'Signal to noise ratio'}\NormalTok{)}
\NormalTok{knitr}\OperatorTok{::}\KeywordTok{kable}\NormalTok{(table.noise,}\DataTypeTok{caption=}\KeywordTok{paste}\NormalTok{(}\StringTok{'Sampling noise of $}\CharTok{\textbackslash{}\textbackslash{}}\StringTok{hat\{WW\}$ for the population treatment effect with $}\CharTok{\textbackslash{}\textbackslash{}}\StringTok{delta=$'}\NormalTok{,delta,}\StringTok{'for various sample sizes'}\NormalTok{,}\DataTypeTok{sep=}\StringTok{' '}\NormalTok{),}\DataTypeTok{booktabs=}\OtherTok{TRUE}\NormalTok{,}\DataTypeTok{digits =} \KeywordTok{c}\NormalTok{(}\DecValTok{2}\NormalTok{,}\DecValTok{2}\NormalTok{,}\DecValTok{2}\NormalTok{),}\DataTypeTok{align=}\KeywordTok{c}\NormalTok{(}\StringTok{'c'}\NormalTok{,}\StringTok{'c'}\NormalTok{,}\StringTok{'c'}\NormalTok{))}
\end{Highlighting}
\end{Shaded}

\begin{table}[t]

\caption{\label{tab:precisionsignal}Sampling noise of $\hat{WW}$ for the population treatment effect with $\delta=$ 0.99 for various sample sizes}
\centering
\begin{tabular}{lccc}
\toprule
  & Sampling noise & Precision & Signal to noise ratio\\
\midrule
100 & 1.10 & 0.91 & 0.16\\
1000 & 0.39 & 2.56 & 0.46\\
10000 & 0.12 & 8.63 & 1.55\\
1e+05 & 0.04 & 28.35 & 5.10\\
\bottomrule
\end{tabular}
\end{table}

Finally, a nice way to summarize the extent of sampling noise is to
graph how sampling noise varies around the true treatment effect, as
shown on Figure \ref{fig:precision}.

\begin{Shaded}
\begin{Highlighting}[]
\KeywordTok{colnames}\NormalTok{(table.noise) <-}\StringTok{ }\KeywordTok{c}\NormalTok{(}\StringTok{'sampling.noise'}\NormalTok{, }\StringTok{'precision'}\NormalTok{, }\StringTok{'signal.to.noise'}\NormalTok{)}
\NormalTok{table.noise <-}\StringTok{ }\KeywordTok{as.data.frame}\NormalTok{(table.noise)}
\NormalTok{table.noise}\OperatorTok{$}\NormalTok{N <-}\StringTok{ }\KeywordTok{as.numeric}\NormalTok{(}\KeywordTok{rownames}\NormalTok{(table.noise))}
\NormalTok{table.noise}\OperatorTok{$}\NormalTok{TT <-}\StringTok{ }\KeywordTok{rep}\NormalTok{(}\KeywordTok{delta.y.ate}\NormalTok{(param),}\KeywordTok{nrow}\NormalTok{(table.noise))}
\KeywordTok{ggplot}\NormalTok{(table.noise, }\KeywordTok{aes}\NormalTok{(}\DataTypeTok{x=}\KeywordTok{as.factor}\NormalTok{(N), }\DataTypeTok{y=}\NormalTok{TT)) }\OperatorTok{+}
\StringTok{  }\KeywordTok{geom_bar}\NormalTok{(}\DataTypeTok{position=}\KeywordTok{position_dodge}\NormalTok{(), }\DataTypeTok{stat=}\StringTok{"identity"}\NormalTok{, }\DataTypeTok{colour=}\StringTok{'black'}\NormalTok{) }\OperatorTok{+}
\StringTok{  }\KeywordTok{geom_errorbar}\NormalTok{(}\KeywordTok{aes}\NormalTok{(}\DataTypeTok{ymin=}\NormalTok{TT}\OperatorTok{-}\NormalTok{sampling.noise}\OperatorTok{/}\DecValTok{2}\NormalTok{, }\DataTypeTok{ymax=}\NormalTok{TT}\OperatorTok{+}\NormalTok{sampling.noise}\OperatorTok{/}\DecValTok{2}\NormalTok{), }\DataTypeTok{width=}\NormalTok{.}\DecValTok{2}\NormalTok{,}\DataTypeTok{position=}\KeywordTok{position_dodge}\NormalTok{(.}\DecValTok{9}\NormalTok{),}\DataTypeTok{color=}\StringTok{'red'}\NormalTok{) }\OperatorTok{+}
\StringTok{  }\KeywordTok{xlab}\NormalTok{(}\StringTok{"Sample Size"}\NormalTok{)}\OperatorTok{+}
\StringTok{  }\KeywordTok{theme_bw}\NormalTok{()}
\end{Highlighting}
\end{Shaded}

\begin{figure}[htbp]

{\centering \includegraphics[width=0.6\linewidth]{STCI_files/figure-latex/precision-1} 

}

\caption{Sampling noise of $\hat{WW}$ (99\% confidence) around $TT$ for various sample sizes}\label{fig:precision}
\end{figure}

With \(N=\) 100, we can definitely see on Figure \ref{fig:precision}
that sampling noise is ridiculously large, especially compared with the
treatment effect that we are trying to estimate. The signal to noise
ratio is 0.16, which means that sampling noise is an order of magnitude
bigger than the signal we are trying to extract. As a consequence, in
22.2\% of our samples, we are going to estimate a negative effect of the
treatment. There is also a 20.4\% chance that we end up estimating an
effect that is double the true effect. So how much can we trust our
estimate from one sample to be close to the true effect of the treatment
when \(N=\) 100? Not much.

With \(N=\) 1000, sampling noise is still large: the signal to noise
ratio is 0.46, which means that sampling noise is double the signal we
are trying to extract. As a consequence, the chance that we end up with
a negative treatment effect has decreased to 0.9\% and that we end up
with an effect double the true one is 1\%. But still, the chances that
we end up with an effect that is smaller than three quarters of the true
effect is 25.6\% and the chances that we end up with an estimator that
is 25\% bigger than the true effect is 26.2\%. These are nontrivial
differences: compare a program that increases earnings by 13.5\% to one
that increases them by 18\% and another by 22.5\%. They would have
completely different cost/benefit ratios. But we at least trust our
estimator to give us a correct idea of the sign of the treatment effect
and a vague and imprecise idea of its magnitude.

With \(N=\) 10\^{}\{4\}, sampling noise is smaller than the signal,
which is encouraging. The signal to noise ratio is 1.55. In only 1\% of
the samples does the estimated effect of the treatment become smaller
than 0.125 or bigger than 0.247. We start gaining a lot of confidence in
the relative magnitude of the effect, even if sampling noise is still
responsible for economically significant variation.

With \(N=\) 10\^{}\{5\}, sampling noise has become trivial. The signal
to noise ratio is 5.1, which means that the signal is now 5 times bigger
than the sampling noise. In only 1\% of the samples does the estimated
effect of the treatment become smaller than 0.163 or bigger than 0.198.
Sampling noise is not any more responsible for economically meaningful
variation.

\subsection{Sampling noise for the sample treatment
effect}\label{sec:illusnoisesamp}

Sampling noise for the sample parameter stems from the fact that the
treated and control groups are not perfectly identical. The distribution
of observed and unobserved covariates is actually different, because of
sampling variation. This makes the actual comparison of means in the
sample a noisy estimate of the true comparison that we would obtain by
comparing the potential outcomes of the treated directly.

In order to understand this issue well and to be able to illustrate it
correctly, I am going to focus on the average treatment effect in the
whole sample, not on the treated:
\(\Delta^Y_{ATE_s}=\frac{1}{N}\sum_{i=1}^N(Y_i^1-Y_i^0)\). This enables
me to define a sample parameter that is independent of the allocation of
\(D_i\). This is without important consequences since these two
parameters are equal in the population when there is no selection bias,
as we are assuming since the beginning of this lecture. Furthermore, if
we view the treatment allocation generating no selection bias as a true
random assignment in a Randomized Controlled Trial (RCT), then it is
still possible to use this approach to estimate \(TT\) if we view the
population over which we randomise as the population selected for
receiving the treatment, as we will see in the lecture on RCTs.

\BeginKnitrBlock{example}
\protect\hypertarget{exm:unnamed-chunk-34}{}{\label{exm:unnamed-chunk-34}
}In order to assess the scope of sampling noise for our sample treatment
effect estimate, we first have to draw a sample:
\EndKnitrBlock{example}

\begin{Shaded}
\begin{Highlighting}[]
\KeywordTok{set.seed}\NormalTok{(}\DecValTok{1234}\NormalTok{)}
\NormalTok{N <-}\DecValTok{1000}
\NormalTok{mu <-}\StringTok{ }\KeywordTok{rnorm}\NormalTok{(N,param[}\StringTok{"barmu"}\NormalTok{],}\KeywordTok{sqrt}\NormalTok{(param[}\StringTok{"sigma2mu"}\NormalTok{]))}
\NormalTok{UB <-}\StringTok{ }\KeywordTok{rnorm}\NormalTok{(N,}\DecValTok{0}\NormalTok{,}\KeywordTok{sqrt}\NormalTok{(param[}\StringTok{"sigma2U"}\NormalTok{]))}
\NormalTok{yB <-}\StringTok{ }\NormalTok{mu }\OperatorTok{+}\StringTok{ }\NormalTok{UB }
\NormalTok{YB <-}\StringTok{ }\KeywordTok{exp}\NormalTok{(yB)}
\NormalTok{Ds <-}\StringTok{ }\KeywordTok{rep}\NormalTok{(}\DecValTok{0}\NormalTok{,N)}
\NormalTok{V <-}\StringTok{ }\KeywordTok{rnorm}\NormalTok{(N,param[}\StringTok{"barmu"}\NormalTok{],}\KeywordTok{sqrt}\NormalTok{(param[}\StringTok{"sigma2mu"}\NormalTok{]}\OperatorTok{+}\NormalTok{param[}\StringTok{"sigma2U"}\NormalTok{]))}
\NormalTok{Ds[V}\OperatorTok{<=}\KeywordTok{log}\NormalTok{(param[}\StringTok{"barY"}\NormalTok{])] <-}\StringTok{ }\DecValTok{1} 
\NormalTok{epsilon <-}\StringTok{ }\KeywordTok{rnorm}\NormalTok{(N,}\DecValTok{0}\NormalTok{,}\KeywordTok{sqrt}\NormalTok{(param[}\StringTok{"sigma2epsilon"}\NormalTok{]))}
\NormalTok{eta<-}\StringTok{ }\KeywordTok{rnorm}\NormalTok{(N,}\DecValTok{0}\NormalTok{,}\KeywordTok{sqrt}\NormalTok{(param[}\StringTok{"sigma2eta"}\NormalTok{]))}
\NormalTok{U0 <-}\StringTok{ }\NormalTok{param[}\StringTok{"rho"}\NormalTok{]}\OperatorTok{*}\NormalTok{UB }\OperatorTok{+}\StringTok{ }\NormalTok{epsilon}
\NormalTok{y0 <-}\StringTok{ }\NormalTok{mu }\OperatorTok{+}\StringTok{  }\NormalTok{U0 }\OperatorTok{+}\StringTok{ }\NormalTok{param[}\StringTok{"delta"}\NormalTok{]}
\NormalTok{alpha <-}\StringTok{ }\NormalTok{param[}\StringTok{"baralpha"}\NormalTok{]}\OperatorTok{+}\StringTok{  }\NormalTok{param[}\StringTok{"theta"}\NormalTok{]}\OperatorTok{*}\NormalTok{mu }\OperatorTok{+}\StringTok{ }\NormalTok{eta}
\NormalTok{y1 <-}\StringTok{ }\NormalTok{y0}\OperatorTok{+}\NormalTok{alpha}
\NormalTok{Y0 <-}\StringTok{ }\KeywordTok{exp}\NormalTok{(y0)}
\NormalTok{Y1 <-}\StringTok{ }\KeywordTok{exp}\NormalTok{(y1)}
\NormalTok{y <-}\StringTok{ }\NormalTok{y1}\OperatorTok{*}\NormalTok{Ds}\OperatorTok{+}\NormalTok{y0}\OperatorTok{*}\NormalTok{(}\DecValTok{1}\OperatorTok{-}\NormalTok{Ds)}
\NormalTok{Y <-}\StringTok{ }\NormalTok{Y1}\OperatorTok{*}\NormalTok{Ds}\OperatorTok{+}\NormalTok{Y0}\OperatorTok{*}\NormalTok{(}\DecValTok{1}\OperatorTok{-}\NormalTok{Ds)}
\end{Highlighting}
\end{Shaded}

In this sample, the treatment effect parameter is \(\Delta^y_{ATE_s}=\)
0.171. The \(WW\) estimator yields an estimate of
\(\hat{\Delta^y_{WW}}=\) 0.133. Despite random assignment, we have
\(\Delta^y_{ATE_s}\neq\hat{\Delta^y_{WW}}\), an instance of the FPSI.

In order to see how sampling noise varies, let's draw a new treatment
allocation, while retaining the same sample and the same potential
outcomes.

\begin{Shaded}
\begin{Highlighting}[]
\KeywordTok{set.seed}\NormalTok{(}\DecValTok{12345}\NormalTok{)}
\NormalTok{N <-}\DecValTok{1000}
\NormalTok{Ds <-}\StringTok{ }\KeywordTok{rep}\NormalTok{(}\DecValTok{0}\NormalTok{,N)}
\NormalTok{V <-}\StringTok{ }\KeywordTok{rnorm}\NormalTok{(N,param[}\StringTok{"barmu"}\NormalTok{],}\KeywordTok{sqrt}\NormalTok{(param[}\StringTok{"sigma2mu"}\NormalTok{]}\OperatorTok{+}\NormalTok{param[}\StringTok{"sigma2U"}\NormalTok{]))}
\NormalTok{Ds[V}\OperatorTok{<=}\KeywordTok{log}\NormalTok{(param[}\StringTok{"barY"}\NormalTok{])] <-}\StringTok{ }\DecValTok{1} 
\NormalTok{y <-}\StringTok{ }\NormalTok{y1}\OperatorTok{*}\NormalTok{Ds}\OperatorTok{+}\NormalTok{y0}\OperatorTok{*}\NormalTok{(}\DecValTok{1}\OperatorTok{-}\NormalTok{Ds)}
\NormalTok{Y <-}\StringTok{ }\NormalTok{Y1}\OperatorTok{*}\NormalTok{Ds}\OperatorTok{+}\NormalTok{Y0}\OperatorTok{*}\NormalTok{(}\DecValTok{1}\OperatorTok{-}\NormalTok{Ds)}
\end{Highlighting}
\end{Shaded}

In this sample, the treatment effect parameter is still
\(\Delta^y_{ATE_s}=\) 0.171. The \(WW\) estimator yields now an estimate
of \(\hat{\Delta^y_{WW}}=\) 0.051. The \(WW\) estimate is different from
our previous estimate because the treatment was allocated to a different
random subset of people.

Why is this second estimate so imprecise? It might because it estimates
one of the two components of the average treatment effect badly, or
both. The true average potential outcome with the treatment is, in this
sample, \(\frac{1}{N}\sum_{i=1}^Ny_i^1=\) 8.207 while the \(WW\)
estimate of this quantity is
\(\frac{1}{\sum_{i=1}^ND_i}\sum_{i=1}^ND_iy_i=\) 8.113. The true average
potential outcome without the treatment is, in this sample,
\(\frac{1}{N}\sum_{i=1}^Ny_i^0=\) 8.036 while the \(WW\) estimate of
this quantity is
\(\frac{1}{\sum_{i=1}^N(1-D_i)}\sum_{i=1}^N(1-D_i)y_i=\) 8.062. It thus
seems that most of the bias in the estimated effect stems from the fact
that the treatment has been allocated to individuals with lower than
expected outcomes with the treatment, be it because they did not react
strongly to the treatment, or because they were in worse shape without
the treatment. We can check which one of these two explanations is more
important. The true average effect of the treatment is, in this sample,
\(\frac{1}{N}\sum_{i=1}^N(y_i^1-y^0_i)=\) 0.171 while, in the treated
group, this quantity is
\(\frac{1}{\sum_{i=1}^ND_i}\sum_{i=1}^ND_i(y_i^1-y_i^0)=\) 0.18. The
true average potential outcome without the treatment is, in this sample,
\(\frac{1}{N}\sum_{i=1}^Ny^0_i=\) 8.036 while, in the treated group,
this quantity is \(\frac{1}{\sum_{i=1}^ND_i}\sum_{i=1}^ND_iy_i^0=\)
7.933. The reason for the poor performance of the \(WW\) estimator in
this sample is that individuals with lower counterfactual outcomes were
included in the treated group, not that the treatment had lower effects
on them. The bad counterfactual outcomes of the treated generates a bias
of -0.103, while the bias due to heterogeneous reactions to the
treatment is of 0.009. The last part of the bias is the one due to the
fact that the individuals in the control group have slightly better
counterfactual outcomes than in the sample: -0.026. The sum of these
three terms yields the total bias of our \(WW\) estimator in this second
sample: -0.12.

Let's now assess the overall effect of sampling noise on the estimate of
the sample treatment effect for various sample sizes. In order to do
this, I am going to use parallelized Monte Carlo simulations again. For
the sake of simplicity, I am going to generate the same potential
outcomes in each replication, using the same seed, and only choose a
different treatment allocation.

\begin{Shaded}
\begin{Highlighting}[]
\NormalTok{monte.carlo.ww.sample <-}\StringTok{ }\ControlFlowTok{function}\NormalTok{(s,N,param)\{}
  \KeywordTok{set.seed}\NormalTok{(}\DecValTok{1234}\NormalTok{)}
\NormalTok{  mu <-}\StringTok{ }\KeywordTok{rnorm}\NormalTok{(N,param[}\StringTok{"barmu"}\NormalTok{],}\KeywordTok{sqrt}\NormalTok{(param[}\StringTok{"sigma2mu"}\NormalTok{]))}
\NormalTok{  UB <-}\StringTok{ }\KeywordTok{rnorm}\NormalTok{(N,}\DecValTok{0}\NormalTok{,}\KeywordTok{sqrt}\NormalTok{(param[}\StringTok{"sigma2U"}\NormalTok{]))}
\NormalTok{  yB <-}\StringTok{ }\NormalTok{mu }\OperatorTok{+}\StringTok{ }\NormalTok{UB }
\NormalTok{  YB <-}\StringTok{ }\KeywordTok{exp}\NormalTok{(yB)}
\NormalTok{  epsilon <-}\StringTok{ }\KeywordTok{rnorm}\NormalTok{(N,}\DecValTok{0}\NormalTok{,}\KeywordTok{sqrt}\NormalTok{(param[}\StringTok{"sigma2epsilon"}\NormalTok{]))}
\NormalTok{  eta<-}\StringTok{ }\KeywordTok{rnorm}\NormalTok{(N,}\DecValTok{0}\NormalTok{,}\KeywordTok{sqrt}\NormalTok{(param[}\StringTok{"sigma2eta"}\NormalTok{]))}
\NormalTok{  U0 <-}\StringTok{ }\NormalTok{param[}\StringTok{"rho"}\NormalTok{]}\OperatorTok{*}\NormalTok{UB }\OperatorTok{+}\StringTok{ }\NormalTok{epsilon}
\NormalTok{  y0 <-}\StringTok{ }\NormalTok{mu }\OperatorTok{+}\StringTok{  }\NormalTok{U0 }\OperatorTok{+}\StringTok{ }\NormalTok{param[}\StringTok{"delta"}\NormalTok{]}
\NormalTok{  alpha <-}\StringTok{ }\NormalTok{param[}\StringTok{"baralpha"}\NormalTok{]}\OperatorTok{+}\StringTok{  }\NormalTok{param[}\StringTok{"theta"}\NormalTok{]}\OperatorTok{*}\NormalTok{mu }\OperatorTok{+}\StringTok{ }\NormalTok{eta}
\NormalTok{  y1 <-}\StringTok{ }\NormalTok{y0}\OperatorTok{+}\NormalTok{alpha}
\NormalTok{  Y0 <-}\StringTok{ }\KeywordTok{exp}\NormalTok{(y0)}
\NormalTok{  Y1 <-}\StringTok{ }\KeywordTok{exp}\NormalTok{(y1)}
  \KeywordTok{set.seed}\NormalTok{(s)}
\NormalTok{  Ds <-}\StringTok{ }\KeywordTok{rep}\NormalTok{(}\DecValTok{0}\NormalTok{,N)}
\NormalTok{  V <-}\StringTok{ }\KeywordTok{rnorm}\NormalTok{(N,param[}\StringTok{"barmu"}\NormalTok{],}\KeywordTok{sqrt}\NormalTok{(param[}\StringTok{"sigma2mu"}\NormalTok{]}\OperatorTok{+}\NormalTok{param[}\StringTok{"sigma2U"}\NormalTok{]))}
\NormalTok{  Ds[V}\OperatorTok{<=}\KeywordTok{log}\NormalTok{(param[}\StringTok{"barY"}\NormalTok{])] <-}\StringTok{ }\DecValTok{1} 
\NormalTok{  y <-}\StringTok{ }\NormalTok{y1}\OperatorTok{*}\NormalTok{Ds}\OperatorTok{+}\NormalTok{y0}\OperatorTok{*}\NormalTok{(}\DecValTok{1}\OperatorTok{-}\NormalTok{Ds)}
\NormalTok{  Y <-}\StringTok{ }\NormalTok{Y1}\OperatorTok{*}\NormalTok{Ds}\OperatorTok{+}\NormalTok{Y0}\OperatorTok{*}\NormalTok{(}\DecValTok{1}\OperatorTok{-}\NormalTok{Ds)}
  \KeywordTok{return}\NormalTok{((}\DecValTok{1}\OperatorTok{/}\KeywordTok{sum}\NormalTok{(Ds))}\OperatorTok{*}\KeywordTok{sum}\NormalTok{(y}\OperatorTok{*}\NormalTok{Ds)}\OperatorTok{-}\NormalTok{(}\DecValTok{1}\OperatorTok{/}\KeywordTok{sum}\NormalTok{(}\DecValTok{1}\OperatorTok{-}\NormalTok{Ds))}\OperatorTok{*}\KeywordTok{sum}\NormalTok{(y}\OperatorTok{*}\NormalTok{(}\DecValTok{1}\OperatorTok{-}\NormalTok{Ds)))}
\NormalTok{\}}

\NormalTok{simuls.ww.N.sample <-}\StringTok{ }\ControlFlowTok{function}\NormalTok{(N,Nsim,param)\{}
  \KeywordTok{return}\NormalTok{(}\KeywordTok{unlist}\NormalTok{(}\KeywordTok{lapply}\NormalTok{(}\DecValTok{1}\OperatorTok{:}\NormalTok{Nsim,monte.carlo.ww.sample,}\DataTypeTok{N=}\NormalTok{N,}\DataTypeTok{param=}\NormalTok{param)))}
\NormalTok{\}}

\NormalTok{sf.simuls.ww.N.sample <-}\StringTok{ }\ControlFlowTok{function}\NormalTok{(N,Nsim,param)\{}
  \KeywordTok{sfInit}\NormalTok{(}\DataTypeTok{parallel=}\OtherTok{TRUE}\NormalTok{,}\DataTypeTok{cpus=}\NormalTok{ncpus)}
\NormalTok{  sim <-}\StringTok{ }\KeywordTok{sfLapply}\NormalTok{(}\DecValTok{1}\OperatorTok{:}\NormalTok{Nsim,monte.carlo.ww.sample,}\DataTypeTok{N=}\NormalTok{N,}\DataTypeTok{param=}\NormalTok{param)}
  \KeywordTok{sfStop}\NormalTok{()}
  \KeywordTok{return}\NormalTok{(}\KeywordTok{unlist}\NormalTok{(sim))}
\NormalTok{\}}

\NormalTok{simuls.ww.sample <-}\StringTok{ }\KeywordTok{lapply}\NormalTok{(N.sample,sf.simuls.ww.N.sample,}\DataTypeTok{Nsim=}\NormalTok{Nsim,}\DataTypeTok{param=}\NormalTok{param)}

\NormalTok{monte.carlo.ate.sample <-}\StringTok{ }\ControlFlowTok{function}\NormalTok{(N,s,param)\{}
  \KeywordTok{set.seed}\NormalTok{(s)}
\NormalTok{  mu <-}\StringTok{ }\KeywordTok{rnorm}\NormalTok{(N,param[}\StringTok{"barmu"}\NormalTok{],}\KeywordTok{sqrt}\NormalTok{(param[}\StringTok{"sigma2mu"}\NormalTok{]))}
\NormalTok{  UB <-}\StringTok{ }\KeywordTok{rnorm}\NormalTok{(N,}\DecValTok{0}\NormalTok{,}\KeywordTok{sqrt}\NormalTok{(param[}\StringTok{"sigma2U"}\NormalTok{]))}
\NormalTok{  yB <-}\StringTok{ }\NormalTok{mu }\OperatorTok{+}\StringTok{ }\NormalTok{UB }
\NormalTok{  YB <-}\StringTok{ }\KeywordTok{exp}\NormalTok{(yB)}
\NormalTok{  epsilon <-}\StringTok{ }\KeywordTok{rnorm}\NormalTok{(N,}\DecValTok{0}\NormalTok{,}\KeywordTok{sqrt}\NormalTok{(param[}\StringTok{"sigma2epsilon"}\NormalTok{]))}
\NormalTok{  eta<-}\StringTok{ }\KeywordTok{rnorm}\NormalTok{(N,}\DecValTok{0}\NormalTok{,}\KeywordTok{sqrt}\NormalTok{(param[}\StringTok{"sigma2eta"}\NormalTok{]))}
\NormalTok{  U0 <-}\StringTok{ }\NormalTok{param[}\StringTok{"rho"}\NormalTok{]}\OperatorTok{*}\NormalTok{UB }\OperatorTok{+}\StringTok{ }\NormalTok{epsilon}
\NormalTok{  y0 <-}\StringTok{ }\NormalTok{mu }\OperatorTok{+}\StringTok{  }\NormalTok{U0 }\OperatorTok{+}\StringTok{ }\NormalTok{param[}\StringTok{"delta"}\NormalTok{]}
\NormalTok{  alpha <-}\StringTok{ }\NormalTok{param[}\StringTok{"baralpha"}\NormalTok{]}\OperatorTok{+}\StringTok{  }\NormalTok{param[}\StringTok{"theta"}\NormalTok{]}\OperatorTok{*}\NormalTok{mu }\OperatorTok{+}\StringTok{ }\NormalTok{eta}
\NormalTok{  y1 <-}\StringTok{ }\NormalTok{y0}\OperatorTok{+}\NormalTok{alpha}
\NormalTok{  Y0 <-}\StringTok{ }\KeywordTok{exp}\NormalTok{(y0)}
\NormalTok{  Y1 <-}\StringTok{ }\KeywordTok{exp}\NormalTok{(y1)}
\NormalTok{  Ds <-}\StringTok{ }\KeywordTok{rep}\NormalTok{(}\DecValTok{0}\NormalTok{,N)}
\NormalTok{  V <-}\StringTok{ }\KeywordTok{rnorm}\NormalTok{(N,param[}\StringTok{"barmu"}\NormalTok{],}\KeywordTok{sqrt}\NormalTok{(param[}\StringTok{"sigma2mu"}\NormalTok{]}\OperatorTok{+}\NormalTok{param[}\StringTok{"sigma2U"}\NormalTok{]))}
\NormalTok{  Ds[V}\OperatorTok{<=}\KeywordTok{log}\NormalTok{(param[}\StringTok{"barY"}\NormalTok{])] <-}\StringTok{ }\DecValTok{1} 
\NormalTok{  y <-}\StringTok{ }\NormalTok{y1}\OperatorTok{*}\NormalTok{Ds}\OperatorTok{+}\NormalTok{y0}\OperatorTok{*}\NormalTok{(}\DecValTok{1}\OperatorTok{-}\NormalTok{Ds)}
\NormalTok{  Y <-}\StringTok{ }\NormalTok{Y1}\OperatorTok{*}\NormalTok{Ds}\OperatorTok{+}\NormalTok{Y0}\OperatorTok{*}\NormalTok{(}\DecValTok{1}\OperatorTok{-}\NormalTok{Ds)}
  \KeywordTok{return}\NormalTok{(}\KeywordTok{mean}\NormalTok{(alpha))}
\NormalTok{\}}

\KeywordTok{par}\NormalTok{(}\DataTypeTok{mfrow=}\KeywordTok{c}\NormalTok{(}\DecValTok{2}\NormalTok{,}\DecValTok{2}\NormalTok{))}
\ControlFlowTok{for}\NormalTok{ (i }\ControlFlowTok{in} \DecValTok{1}\OperatorTok{:}\DecValTok{4}\NormalTok{)\{}
  \KeywordTok{hist}\NormalTok{(simuls.ww.sample[[i]],}\DataTypeTok{main=}\KeywordTok{paste}\NormalTok{(}\StringTok{'N='}\NormalTok{,}\KeywordTok{as.character}\NormalTok{(N.sample[i])),}\DataTypeTok{xlab=}\KeywordTok{expression}\NormalTok{(}\KeywordTok{hat}\NormalTok{(Delta}\OperatorTok{^}\NormalTok{yWW)),}\DataTypeTok{xlim=}\KeywordTok{c}\NormalTok{(}\OperatorTok{-}\FloatTok{0.15}\NormalTok{,}\FloatTok{0.55}\NormalTok{))}
  \KeywordTok{abline}\NormalTok{(}\DataTypeTok{v=}\KeywordTok{monte.carlo.ate.sample}\NormalTok{(N.sample[[i]],}\DecValTok{1234}\NormalTok{,param),}\DataTypeTok{col=}\StringTok{"red"}\NormalTok{)}
\NormalTok{\}}
\end{Highlighting}
\end{Shaded}

\begin{figure}[htbp]

{\centering \includegraphics[width=0.6\linewidth]{STCI_files/figure-latex/montecarlosample-1} 

}

\caption{Distribution of the $WW$ estimator over replications of treatment allocation for samples of different sizes}\label{fig:montecarlosample}
\end{figure}

Let's also compute sampling noise, precision and the signal to noise
ratio in these examples.

\begin{Shaded}
\begin{Highlighting}[]
\NormalTok{samp.noise.sample <-}\StringTok{ }\ControlFlowTok{function}\NormalTok{(i,delta,param)\{}
  \KeywordTok{return}\NormalTok{(}\DecValTok{2}\OperatorTok{*}\KeywordTok{quantile}\NormalTok{(}\KeywordTok{abs}\NormalTok{(}\KeywordTok{monte.carlo.ate.sample}\NormalTok{(}\DecValTok{1234}\NormalTok{,N.sample[[i]],param)}\OperatorTok{-}\NormalTok{simuls.ww.sample[[i]]),}\DataTypeTok{prob=}\NormalTok{delta))}
\NormalTok{\}}
\NormalTok{samp.noise.ww.sample <-}\StringTok{ }\KeywordTok{sapply}\NormalTok{(}\DecValTok{1}\OperatorTok{:}\DecValTok{4}\NormalTok{,samp.noise.sample,}\DataTypeTok{delta=}\NormalTok{delta,}\DataTypeTok{param=}\NormalTok{param)}
\KeywordTok{names}\NormalTok{(samp.noise.ww.sample) <-}\StringTok{ }\NormalTok{N.sample}

\NormalTok{precision.sample <-}\StringTok{ }\ControlFlowTok{function}\NormalTok{(i,delta,param)\{}
  \KeywordTok{return}\NormalTok{(}\DecValTok{1}\OperatorTok{/}\KeywordTok{samp.noise.sample}\NormalTok{(i,delta,}\DataTypeTok{param=}\NormalTok{param))}
\NormalTok{\}}
\NormalTok{signal.to.noise.sample <-}\StringTok{ }\ControlFlowTok{function}\NormalTok{(i,delta,param)\{}
  \KeywordTok{return}\NormalTok{(}\KeywordTok{monte.carlo.ate.sample}\NormalTok{(}\DecValTok{1234}\NormalTok{,N.sample[[i]],param)}\OperatorTok{/}\KeywordTok{samp.noise.sample}\NormalTok{(i,delta,}\DataTypeTok{param=}\NormalTok{param))}
\NormalTok{\}}
\NormalTok{precision.ww.sample <-}\StringTok{ }\KeywordTok{sapply}\NormalTok{(}\DecValTok{1}\OperatorTok{:}\DecValTok{4}\NormalTok{,precision.sample,}\DataTypeTok{delta=}\NormalTok{delta,}\DataTypeTok{param=}\NormalTok{param)}
\KeywordTok{names}\NormalTok{(precision.ww.sample) <-}\StringTok{ }\NormalTok{N.sample}
\NormalTok{signal.to.noise.ww.sample <-}\StringTok{ }\KeywordTok{sapply}\NormalTok{(}\DecValTok{1}\OperatorTok{:}\DecValTok{4}\NormalTok{,signal.to.noise.sample,}\DataTypeTok{delta=}\NormalTok{delta,}\DataTypeTok{param=}\NormalTok{param)}
\KeywordTok{names}\NormalTok{(signal.to.noise.ww.sample) <-}\StringTok{ }\NormalTok{N.sample}
\NormalTok{table.noise.sample <-}\StringTok{ }\KeywordTok{cbind}\NormalTok{(samp.noise.ww.sample,precision.ww.sample,signal.to.noise.ww.sample)}
\KeywordTok{colnames}\NormalTok{(table.noise.sample) <-}\StringTok{ }\KeywordTok{c}\NormalTok{(}\StringTok{'Sampling noise'}\NormalTok{, }\StringTok{'Precision'}\NormalTok{, }\StringTok{'Signal to noise ratio'}\NormalTok{)}
\NormalTok{knitr}\OperatorTok{::}\KeywordTok{kable}\NormalTok{(table.noise.sample,}\DataTypeTok{caption=}\KeywordTok{paste}\NormalTok{(}\StringTok{'Sampling noise of $}\CharTok{\textbackslash{}\textbackslash{}}\StringTok{hat\{WW\}$ for the sample treatment effect with $}\CharTok{\textbackslash{}\textbackslash{}}\StringTok{delta=$'}\NormalTok{,delta,}\StringTok{'and for various sample sizes'}\NormalTok{,}\DataTypeTok{sep=}\StringTok{' '}\NormalTok{),}\DataTypeTok{booktabs=}\OtherTok{TRUE}\NormalTok{,}\DataTypeTok{align=}\KeywordTok{c}\NormalTok{(}\StringTok{'c'}\NormalTok{,}\StringTok{'c'}\NormalTok{,}\StringTok{'c'}\NormalTok{),}\DataTypeTok{digits=}\KeywordTok{c}\NormalTok{(}\DecValTok{3}\NormalTok{,}\DecValTok{3}\NormalTok{,}\DecValTok{3}\NormalTok{))}
\end{Highlighting}
\end{Shaded}

\begin{table}[t]

\caption{\label{tab:sampnoisesample}Sampling noise of $\hat{WW}$ for the sample treatment effect with $\delta=$ 0.99 and for various sample sizes}
\centering
\begin{tabular}{lccc}
\toprule
  & Sampling noise & Precision & Signal to noise ratio\\
\midrule
100 & 1.208 & 0.828 & 0.149\\
1000 & 0.366 & 2.729 & 0.482\\
10000 & 0.122 & 8.218 & 1.585\\
1e+05 & 0.033 & 30.283 & 5.453\\
\bottomrule
\end{tabular}
\end{table}

Finally, let's compare the extent of sampling noise for the population
and the sample treatment effect parameters.

\begin{Shaded}
\begin{Highlighting}[]
\KeywordTok{colnames}\NormalTok{(table.noise.sample) <-}\StringTok{ }\KeywordTok{c}\NormalTok{(}\StringTok{'sampling.noise'}\NormalTok{, }\StringTok{'precision'}\NormalTok{, }\StringTok{'signal.to.noise'}\NormalTok{)}
\NormalTok{table.noise.sample <-}\StringTok{ }\KeywordTok{as.data.frame}\NormalTok{(table.noise.sample)}
\NormalTok{table.noise.sample}\OperatorTok{$}\NormalTok{N <-}\StringTok{ }\KeywordTok{as.numeric}\NormalTok{(}\KeywordTok{rownames}\NormalTok{(table.noise.sample))}
\NormalTok{table.noise.sample}\OperatorTok{$}\NormalTok{TT <-}\StringTok{ }\KeywordTok{sapply}\NormalTok{(N.sample,monte.carlo.ate.sample,}\DataTypeTok{s=}\DecValTok{1234}\NormalTok{,}\DataTypeTok{param=}\NormalTok{param)}
\NormalTok{table.noise.sample}\OperatorTok{$}\NormalTok{Type <-}\StringTok{ 'TTs'}
\NormalTok{table.noise}\OperatorTok{$}\NormalTok{Type <-}\StringTok{ 'TT'}
\NormalTok{table.noise.tot <-}\StringTok{ }\KeywordTok{rbind}\NormalTok{(table.noise,table.noise.sample)}
\NormalTok{table.noise.tot}\OperatorTok{$}\NormalTok{Type <-}\StringTok{ }\KeywordTok{factor}\NormalTok{(table.noise.tot}\OperatorTok{$}\NormalTok{Type)}

\KeywordTok{ggplot}\NormalTok{(table.noise.tot, }\KeywordTok{aes}\NormalTok{(}\DataTypeTok{x=}\KeywordTok{as.factor}\NormalTok{(N), }\DataTypeTok{y=}\NormalTok{TT,}\DataTypeTok{fill=}\NormalTok{Type)) }\OperatorTok{+}
\StringTok{  }\KeywordTok{geom_bar}\NormalTok{(}\DataTypeTok{position=}\KeywordTok{position_dodge}\NormalTok{(), }\DataTypeTok{stat=}\StringTok{"identity"}\NormalTok{, }\DataTypeTok{colour=}\StringTok{'black'}\NormalTok{) }\OperatorTok{+}
\StringTok{  }\KeywordTok{geom_errorbar}\NormalTok{(}\KeywordTok{aes}\NormalTok{(}\DataTypeTok{ymin=}\NormalTok{TT}\OperatorTok{-}\NormalTok{sampling.noise}\OperatorTok{/}\DecValTok{2}\NormalTok{, }\DataTypeTok{ymax=}\NormalTok{TT}\OperatorTok{+}\NormalTok{sampling.noise}\OperatorTok{/}\DecValTok{2}\NormalTok{), }\DataTypeTok{width=}\NormalTok{.}\DecValTok{2}\NormalTok{,}\DataTypeTok{position=}\KeywordTok{position_dodge}\NormalTok{(.}\DecValTok{9}\NormalTok{),}\DataTypeTok{color=}\StringTok{'red'}\NormalTok{) }\OperatorTok{+}
\StringTok{  }\KeywordTok{xlab}\NormalTok{(}\StringTok{"Sample Size"}\NormalTok{)}\OperatorTok{+}
\StringTok{  }\KeywordTok{theme_bw}\NormalTok{()}\OperatorTok{+}
\StringTok{  }\KeywordTok{theme}\NormalTok{(}\DataTypeTok{legend.position=}\KeywordTok{c}\NormalTok{(}\FloatTok{0.85}\NormalTok{,}\FloatTok{0.88}\NormalTok{))}
\end{Highlighting}
\end{Shaded}

\begin{figure}[htbp]

{\centering \includegraphics[width=0.6\linewidth]{STCI_files/figure-latex/precisionpopsample-1} 

}

\caption{Sampling noise of $\hat{WW}$ (99\% confidence) around $TT$ and $TT_s$ for various sample sizes}\label{fig:precisionpopsample}
\end{figure}

Figure \ref{fig:montecarlosample} and Table \ref{tab:sampnoisesample}
present the results of the simulations of sampling noise for the sample
treatment effect parameter. Figure \ref{fig:precisionpopsample} compares
sampling noise for the population and sample treatment effects.\\
For all practical purposes, the estimates of sampling noise for the
sample treatment effect are extremely close to the ones we have
estimated for the population treatment effect. I am actually surprised
by this result, since I expected that keeping the potential outcomes
constant over replications would decrease sampling noise. It seems that
the variability in potential outcomes over replications of random
allocations of the treatment in a given sample mimicks very well the
sampling process from a population. I do not know if this result of
similarity of sampling noise for the population and sample treatment
effect is a general one, but considering them as similar or close seems
innocuous in our example.

\subsection{Building confidence intervals from estimates of sampling
noise}\label{sec:confinterv}

In real life, we do not observe \(TT\). We only have access to
\(\hat{E}\). Let's also assume for now that we have access to an
estimate of sampling noise, \(2\epsilon\). How can we use these two
quantities to assess the set of values that \(TT\) might take? One very
useful device that we can use is the confidence interval. Confidence
intervals are very useful because they quantify the zone within which we
have a chance to find the true effect \(TT\):

\BeginKnitrBlock{theorem}[Confidence interval]
\protect\hypertarget{thm:confinter}{}{\label{thm:confinter}
\iffalse (Confidence interval) \fi{} }For a given level of confidence
\(\delta\) and corresponding level of sampling noise \(2\epsilon\) of
the estimator \(\hat{E}\) of \(TT\), the confidence interval
\(\left\{\hat{E}-\epsilon,\hat{E}+\epsilon\right\}\) is such that the
probability that it contains \(TT\) is equal to \(\delta\) over sample
replications:

\begin{align*}
  \Pr(\hat{E}-\epsilon\leq TT\leq\hat{E}+\epsilon) & = \delta.
\end{align*}
\EndKnitrBlock{theorem}

\BeginKnitrBlock{proof}
\iffalse{} {Proof. } \fi{}From the definition of sampling noise, we know
that:

\begin{align*}
  \Pr(|\hat{E}-TT|\leq\epsilon) & = \delta.
\end{align*}

Now:

\begin{align*}
  \Pr(|\hat{E}-TT|\leq\epsilon) & = \Pr(TT-\epsilon\leq\hat{E}\leq TT+\epsilon)\\
                                & = \Pr(-\hat{E}-\epsilon\leq-TT\leq -\hat{E}+\epsilon)\\
                                & = \Pr(\hat{E}-\epsilon\leq TT\leq\hat{E}+\epsilon),
\end{align*}

which proves the result.
\EndKnitrBlock{proof}

It is very important to note that confidence intervals are centered
around \(\hat{E}\) and not around \(TT\). When estimating sampling noise
and building Figure \ref{fig:precision}, we have centered our intervals
around \(TT\). The interval was fixed and \(\hat{E}\) was moving across
replications and \(2\epsilon\) was defined as the length of the interval
around \(TT\) containing a proportion \(\delta\) of the estimates
\(\hat{E}\). A confidence interval cannot be centered around \(TT\),
which is unknown, but is centered around \(\hat{E}\), that we can
observe. As a consequence, it is the interval that moves around across
replications, and \(\delta\) is the proportion of samples in which the
interval contains \(TT\).

\BeginKnitrBlock{example}
\protect\hypertarget{exm:unnamed-chunk-36}{}{\label{exm:unnamed-chunk-36}
}Let's see how confidence intervals behave in our numerical example.
\EndKnitrBlock{example}

\begin{Shaded}
\begin{Highlighting}[]
\NormalTok{N.plot <-}\StringTok{ }\DecValTok{40}
\NormalTok{plot.list <-}\StringTok{ }\KeywordTok{list}\NormalTok{()}

\ControlFlowTok{for}\NormalTok{ (k }\ControlFlowTok{in} \DecValTok{1}\OperatorTok{:}\KeywordTok{length}\NormalTok{(N.sample))\{}
  \KeywordTok{set.seed}\NormalTok{(}\DecValTok{1234}\NormalTok{)}
\NormalTok{  test <-}\StringTok{ }\KeywordTok{sample}\NormalTok{(simuls.ww[[k]][,}\StringTok{'WW'}\NormalTok{],N.plot)}
\NormalTok{  test <-}\StringTok{ }\KeywordTok{as.data.frame}\NormalTok{(}\KeywordTok{cbind}\NormalTok{(test,}\KeywordTok{rep}\NormalTok{(}\KeywordTok{samp.noise}\NormalTok{(simuls.ww[[k]][,}\StringTok{'WW'}\NormalTok{],}\DataTypeTok{delta=}\NormalTok{delta)),}\KeywordTok{rep}\NormalTok{(}\KeywordTok{samp.noise}\NormalTok{(simuls.ww[[k]][,}\StringTok{'WW'}\NormalTok{],}\DataTypeTok{delta=}\NormalTok{delta.}\DecValTok{2}\NormalTok{))))}
  \KeywordTok{colnames}\NormalTok{(test) <-}\StringTok{ }\KeywordTok{c}\NormalTok{(}\StringTok{'WW'}\NormalTok{,}\StringTok{'sampling.noise.1'}\NormalTok{,}\StringTok{'sampling.noise.2'}\NormalTok{)}
\NormalTok{  test}\OperatorTok{$}\NormalTok{id <-}\StringTok{ }\DecValTok{1}\OperatorTok{:}\NormalTok{N.plot}
\NormalTok{  plot.test <-}\StringTok{ }\KeywordTok{ggplot}\NormalTok{(test, }\KeywordTok{aes}\NormalTok{(}\DataTypeTok{x=}\KeywordTok{as.factor}\NormalTok{(id), }\DataTypeTok{y=}\NormalTok{WW)) }\OperatorTok{+}
\StringTok{      }\KeywordTok{geom_bar}\NormalTok{(}\DataTypeTok{position=}\KeywordTok{position_dodge}\NormalTok{(), }\DataTypeTok{stat=}\StringTok{"identity"}\NormalTok{, }\DataTypeTok{colour=}\StringTok{'black'}\NormalTok{) }\OperatorTok{+}
\StringTok{      }\KeywordTok{geom_errorbar}\NormalTok{(}\KeywordTok{aes}\NormalTok{(}\DataTypeTok{ymin=}\NormalTok{WW}\OperatorTok{-}\NormalTok{sampling.noise.}\DecValTok{1}\OperatorTok{/}\DecValTok{2}\NormalTok{, }\DataTypeTok{ymax=}\NormalTok{WW}\OperatorTok{+}\NormalTok{sampling.noise.}\DecValTok{1}\OperatorTok{/}\DecValTok{2}\NormalTok{), }\DataTypeTok{width=}\NormalTok{.}\DecValTok{2}\NormalTok{,}\DataTypeTok{position=}\KeywordTok{position_dodge}\NormalTok{(.}\DecValTok{9}\NormalTok{),}\DataTypeTok{color=}\StringTok{'red'}\NormalTok{) }\OperatorTok{+}
\StringTok{      }\KeywordTok{geom_errorbar}\NormalTok{(}\KeywordTok{aes}\NormalTok{(}\DataTypeTok{ymin=}\NormalTok{WW}\OperatorTok{-}\NormalTok{sampling.noise.}\DecValTok{2}\OperatorTok{/}\DecValTok{2}\NormalTok{, }\DataTypeTok{ymax=}\NormalTok{WW}\OperatorTok{+}\NormalTok{sampling.noise.}\DecValTok{2}\OperatorTok{/}\DecValTok{2}\NormalTok{), }\DataTypeTok{width=}\NormalTok{.}\DecValTok{2}\NormalTok{,}\DataTypeTok{position=}\KeywordTok{position_dodge}\NormalTok{(.}\DecValTok{9}\NormalTok{),}\DataTypeTok{color=}\StringTok{'blue'}\NormalTok{) }\OperatorTok{+}
\StringTok{      }\KeywordTok{geom_hline}\NormalTok{(}\KeywordTok{aes}\NormalTok{(}\DataTypeTok{yintercept=}\KeywordTok{delta.y.ate}\NormalTok{(param)), }\DataTypeTok{colour=}\StringTok{"#990000"}\NormalTok{, }\DataTypeTok{linetype=}\StringTok{"dashed"}\NormalTok{)}\OperatorTok{+}
\StringTok{      }\CommentTok{#ylim(-0.5,1.2)+}
\StringTok{      }\KeywordTok{xlab}\NormalTok{(}\StringTok{"Sample id"}\NormalTok{)}\OperatorTok{+}
\StringTok{      }\KeywordTok{theme_bw}\NormalTok{()}\OperatorTok{+}
\StringTok{      }\KeywordTok{ggtitle}\NormalTok{(}\KeywordTok{paste}\NormalTok{(}\StringTok{"N="}\NormalTok{,N.sample[k]))}
\NormalTok{  plot.list[[k]] <-}\StringTok{ }\NormalTok{plot.test }
\NormalTok{\}}
\NormalTok{plot.CI <-}\StringTok{ }\KeywordTok{plot_grid}\NormalTok{(plot.list[[}\DecValTok{1}\NormalTok{]],plot.list[[}\DecValTok{2}\NormalTok{]],plot.list[[}\DecValTok{3}\NormalTok{]],plot.list[[}\DecValTok{4}\NormalTok{]],}\DataTypeTok{ncol=}\DecValTok{1}\NormalTok{,}\DataTypeTok{nrow=}\KeywordTok{length}\NormalTok{(N.sample))}
\KeywordTok{print}\NormalTok{(plot.CI)}
\end{Highlighting}
\end{Shaded}

\begin{figure}[htbp]

{\centering \includegraphics[width=0.6\linewidth]{STCI_files/figure-latex/confinterval-1} 

}

\caption{Confidence intervals of $\hat{WW}$ for $\delta=$ 0.99 (red) and 0.95 (blue) over sample replications for various sample sizes}\label{fig:confinterval}
\end{figure}

Figure \ref{fig:confinterval} presents the 99\% and 95\% confidence
intervals for 40 samples selected from our simulations. First,
confidence intervals do their job: they contain the true effect most of
the time. Second, the 95\% confidence interval misses the true effect
more often, as expected. For example, with \(N=\) 1000, the confidence
intervals in samples 13 and 23 do not contain the true effect, but it is
not far from their lower bound. Third, confidence intervals faithfully
reflect what we can learn from our estimates at each sample size. With
\(N=\) 100, the confidence intervals make it clear that the effect might
be very large or very small, even strongly negative. With \(N=\) 1000,
the confidence intervals suggest that the effect is either positive or
null, but unlikely to be strongly negative. Most of the time, we get the
sign right. With \(N=\) 10\^{}\{4\}, we know that the true effect is
bigger than 0.1 and smaller than 0.3 and most intervals place the true
effect somewhere between 0.11 and 0.25. With \(N=\) 10\^{}\{5\}, we know
that the true effect is bigger than 0.15 and smaller than 0.21 and most
intervals place the true effect somewhere between 0.16 and 0.20.

\subsection{Reporting sampling noise: a
proposal}\label{reporting-sampling-noise-a-proposal}

Once sampling noise is measured (and we'll see how to get an estimate in
the next section), one still has to communicate it to others. There are
many ways to report sampling noise:

\begin{itemize}
\tightlist
\item
  Sampling noise as defined in this book (\(2*\epsilon\))
\item
  The corresponding confidence interval
\item
  The signal to noise ratio
\item
  A standard error
\item
  A significance level
\item
  A p-value
\item
  A t-statistic
\end{itemize}

The main problem with all of these approaches is that they do not
express sampling noise in a way that is directly comparable to the
magnitude of the \(TT\) estimate. Other ways of reporting sampling noise
such as p-values and t-stats are nonlinear transforms of sampling noise,
making it difficult to really gauge the size of sampling noise as it
relates to the magnitude of \(TT\).

My own preference goes to the following format for reporting results:
\(TT \pm \epsilon\). As such, we can readily compare the size of the
noise to the sizee of the \(TT\) estimate. We can also form all the
other ways of expressing sampling noise directly.

\BeginKnitrBlock{example}
\protect\hypertarget{exm:unnamed-chunk-37}{}{\label{exm:unnamed-chunk-37}
}Let's see how this approach behaves in our numerical example.
\EndKnitrBlock{example}

\begin{Shaded}
\begin{Highlighting}[]
\NormalTok{test.all <-}\StringTok{ }\KeywordTok{list}\NormalTok{()}
\ControlFlowTok{for}\NormalTok{ (k }\ControlFlowTok{in} \DecValTok{1}\OperatorTok{:}\KeywordTok{length}\NormalTok{(N.sample))\{}
  \KeywordTok{set.seed}\NormalTok{(}\DecValTok{1234}\NormalTok{)}
\NormalTok{  test <-}\StringTok{ }\KeywordTok{sample}\NormalTok{(simuls.ww[[k]][,}\StringTok{'WW'}\NormalTok{],N.plot)}
\NormalTok{  test <-}\StringTok{ }\KeywordTok{as.data.frame}\NormalTok{(}\KeywordTok{cbind}\NormalTok{(test,}\KeywordTok{rep}\NormalTok{(}\KeywordTok{samp.noise}\NormalTok{(simuls.ww[[k]][,}\StringTok{'WW'}\NormalTok{],}\DataTypeTok{delta=}\NormalTok{delta)),}\KeywordTok{rep}\NormalTok{(}\KeywordTok{samp.noise}\NormalTok{(simuls.ww[[k]][,}\StringTok{'WW'}\NormalTok{],}\DataTypeTok{delta=}\NormalTok{delta.}\DecValTok{2}\NormalTok{))))}
  \KeywordTok{colnames}\NormalTok{(test) <-}\StringTok{ }\KeywordTok{c}\NormalTok{(}\StringTok{'WW'}\NormalTok{,}\StringTok{'sampling.noise.1'}\NormalTok{,}\StringTok{'sampling.noise.2'}\NormalTok{)}
\NormalTok{  test}\OperatorTok{$}\NormalTok{id <-}\StringTok{ }\DecValTok{1}\OperatorTok{:}\NormalTok{N.plot}
\NormalTok{  test.all[[k]] <-}\StringTok{ }\NormalTok{test}
\NormalTok{\}}
\end{Highlighting}
\end{Shaded}

With \(N=\) 100, the reporting of the results for sample 1 would be
something like: ``we find an effect of 0.07 \(\pm\) 0.55.'' Note how the
choice of \(\delta\) does not matter much for the result. The previous
result was for \(\delta=0.99\) while the result for \(\delta=0.95\)
would have been: ``we find an effect of 0.07 \(\pm\) 0.45.'' The precise
result changes with \(\delta\), but the qualitative result stays the
same: the magnitude of sampling noise is large and it dwarfs the
treatment effect estimate.

With \(N=\) 1000, the reporting of the results for sample 1 with
\(\delta=0.99\) would be something like: ``we find an effect of 0.19
\(\pm\) 0.2.'' With \(\delta=0.95\): ``we find an effect of 0.19 \(\pm\)
0.15.'' Again, although the precise quantitative result is affected by
the choice of \(\delta\), but hte qualitative message stays the same:
sampling noise is of the same order of magnitude as the estimated
treatment effect.

With \(N=\) 10\^{}\{4\}, the reporting of the results for sample 1 with
\(\delta=0.99\) would be something like: ``we find an effect of 0.2
\(\pm\) 0.06.'' With \(\delta=0.95\): ``we find an effect of 0.2 \(\pm\)
0.04.'' Again, see how the qualitative result is independent of the
precise choice of \(\delta\): sampling noise is almost one order of
magnitude smaller than the treatment effect estimate.

With \(N=\) 10\^{}\{5\}, the reporting of the results for sample 1 with
\(\delta=0.99\) would be something like: ``we find an effect of 0.17
\(\pm\) 0.02.'' With \(\delta=0.95\): ``we find an effect of 0.17
\(\pm\) 0.01.'' Again, see how the qualitative result is independent of
the precise choice of \(\delta\): sampling noise is one order of
magnitude smaller than the treatment effect estimate.

\BeginKnitrBlock{remark}
\iffalse{} {Remark. } \fi{}What I hope the example makes clear is that
my proposed way of reporting results gives the same importance to
sampling noise as it gives to the treatment effect estimate. Also,
comparing them is easy, without requiring a huge computational burden on
our brain.
\EndKnitrBlock{remark}

\BeginKnitrBlock{remark}
\iffalse{} {Remark. } \fi{}One problem with the approach that I propose
is when you have a non-symetric distribution of sampling noise, or when
\(TT \pm \epsilon\) exceeds natural bounds on \(TT\) (such as if the
effect cannot be bigger than one, for example). I think these issues are
minor and rare and can be dealt with on a case by case basis. The
advantage of having one simple and directly readable number comparable
to the magnitude of the treatment effect is overwhelming and makes this
approach the most natural and adequate, in my opinion.
\EndKnitrBlock{remark}

\subsection{Using effect sizes to normalize the reporting of treatment
effects and their
precision}\label{using-effect-sizes-to-normalize-the-reporting-of-treatment-effects-and-their-precision}

When looking at the effect of a program on an outcome, we depend on the
scaling on that outcome to appreciate the relative size of the estimated
treatment effect.\\
It is often difficult to appreciate the relative importance of the size
of an effect, even if we know the scale of the outcome of interest. One
useful device to normalize the treatment effects is called Cohen's
\(d\), or effect size. The idea is to compare the magnitude of the
treatment effect to an estimate of the usual amount of variation that
the outcome undergoes in the population. The way to build Cohen's \(d\)
is by dividing the estimated treatment effect by the standard deviation
of the outcome. I generally prefer to use the standard devaition of the
outcome in the control group, so as not to include the additional
amoiunt of variation due to the heterogeneity in treatment effects.

\BeginKnitrBlock{definition}[Cohen's $d$]
\protect\hypertarget{def:unnamed-chunk-40}{}{\label{def:unnamed-chunk-40}
\iffalse (Cohen's \(d\)) \fi{} }Cohen's \(d\), or effect size, is the
ratio of the estimated treatment effect to the standard deviation of
outcomes in the control group:
\EndKnitrBlock{definition}

\[
d = \frac{\hat{TT}}{\sqrt{\frac{1}{N^0}\sum_{i=1}^{N^0}(Y_i-\bar{Y^0})^2}}
\] where \(\hat{TT}\) is an estimate of the treatment effect, \(N^0\) is
the number of individuals in the treatment group and \(\bar{Y^0}\) is
the average outcome in the treatment group.

Cohen's \(d\) can be interpreted in terms of magnitude of effect size:

\begin{itemize}
\tightlist
\item
  It is generally considered that an effect is large when its \(d\) is
  larger than 0.8.
\item
  An effect size around 0.5 is considered medium
\item
  An effect size around 0.2 is considered to be small
\item
  An effect size around 0.02 is considered to be very small.
\end{itemize}

There probably could be a rescaling of these terms, but that is the
actual state of the art.

What I like about effect sizes is that they encourage an interpretation
of the order of magnitude of the treatment effect. As such, they enable
to include the information on precision by looking at which orders of
magnitude are compatible with the estimated effect at the estimated
precision level. Effect sizes and orders of magnitude help make us aware
that our results might be imprecise, and that the precise value that we
have estimated is probably not the truth. What is important is the range
of effect sizes compatible with our results (both point estimate and
precision).

\BeginKnitrBlock{example}
\protect\hypertarget{exm:unnamed-chunk-41}{}{\label{exm:unnamed-chunk-41}
}Let's see how Cohen's \(d\) behaves in our numerical example.
\EndKnitrBlock{example}

The value of Cohen's \(d\) (or effect size) in the population is equal
to:

\begin{align*}
  ES & = \frac{TT}{\sqrt{V^0}} = \frac{\bar{\alpha}+\theta\bar{\mu}}{\sqrt{\sigma^2_{\mu}+\rho^2\sigma^2_{U}+\sigma^2_{\epsilon}}}
\end{align*}

We can write a function to compute this parameter, as well as functions
to implement its estimator in the simulated samples:

\begin{Shaded}
\begin{Highlighting}[]
\NormalTok{V0 <-}\StringTok{ }\ControlFlowTok{function}\NormalTok{(param)\{}
  \KeywordTok{return}\NormalTok{(param[}\StringTok{"sigma2mu"}\NormalTok{]}\OperatorTok{+}\NormalTok{param[}\StringTok{"rho"}\NormalTok{]}\OperatorTok{^}\DecValTok{2}\OperatorTok{*}\NormalTok{param[}\StringTok{"sigma2U"}\NormalTok{]}\OperatorTok{+}\NormalTok{param[}\StringTok{"sigma2epsilon"}\NormalTok{])}
\NormalTok{\}}

\NormalTok{ES <-}\StringTok{ }\ControlFlowTok{function}\NormalTok{(param)\{}
  \KeywordTok{return}\NormalTok{(}\KeywordTok{delta.y.ate}\NormalTok{(param)}\OperatorTok{/}\KeywordTok{sqrt}\NormalTok{(}\KeywordTok{V0}\NormalTok{(param)))}
\NormalTok{\}}

\NormalTok{samp.noise.ES <-}\StringTok{ }\ControlFlowTok{function}\NormalTok{(estim,delta,}\DataTypeTok{param=}\NormalTok{param)\{}
  \KeywordTok{return}\NormalTok{(}\DecValTok{2}\OperatorTok{*}\KeywordTok{quantile}\NormalTok{(}\KeywordTok{abs}\NormalTok{(}\KeywordTok{delta.y.ate}\NormalTok{(param)}\OperatorTok{/}\KeywordTok{sqrt}\NormalTok{(}\KeywordTok{V0}\NormalTok{(param))}\OperatorTok{-}\NormalTok{estim),}\DataTypeTok{prob=}\NormalTok{delta))}
\NormalTok{\}}

\ControlFlowTok{for}\NormalTok{ (i }\ControlFlowTok{in} \DecValTok{1}\OperatorTok{:}\DecValTok{4}\NormalTok{)\{}
\NormalTok{  simuls.ww[[i]][,}\StringTok{'ES'}\NormalTok{] <-}\StringTok{ }\NormalTok{simuls.ww[[i]][,}\StringTok{'WW'}\NormalTok{]}\OperatorTok{/}\KeywordTok{sqrt}\NormalTok{(simuls.ww[[i]][,}\StringTok{'V0'}\NormalTok{])}
\NormalTok{\}}
\end{Highlighting}
\end{Shaded}

The true effect size in the population is thus 0.2. It is considered to
be small according to the current classification, although I'd say that
a treatment able to move the outcomes by 20\% of their usual variation
is a pretty effective treatment, and this effect should be labelled at
least medium. Let's stick with the classification though. In our
example, the effect size does not differ much from the treatment effect
since the standard deviation of outcomes in the control group is pretty
close to one: it is equal to 0.88. Let's now build confidence intervals
for the effect size and try to comment on the magnitudes of these
effects using the normalized classification.

\begin{Shaded}
\begin{Highlighting}[]
\NormalTok{N.plot.ES <-}\StringTok{ }\DecValTok{40}
\NormalTok{plot.list.ES <-}\StringTok{ }\KeywordTok{list}\NormalTok{()}

\ControlFlowTok{for}\NormalTok{ (k }\ControlFlowTok{in} \DecValTok{1}\OperatorTok{:}\KeywordTok{length}\NormalTok{(N.sample))\{}
  \KeywordTok{set.seed}\NormalTok{(}\DecValTok{1234}\NormalTok{)}
\NormalTok{  test.ES <-}\StringTok{ }\KeywordTok{sample}\NormalTok{(simuls.ww[[k]][,}\StringTok{'ES'}\NormalTok{],N.plot)}
\NormalTok{  test.ES <-}\StringTok{ }\KeywordTok{as.data.frame}\NormalTok{(}\KeywordTok{cbind}\NormalTok{(test.ES,}\KeywordTok{rep}\NormalTok{(}\KeywordTok{samp.noise.ES}\NormalTok{(simuls.ww[[k]][,}\StringTok{'ES'}\NormalTok{],}\DataTypeTok{delta=}\NormalTok{delta,}\DataTypeTok{param=}\NormalTok{param)),}\KeywordTok{rep}\NormalTok{(}\KeywordTok{samp.noise.ES}\NormalTok{(simuls.ww[[k]][,}\StringTok{'ES'}\NormalTok{],}\DataTypeTok{delta=}\NormalTok{delta.}\DecValTok{2}\NormalTok{,}\DataTypeTok{param=}\NormalTok{param))))}
  \KeywordTok{colnames}\NormalTok{(test.ES) <-}\StringTok{ }\KeywordTok{c}\NormalTok{(}\StringTok{'ES'}\NormalTok{,}\StringTok{'sampling.noise.ES.1'}\NormalTok{,}\StringTok{'sampling.noise.ES.2'}\NormalTok{)}
\NormalTok{  test.ES}\OperatorTok{$}\NormalTok{id <-}\StringTok{ }\DecValTok{1}\OperatorTok{:}\NormalTok{N.plot.ES}
\NormalTok{  plot.test.ES <-}\StringTok{ }\KeywordTok{ggplot}\NormalTok{(test.ES, }\KeywordTok{aes}\NormalTok{(}\DataTypeTok{x=}\KeywordTok{as.factor}\NormalTok{(id), }\DataTypeTok{y=}\NormalTok{ES)) }\OperatorTok{+}
\StringTok{      }\KeywordTok{geom_bar}\NormalTok{(}\DataTypeTok{position=}\KeywordTok{position_dodge}\NormalTok{(), }\DataTypeTok{stat=}\StringTok{"identity"}\NormalTok{, }\DataTypeTok{colour=}\StringTok{'black'}\NormalTok{) }\OperatorTok{+}
\StringTok{      }\KeywordTok{geom_errorbar}\NormalTok{(}\KeywordTok{aes}\NormalTok{(}\DataTypeTok{ymin=}\NormalTok{ES}\OperatorTok{-}\NormalTok{sampling.noise.ES.}\DecValTok{1}\OperatorTok{/}\DecValTok{2}\NormalTok{, }\DataTypeTok{ymax=}\NormalTok{ES}\OperatorTok{+}\NormalTok{sampling.noise.ES.}\DecValTok{1}\OperatorTok{/}\DecValTok{2}\NormalTok{), }\DataTypeTok{width=}\NormalTok{.}\DecValTok{2}\NormalTok{,}\DataTypeTok{position=}\KeywordTok{position_dodge}\NormalTok{(.}\DecValTok{9}\NormalTok{),}\DataTypeTok{color=}\StringTok{'red'}\NormalTok{) }\OperatorTok{+}
\StringTok{      }\KeywordTok{geom_errorbar}\NormalTok{(}\KeywordTok{aes}\NormalTok{(}\DataTypeTok{ymin=}\NormalTok{ES}\OperatorTok{-}\NormalTok{sampling.noise.ES.}\DecValTok{2}\OperatorTok{/}\DecValTok{2}\NormalTok{, }\DataTypeTok{ymax=}\NormalTok{ES}\OperatorTok{+}\NormalTok{sampling.noise.ES.}\DecValTok{2}\OperatorTok{/}\DecValTok{2}\NormalTok{), }\DataTypeTok{width=}\NormalTok{.}\DecValTok{2}\NormalTok{,}\DataTypeTok{position=}\KeywordTok{position_dodge}\NormalTok{(.}\DecValTok{9}\NormalTok{),}\DataTypeTok{color=}\StringTok{'blue'}\NormalTok{) }\OperatorTok{+}
\StringTok{      }\KeywordTok{geom_hline}\NormalTok{(}\KeywordTok{aes}\NormalTok{(}\DataTypeTok{yintercept=}\KeywordTok{delta.y.ate}\NormalTok{(param)}\OperatorTok{/}\KeywordTok{sqrt}\NormalTok{(}\KeywordTok{V0}\NormalTok{(param))), }\DataTypeTok{colour=}\StringTok{"#990000"}\NormalTok{, }\DataTypeTok{linetype=}\StringTok{"dashed"}\NormalTok{)}\OperatorTok{+}
\StringTok{      }\CommentTok{#ylim(-0.5,1.2)+}
\StringTok{      }\KeywordTok{xlab}\NormalTok{(}\StringTok{"Sample id"}\NormalTok{)}\OperatorTok{+}
\StringTok{      }\KeywordTok{ylab}\NormalTok{(}\StringTok{"Effect Size"}\NormalTok{)}\OperatorTok{+}
\StringTok{      }\KeywordTok{theme_bw}\NormalTok{()}\OperatorTok{+}
\StringTok{      }\KeywordTok{ggtitle}\NormalTok{(}\KeywordTok{paste}\NormalTok{(}\StringTok{"N="}\NormalTok{,N.sample[k]))}
\NormalTok{  plot.list.ES[[k]] <-}\StringTok{ }\NormalTok{plot.test.ES }
\NormalTok{\}}
\NormalTok{plot.CI.ES <-}\StringTok{ }\KeywordTok{plot_grid}\NormalTok{(plot.list.ES[[}\DecValTok{1}\NormalTok{]],plot.list.ES[[}\DecValTok{2}\NormalTok{]],plot.list.ES[[}\DecValTok{3}\NormalTok{]],plot.list.ES[[}\DecValTok{4}\NormalTok{]],}\DataTypeTok{ncol=}\DecValTok{1}\NormalTok{,}\DataTypeTok{nrow=}\KeywordTok{length}\NormalTok{(N.sample))}
\KeywordTok{print}\NormalTok{(plot.CI.ES)}
\end{Highlighting}
\end{Shaded}

\begin{figure}[htbp]

{\centering \includegraphics[width=0.6\linewidth]{STCI_files/figure-latex/confintervalES-1} 

}

\caption{Confidence intervals of $\hat{ES}$ for $\delta=$ 0.99 (red) and 0.95 (blue) over sample replications for various sample sizes}\label{fig:confintervalES}
\end{figure}

Figure \ref{fig:confintervalES} presents the 99\% and 95\% confidence
intervals for the effect size estimated in 40 samples selected from our
simulations. Let's regroup our estimate and see how we could present
their results.

\begin{Shaded}
\begin{Highlighting}[]
\NormalTok{test.all.ES <-}\StringTok{ }\KeywordTok{list}\NormalTok{()}
\ControlFlowTok{for}\NormalTok{ (k }\ControlFlowTok{in} \DecValTok{1}\OperatorTok{:}\KeywordTok{length}\NormalTok{(N.sample))\{}
  \KeywordTok{set.seed}\NormalTok{(}\DecValTok{1234}\NormalTok{)}
\NormalTok{  test.ES <-}\StringTok{ }\KeywordTok{sample}\NormalTok{(simuls.ww[[k]][,}\StringTok{'ES'}\NormalTok{],N.plot)}
\NormalTok{  test.ES <-}\StringTok{ }\KeywordTok{as.data.frame}\NormalTok{(}\KeywordTok{cbind}\NormalTok{(test.ES,}\KeywordTok{rep}\NormalTok{(}\KeywordTok{samp.noise.ES}\NormalTok{(simuls.ww[[k]][,}\StringTok{'ES'}\NormalTok{],}\DataTypeTok{delta=}\NormalTok{delta,}\DataTypeTok{param=}\NormalTok{param)),}\KeywordTok{rep}\NormalTok{(}\KeywordTok{samp.noise.ES}\NormalTok{(simuls.ww[[k]][,}\StringTok{'ES'}\NormalTok{],}\DataTypeTok{delta=}\NormalTok{delta.}\DecValTok{2}\NormalTok{,}\DataTypeTok{param=}\NormalTok{param))))}
  \KeywordTok{colnames}\NormalTok{(test.ES) <-}\StringTok{ }\KeywordTok{c}\NormalTok{(}\StringTok{'ES'}\NormalTok{,}\StringTok{'sampling.noise.ES.1'}\NormalTok{,}\StringTok{'sampling.noise.ES.2'}\NormalTok{)}
\NormalTok{  test.ES}\OperatorTok{$}\NormalTok{id <-}\StringTok{ }\DecValTok{1}\OperatorTok{:}\NormalTok{N.plot.ES}
\NormalTok{  test.all.ES[[k]] <-}\StringTok{ }\NormalTok{test.ES}
\NormalTok{\}}
\end{Highlighting}
\end{Shaded}

With \(N=\) 100, the reporting of the results for sample 1 would be
something like: ``we find an effect size of 0.09 \(\pm\) 0.66'' with
\(\delta=0,99\). With \(\delta=0.95\) we would say: ``we find an effect
of 0.09 \(\pm\) 0.5.'' All in all, our estimate is compatible with the
treatment having a large positive effect size and a medium negative
effect size. Low precision prevents us from saying much else.

With \(N=\) 1000, the reporting of the results for sample 1 with
\(\delta=0.99\) would be something like: ``we find an effect size of
0.21 \(\pm\) 0.22.'' With \(\delta=0.95\): ``we find an effect size of
0.21 \(\pm\) 0.17.'' Our estimate is compatible with a medium positive
effect or a very small positive or even negative effect (depending on
the choice of \(\delta\)).

With \(N=\) 10\^{}\{4\}, the reporting of the results for sample 1 with
\(\delta=0.99\) would be something like: ``we find an effect size of
0.22 \(\pm\) 0.07.'' With \(\delta=0.95\): ``we find an effect size of
0.22 \(\pm\) 0.05.'' Our estimate is thus compatible with a small effect
of the treatment. We can rule out that the effect of the treatment is
medium since the upper bound of the 99\% confidence interval is equal to
0.29. We can also rule out that the effect of the treatment is very
small since the lower bound of the 99\% confidence interval is equal to
0.16. With this sample size, we have been able to reach a precision
level sufficient enough to pin down the order of magnitude of the effect
size of our treatment. There still remains a considerable amount of
uncertainty about the true effect size, though: the upper bound of our
confidence interval is almost double the lower bound.

With \(N=\) 10\^{}\{5\}, the reporting of the results for sample 1 with
\(\delta=0.99\) would be something like: ``we find an effect size of 0.2
\(\pm\) 0.02.'' With \(\delta=0.95\): ``we find an effect size of 0.2
\(\pm\) 0.02.'' Here, the level of precision of our result is such that,
first, it does not depend on the choice of \(\delta\) in any meaningful
way, and second, we can do more than pinpoint the order of magnitude of
the effect size, we can start to zero in on its precise value. From our
estimate, the true value of the effect size is really close to 0.2. It
could be equal to 0.18 or 0.22, but not further away from 0.2 than that.
Remember that is actually equal to 0.2.

\BeginKnitrBlock{remark}
\iffalse{} {Remark. } \fi{}One issue with Cohen's \(d\) is that its
magnitude depends on the dispersion of the outcomes in the control
group. That means that for the same treatment, and same value of the
treatment effect, the effect size is larger in a population where
oucomes are more homogeneous. This is not an attractive feature of a
normalizing scale that its size depends on the particular application.
One solution would be, for each outcome, to provide a standardized
scale, using for example the estmated standard deviation in a reference
population. This would be similar to the invention of the metric system,
where a reference scale was agreed uppon once and for all.
\EndKnitrBlock{remark}

\BeginKnitrBlock{remark}
\iffalse{} {Remark. } \fi{}Cohen's \(d\) is well defined for continuous
outcomes. For discrete outcomes, the use of Cohen's \(d\) poses a series
of problems, and alternatives such as relative risk ratios and odds
ratios have been proposed. I'll comment on that in the last chapter.
\EndKnitrBlock{remark}

\section{Estimating sampling noise}\label{sec:estimsampnoise}

Gauging the extent sampling noise is very useful in order to be able to
determine how much we should trust our results. Are they precise, so
that the true treatment effect lies very close to our estimate? Or are
our results imprecise, the true treatment effect maybe lying very far
from our estimate?

Estimating sampling noise is hard because we want to infer a property of
our estimator over repeated samples using only one sample. In this
lecture, I am going to introduce four tools that enable you to gauge
sampling noise and to choose sample size. The four tools are Chebyshev's
inequality, the Central Limit Theorem, resampling methods and Fisher's
permutation method. The idea of all these methods is to use the
properties of the sample to infer the properties of our estimator over
replications. Chebyshev's inequality gives an upper bound on the
sampling noise and a lower bound on sample size, but these bounds are
generally too wide to be useful. The Central Limit Theorem (CLT)
approximates the distribution of \(\hat{E}\) by a normal distribution,
and quantifies sampling noise as a multiple of the standard deviation.
Resampling methods use the sample as a population and draw new samples
from it in order to approximate sampling noise. Fisher's permutation
method, also called randomization inference, derives the distribution of
\(\hat{E}\) under the assumption that all treatment effects are null, by
reallocating the treatment indicator among the treatment and control
group. Both the CLT and resampling methods are approximation methods,
and their approximation of the true extent of sampling noise gets better
and better as sample size increases. Fisher's permutation method is
exact-it is not an approximation-but it only works for the special case
of the \(WW\) estimator in a randomized design.

The remaining of this section is structured as follows. Section
\ref{sec:assumptions} introduces the assumptions that we will need in
order to implement the methods. Section \ref{sec:cheb} presents the
Chebyshev approach to gauging sampling noise and choosing sample size.
Section \ref{sec:CLT} introduces the CLT way of approximating sampling
noise and choosing sample size. Section \ref{sec:resamp} presents the
resampling methods. Section \ref{sec:Fisher} introduces the Fisher's
exact permutation approach.

\BeginKnitrBlock{remark}
\iffalse{} {Remark. } \fi{}I am going to derive the estimators for the
precision only for the \(WW\) estimator. In the following lectures, I
will show how these methods adapt to other estimators.
\EndKnitrBlock{remark}

\subsection{Assumptions}\label{sec:assumptions}

In order to be able to use the theorems that power up the methods that
we are going to use to gauge sampling noise, we need to make some
assumptions on the properties of the data. The main assumptions that we
need are that the estimator identifies the true effect of the treatment
in the population, that the estimator is well-defined in the sample,
that the observations in the sample are independently and identically
distributed (i.i.d.), that there is no interaction between units and
that the variances of the outcomes in the treated and untreated group
are finite.

We know from last lecture that for the \(WW\) estimator to identify
\(TT\), we need to assume that there is no selection bias, as stated in
Assumption \ref{def:noselb}. One way to ensure that this assumption
holds is to use a RCT.

In order to be able to form the \(WW\) estimator in the sample, we also
need that there is at least one treated and one untreated in the sample:

\BeginKnitrBlock{definition}[Full rank]
\protect\hypertarget{def:fullrank}{}{\label{def:fullrank} \iffalse (Full
rank) \fi{} }We assume that there is at least one observation in the
sample that receives the treatment and one observation that does not
receive it:

\begin{align*}
\exists i,j\leq N \text{ such that } & D_i=1 \& D_j=0.
\end{align*}
\EndKnitrBlock{definition}

One way to ensure that this assumption holds is to sample treated and
untreated units.

In order to be able to estimate the variance of the estimator easily, we
assume that the observations come from random sampling and are i.i.d.:

\BeginKnitrBlock{definition}[i.i.d. sampling]
\protect\hypertarget{def:iid}{}{\label{def:iid} \iffalse (i.i.d. sampling)
\fi{} }We assume that the observations in the sample are identically and
independently distributed:

\begin{align*}
\forall i,j\leq N\text{, }i\neq j\text{, } & (Y_i,D_i)\Ind(Y_j,D_j),\\
                                           & (Y_i,D_i)\&(Y_j,D_j)\sim F_{Y,D}.
\end{align*}
\EndKnitrBlock{definition}

We have to assume something on how the observations are related to each
other and to the population. Identical sampling is natural in the sense
that we are OK to assume that the observations stem from the same
population model. Independent sampling is something else altogether.
Independence means that the fates of two closely related individuals are
assumed to be independent. This rules out two empirically relevant
scenarios:

\begin{enumerate}
\def\labelenumi{\arabic{enumi}.}
\tightlist
\item
  The fates of individuals are related because of common influences, as
  for example the environment, etc,
\item
  The fates of individuals are related because they directly influence
  each other, as for example on a market, but also for example because
  there are diffusion effects, such as contagion of deseases or
  technological adoption by imitation.
\end{enumerate}

We will address both sources of failure of the independence assumption
in future lectures.

Finally, in order for all our derivations to make sense, we need to
assume that the outcomes in both groups have finite variances, otherwise
sampling noise is going to be too extreme to be able to estimate it
using the methods developed in this lecture:

\BeginKnitrBlock{definition}[Finite variance of $\hat{\Delta^Y_{WW}}$]
\protect\hypertarget{def:finitevar}{}{\label{def:finitevar} \iffalse (Finite
variance of \(\hat{\Delta^Y_{WW}}\)) \fi{} }We assume that
\(\var{Y^1|D_i=1}\) and \(\var{Y^0|D_i=0}\) are finite.
\EndKnitrBlock{definition}

\subsection{Using Chebyshev's inequality}\label{sec:cheb}

Chebyshev's inequality is a fundamental building block of statistics. It
relates the sampling noise of an estimator to its variance. More
precisely, it derives an upper bound on the samplig noise of an unbiased
estimator:

\BeginKnitrBlock{theorem}[Chebyshev's inequality]
\protect\hypertarget{thm:cheb}{}{\label{thm:cheb} \iffalse (Chebyshev's
inequality) \fi{} }For any unbiased estimator \(\hat{\theta}\), sampling
noise level \(2\epsilon\) and confidence level \(\delta\), sampling
noise is bounded from above:

\begin{align*}
2\epsilon \leq 2\sqrt{\frac{\var{\hat{\theta}}}{1-\delta}}.
\end{align*}
\EndKnitrBlock{theorem}

\BeginKnitrBlock{remark}
\iffalse{} {Remark. } \fi{}The more general version of Chebyshev's
inequality that is generally presented is as follows:

\begin{align*}
\Pr(|\hat{\theta}-\esp{\hat{\theta}}|>\epsilon) & \leq \frac{\var{\hat{\theta}}}{\epsilon^2}.
\end{align*}

The version I present in Theorem \ref{thm:cheb} is adapted to the
bouding of sampling noise for a given confidence level, while this
version is adapted to bounding the confidence level for a given level of
sampling noise. In order to go from this general version to Theorem
\ref{thm:cheb}, simply remember that, for an unbiased estimator,
\(\esp{\hat{\theta}}=\theta\) and that, by definition of sampling noise,
\(\Pr(|\hat{\theta}-\theta|>\epsilon)=1-\delta\). As a result,
\(1-\delta\leq\var{\hat{\theta}}/\epsilon^2\), hence the result in
Theorem \ref{thm:cheb}.
\EndKnitrBlock{remark}

Using Chebyshev's inequality, we can obtain an upper bound on the
sampling noise of the \(WW\) estimator:

\BeginKnitrBlock{theorem}[Upper bound on the sampling noise of $\hat{WW}$]
\protect\hypertarget{thm:uppsampnoise}{}{\label{thm:uppsampnoise}
\iffalse (Upper bound on the sampling noise of \(\hat{WW}\)) \fi{}
}Under Assumptions \ref{def:noselb}, \ref{def:fullrank} and
\ref{def:iid}, for a given confidence level \(\delta\), the sampling
noise of the \(\hat{WW}\) estimator is bounded from above:

\begin{align*}
2\epsilon \leq 2\sqrt{\frac{1}{N(1-\delta)}\left(\frac{\var{Y_i^1|D_i=1}}{\Pr(D_i=1)}+\frac{\var{Y_i^0|D_i=0}}{1-\Pr(D_i=1)}\right)}\equiv 2\bar{\epsilon}.
\end{align*}
\EndKnitrBlock{theorem}

\BeginKnitrBlock{proof}
\iffalse{} {Proof. } \fi{}See in appendix \ref{sec:proofcheb}
\EndKnitrBlock{proof}

Theorem \ref{thm:uppsampnoise} is a useful step forward for estimating
sampling noise. Theorem \ref{thm:uppsampnoise} states that the actual
level of sampling noise of the \(\hat{WW}\) estimator (\(2\epsilon\)) is
never bigger than a quantity that depends on sample size, confidence
level and on the variances of outcomes in the treated and control
groups. We either know all the components of the formula for
\(2\bar{\epsilon}\) or we can estimate them in the sample. For example,
\(\Pr(D_i=1)\), \(\var{Y_i^1|D_i=1}\) and \(\var{Y_i^0|D_i=0}\) by can
be approximated by, respectively:

\begin{align*}
  \hat{\Pr(D_i=1)} & = \frac{1}{N}\sum_{i=1}^ND_i\\
  \hat{\var{Y_i^1|D_i=1}} & = \frac{1}{\sum_{i=1}^ND_i}\sum_{i=1}^ND_i(Y_i-\frac{1}{\sum_{i=1}^ND_i}\sum_{i=1}^ND_iY_i)^2\\
  \hat{\var{Y_i^0|D_i=0}} & = \frac{1}{\sum_{i=1}^N(1-D_i)}\sum_{i=1}^N(1-D_i)(Y_i-\frac{1}{\sum_{i=1}^N(1-D_i)}\sum_{i=1}^N(1-D_i)Y_i)^2.
\end{align*}

Using these approximations for the quantities in the formula, we can
compute an estimate of the upper bound on sampling noise,
\(\hat{2\bar{\epsilon}}\).

\BeginKnitrBlock{example}
\protect\hypertarget{exm:unnamed-chunk-47}{}{\label{exm:unnamed-chunk-47}
}Let's write an R function that is going to compute an estimate for the
upper bound of sampling noise for any sample:
\EndKnitrBlock{example}

\begin{Shaded}
\begin{Highlighting}[]
\NormalTok{samp.noise.ww.cheb <-}\StringTok{ }\ControlFlowTok{function}\NormalTok{(N,delta,v1,v0,p)\{}
  \KeywordTok{return}\NormalTok{(}\DecValTok{2}\OperatorTok{*}\KeywordTok{sqrt}\NormalTok{((v1}\OperatorTok{/}\NormalTok{p}\OperatorTok{+}\NormalTok{v0}\OperatorTok{/}\NormalTok{(}\DecValTok{1}\OperatorTok{-}\NormalTok{p))}\OperatorTok{/}\NormalTok{(N}\OperatorTok{*}\NormalTok{(}\DecValTok{1}\OperatorTok{-}\NormalTok{delta))))}
\NormalTok{\}}
\end{Highlighting}
\end{Shaded}

Let's estimate this upper bound in our usual sample:

\begin{Shaded}
\begin{Highlighting}[]
\KeywordTok{set.seed}\NormalTok{(}\DecValTok{1234}\NormalTok{)}
\NormalTok{N <-}\DecValTok{1000}
\NormalTok{delta <-}\StringTok{ }\FloatTok{0.99}
\NormalTok{mu <-}\StringTok{ }\KeywordTok{rnorm}\NormalTok{(N,param[}\StringTok{"barmu"}\NormalTok{],}\KeywordTok{sqrt}\NormalTok{(param[}\StringTok{"sigma2mu"}\NormalTok{]))}
\NormalTok{UB <-}\StringTok{ }\KeywordTok{rnorm}\NormalTok{(N,}\DecValTok{0}\NormalTok{,}\KeywordTok{sqrt}\NormalTok{(param[}\StringTok{"sigma2U"}\NormalTok{]))}
\NormalTok{yB <-}\StringTok{ }\NormalTok{mu }\OperatorTok{+}\StringTok{ }\NormalTok{UB }
\NormalTok{YB <-}\StringTok{ }\KeywordTok{exp}\NormalTok{(yB)}
\NormalTok{Ds <-}\StringTok{ }\KeywordTok{rep}\NormalTok{(}\DecValTok{0}\NormalTok{,N)}
\NormalTok{V <-}\StringTok{ }\KeywordTok{rnorm}\NormalTok{(N,param[}\StringTok{"barmu"}\NormalTok{],}\KeywordTok{sqrt}\NormalTok{(param[}\StringTok{"sigma2mu"}\NormalTok{]}\OperatorTok{+}\NormalTok{param[}\StringTok{"sigma2U"}\NormalTok{]))}
\NormalTok{Ds[V}\OperatorTok{<=}\KeywordTok{log}\NormalTok{(param[}\StringTok{"barY"}\NormalTok{])] <-}\StringTok{ }\DecValTok{1} 
\NormalTok{epsilon <-}\StringTok{ }\KeywordTok{rnorm}\NormalTok{(N,}\DecValTok{0}\NormalTok{,}\KeywordTok{sqrt}\NormalTok{(param[}\StringTok{"sigma2epsilon"}\NormalTok{]))}
\NormalTok{eta<-}\StringTok{ }\KeywordTok{rnorm}\NormalTok{(N,}\DecValTok{0}\NormalTok{,}\KeywordTok{sqrt}\NormalTok{(param[}\StringTok{"sigma2eta"}\NormalTok{]))}
\NormalTok{U0 <-}\StringTok{ }\NormalTok{param[}\StringTok{"rho"}\NormalTok{]}\OperatorTok{*}\NormalTok{UB }\OperatorTok{+}\StringTok{ }\NormalTok{epsilon}
\NormalTok{y0 <-}\StringTok{ }\NormalTok{mu }\OperatorTok{+}\StringTok{  }\NormalTok{U0 }\OperatorTok{+}\StringTok{ }\NormalTok{param[}\StringTok{"delta"}\NormalTok{]}
\NormalTok{alpha <-}\StringTok{ }\NormalTok{param[}\StringTok{"baralpha"}\NormalTok{]}\OperatorTok{+}\StringTok{  }\NormalTok{param[}\StringTok{"theta"}\NormalTok{]}\OperatorTok{*}\NormalTok{mu }\OperatorTok{+}\StringTok{ }\NormalTok{eta}
\NormalTok{y1 <-}\StringTok{ }\NormalTok{y0}\OperatorTok{+}\NormalTok{alpha}
\NormalTok{Y0 <-}\StringTok{ }\KeywordTok{exp}\NormalTok{(y0)}
\NormalTok{Y1 <-}\StringTok{ }\KeywordTok{exp}\NormalTok{(y1)}
\NormalTok{y <-}\StringTok{ }\NormalTok{y1}\OperatorTok{*}\NormalTok{Ds}\OperatorTok{+}\NormalTok{y0}\OperatorTok{*}\NormalTok{(}\DecValTok{1}\OperatorTok{-}\NormalTok{Ds)}
\NormalTok{Y <-}\StringTok{ }\NormalTok{Y1}\OperatorTok{*}\NormalTok{Ds}\OperatorTok{+}\NormalTok{Y0}\OperatorTok{*}\NormalTok{(}\DecValTok{1}\OperatorTok{-}\NormalTok{Ds)}
\end{Highlighting}
\end{Shaded}

In our sample, for \(\delta=\) 0.99, \(\hat{2\bar{\epsilon}}=\) 1.35.
How does this compare with the true extent of sampling noise when \(N=\)
1000? Remember that we have computed an estimate of sampling noise out
of our Monte Carlo replications. In Table \ref{fig:precision}, we can
read that sampling noise is actually equal to 0.39. The Chebyshev upper
bound overestimates the extent of sampling noise by 245\%.

How does the Chebyshev upper bound fares overall? In order to know that,
let's compute the Chebyshev upper bound for all the simulated samples.
You might have noticed that, when running the Monte Carlo simulations
for the population parameter, I have not only recovered \(\hat{WW}\) for
each sample, but also the estimates of the components of the formula for
the upper bound on sampling noise. I can thus easily compute the
Chebyshev upper bound on sampling noise for each replication.

\begin{Shaded}
\begin{Highlighting}[]
\ControlFlowTok{for}\NormalTok{ (k }\ControlFlowTok{in}\NormalTok{ (}\DecValTok{1}\OperatorTok{:}\KeywordTok{length}\NormalTok{(N.sample)))\{}
\NormalTok{  simuls.ww[[k]]}\OperatorTok{$}\NormalTok{cheb.noise <-}\StringTok{ }\KeywordTok{samp.noise.ww.cheb}\NormalTok{(N.sample[[k]],delta,simuls.ww[[k]][,}\StringTok{'V1'}\NormalTok{],simuls.ww[[k]][,}\StringTok{'V0'}\NormalTok{],simuls.ww[[k]][,}\StringTok{'p'}\NormalTok{])}
\NormalTok{\}}
\KeywordTok{par}\NormalTok{(}\DataTypeTok{mfrow=}\KeywordTok{c}\NormalTok{(}\DecValTok{2}\NormalTok{,}\DecValTok{2}\NormalTok{))}
\ControlFlowTok{for}\NormalTok{ (i }\ControlFlowTok{in} \DecValTok{1}\OperatorTok{:}\DecValTok{4}\NormalTok{)\{}
  \KeywordTok{hist}\NormalTok{(simuls.ww[[i]][,}\StringTok{'cheb.noise'}\NormalTok{],}\DataTypeTok{main=}\KeywordTok{paste}\NormalTok{(}\StringTok{'N='}\NormalTok{,}\KeywordTok{as.character}\NormalTok{(N.sample[i])),}\DataTypeTok{xlab=}\KeywordTok{expression}\NormalTok{(}\KeywordTok{hat}\NormalTok{(}\DecValTok{2}\OperatorTok{*}\KeywordTok{bar}\NormalTok{(epsilon))),}\DataTypeTok{xlim=}\KeywordTok{c}\NormalTok{(}\FloatTok{0.25}\OperatorTok{*}\KeywordTok{min}\NormalTok{(simuls.ww[[i]][,}\StringTok{'cheb.noise'}\NormalTok{]),}\KeywordTok{max}\NormalTok{(simuls.ww[[i]][,}\StringTok{'cheb.noise'}\NormalTok{])))}
  \KeywordTok{abline}\NormalTok{(}\DataTypeTok{v=}\NormalTok{table.noise[i,}\KeywordTok{colnames}\NormalTok{(table.noise)}\OperatorTok{==}\StringTok{'sampling.noise'}\NormalTok{],}\DataTypeTok{col=}\StringTok{"red"}\NormalTok{)}
\NormalTok{\}}
\end{Highlighting}
\end{Shaded}

\begin{figure}[htbp]

{\centering \includegraphics[width=0.6\linewidth]{STCI_files/figure-latex/sampnoisewwcheball-1} 

}

\caption{Distribution of the Chebyshev upper bound on sampling noise over replications of samples of different sizes (true sampling noise in red)}\label{fig:sampnoisewwcheball}
\end{figure}

Figure \ref{fig:sampnoisewwcheball} shows that the upper bound works: it
is always bigger than the true sampling noise. Figure
\ref{fig:sampnoisewwcheball} also shows that the upper bound is large:
it generally is of an order of magnitude bigger than the true sampling
noise, and thus offers a blurry and too pessimistic view of the
precision of an estimator. Figure \ref{fig:sampnoisewwchebplot} shows
that the average Chebyshev bound gives an inflated estimate of sampling
noise. Figure \ref{fig:confintervalcheb} shows that the Chebyshev
confidence intervals are clearly less precise than the true unknown
ones. With \(N=\) 1000, the true confidence intervals generally reject
large negative effects, whereas the Chebyshev confidence intervals do
not rule out this possibility. With \(N=\) 10\^{}\{4\}, the true
confidence intervals generally reject effects smaller than 0.1, whereas
the Chebyshev confidence intervals cannot rule out small negative
effects.

As a conclusion on Chebyshev estimates of sampling noise, their
advantage is that they offer an upper bound on the noise: we can never
underestimate noise if we use them. A downside of Chebyshev sampling
noise is their lower precision, which makes it harder to pinpoint the
true confidence intervals.

\begin{Shaded}
\begin{Highlighting}[]
\ControlFlowTok{for}\NormalTok{ (k }\ControlFlowTok{in}\NormalTok{ (}\DecValTok{1}\OperatorTok{:}\KeywordTok{length}\NormalTok{(N.sample)))\{}
\NormalTok{  table.noise}\OperatorTok{$}\NormalTok{cheb.noise[k] <-}\StringTok{ }\KeywordTok{mean}\NormalTok{(simuls.ww[[k]]}\OperatorTok{$}\NormalTok{cheb.noise)}
\NormalTok{\}}
\KeywordTok{ggplot}\NormalTok{(table.noise, }\KeywordTok{aes}\NormalTok{(}\DataTypeTok{x=}\KeywordTok{as.factor}\NormalTok{(N), }\DataTypeTok{y=}\NormalTok{TT)) }\OperatorTok{+}
\StringTok{  }\KeywordTok{geom_bar}\NormalTok{(}\DataTypeTok{position=}\KeywordTok{position_dodge}\NormalTok{(), }\DataTypeTok{stat=}\StringTok{"identity"}\NormalTok{, }\DataTypeTok{colour=}\StringTok{'black'}\NormalTok{) }\OperatorTok{+}
\StringTok{  }\KeywordTok{geom_errorbar}\NormalTok{(}\KeywordTok{aes}\NormalTok{(}\DataTypeTok{ymin=}\NormalTok{TT}\OperatorTok{-}\NormalTok{sampling.noise}\OperatorTok{/}\DecValTok{2}\NormalTok{, }\DataTypeTok{ymax=}\NormalTok{TT}\OperatorTok{+}\NormalTok{sampling.noise}\OperatorTok{/}\DecValTok{2}\NormalTok{), }\DataTypeTok{width=}\NormalTok{.}\DecValTok{2}\NormalTok{,}\DataTypeTok{position=}\KeywordTok{position_dodge}\NormalTok{(.}\DecValTok{9}\NormalTok{),}\DataTypeTok{color=}\StringTok{'red'}\NormalTok{) }\OperatorTok{+}
\StringTok{  }\KeywordTok{geom_errorbar}\NormalTok{(}\KeywordTok{aes}\NormalTok{(}\DataTypeTok{ymin=}\NormalTok{TT}\OperatorTok{-}\NormalTok{cheb.noise}\OperatorTok{/}\DecValTok{2}\NormalTok{, }\DataTypeTok{ymax=}\NormalTok{TT}\OperatorTok{+}\NormalTok{cheb.noise}\OperatorTok{/}\DecValTok{2}\NormalTok{), }\DataTypeTok{width=}\NormalTok{.}\DecValTok{2}\NormalTok{,}\DataTypeTok{position=}\KeywordTok{position_dodge}\NormalTok{(.}\DecValTok{9}\NormalTok{),}\DataTypeTok{color=}\StringTok{'blue'}\NormalTok{) }\OperatorTok{+}
\StringTok{  }\KeywordTok{xlab}\NormalTok{(}\StringTok{"Sample Size"}\NormalTok{)}\OperatorTok{+}
\StringTok{  }\KeywordTok{theme_bw}\NormalTok{()}
\end{Highlighting}
\end{Shaded}

\begin{figure}[htbp]

{\centering \includegraphics[width=0.6\linewidth]{STCI_files/figure-latex/sampnoisewwchebplot-1} 

}

\caption{Average Chebyshev upper bound on sampling noise over replications of samples of different sizes (true sampling noise in red)}\label{fig:sampnoisewwchebplot}
\end{figure}

\begin{Shaded}
\begin{Highlighting}[]
\NormalTok{N.plot <-}\StringTok{ }\DecValTok{40}
\NormalTok{plot.list <-}\StringTok{ }\KeywordTok{list}\NormalTok{()}

\ControlFlowTok{for}\NormalTok{ (k }\ControlFlowTok{in} \DecValTok{1}\OperatorTok{:}\KeywordTok{length}\NormalTok{(N.sample))\{}
  \KeywordTok{set.seed}\NormalTok{(}\DecValTok{1234}\NormalTok{)}
\NormalTok{  test.cheb <-}\StringTok{ }\NormalTok{simuls.ww[[k]][}\KeywordTok{sample}\NormalTok{(N.plot),}\KeywordTok{c}\NormalTok{(}\StringTok{'WW'}\NormalTok{,}\StringTok{'cheb.noise'}\NormalTok{)]}
\NormalTok{  test.cheb <-}\StringTok{ }\KeywordTok{as.data.frame}\NormalTok{(}\KeywordTok{cbind}\NormalTok{(test.cheb,}\KeywordTok{rep}\NormalTok{(}\KeywordTok{samp.noise}\NormalTok{(simuls.ww[[k]][,}\StringTok{'WW'}\NormalTok{],}\DataTypeTok{delta=}\NormalTok{delta),N.plot)))}
  \KeywordTok{colnames}\NormalTok{(test.cheb) <-}\StringTok{ }\KeywordTok{c}\NormalTok{(}\StringTok{'WW'}\NormalTok{,}\StringTok{'cheb.noise'}\NormalTok{,}\StringTok{'sampling.noise'}\NormalTok{)}
\NormalTok{  test.cheb}\OperatorTok{$}\NormalTok{id <-}\StringTok{ }\DecValTok{1}\OperatorTok{:}\NormalTok{N.plot}
\NormalTok{  plot.test.cheb <-}\StringTok{ }\KeywordTok{ggplot}\NormalTok{(test.cheb, }\KeywordTok{aes}\NormalTok{(}\DataTypeTok{x=}\KeywordTok{as.factor}\NormalTok{(id), }\DataTypeTok{y=}\NormalTok{WW)) }\OperatorTok{+}
\StringTok{      }\KeywordTok{geom_bar}\NormalTok{(}\DataTypeTok{position=}\KeywordTok{position_dodge}\NormalTok{(), }\DataTypeTok{stat=}\StringTok{"identity"}\NormalTok{, }\DataTypeTok{colour=}\StringTok{'black'}\NormalTok{) }\OperatorTok{+}
\StringTok{      }\KeywordTok{geom_errorbar}\NormalTok{(}\KeywordTok{aes}\NormalTok{(}\DataTypeTok{ymin=}\NormalTok{WW}\OperatorTok{-}\NormalTok{sampling.noise}\OperatorTok{/}\DecValTok{2}\NormalTok{, }\DataTypeTok{ymax=}\NormalTok{WW}\OperatorTok{+}\NormalTok{sampling.noise}\OperatorTok{/}\DecValTok{2}\NormalTok{), }\DataTypeTok{width=}\NormalTok{.}\DecValTok{2}\NormalTok{,}\DataTypeTok{position=}\KeywordTok{position_dodge}\NormalTok{(.}\DecValTok{9}\NormalTok{),}\DataTypeTok{color=}\StringTok{'red'}\NormalTok{) }\OperatorTok{+}
\StringTok{      }\KeywordTok{geom_errorbar}\NormalTok{(}\KeywordTok{aes}\NormalTok{(}\DataTypeTok{ymin=}\NormalTok{WW}\OperatorTok{-}\NormalTok{cheb.noise}\OperatorTok{/}\DecValTok{2}\NormalTok{, }\DataTypeTok{ymax=}\NormalTok{WW}\OperatorTok{+}\NormalTok{cheb.noise}\OperatorTok{/}\DecValTok{2}\NormalTok{), }\DataTypeTok{width=}\NormalTok{.}\DecValTok{2}\NormalTok{,}\DataTypeTok{position=}\KeywordTok{position_dodge}\NormalTok{(.}\DecValTok{9}\NormalTok{),}\DataTypeTok{color=}\StringTok{'blue'}\NormalTok{) }\OperatorTok{+}
\StringTok{      }\KeywordTok{geom_hline}\NormalTok{(}\KeywordTok{aes}\NormalTok{(}\DataTypeTok{yintercept=}\KeywordTok{delta.y.ate}\NormalTok{(param)), }\DataTypeTok{colour=}\StringTok{"#990000"}\NormalTok{, }\DataTypeTok{linetype=}\StringTok{"dashed"}\NormalTok{)}\OperatorTok{+}
\StringTok{      }\KeywordTok{xlab}\NormalTok{(}\StringTok{"Sample id"}\NormalTok{)}\OperatorTok{+}
\StringTok{      }\KeywordTok{theme_bw}\NormalTok{()}\OperatorTok{+}
\StringTok{      }\KeywordTok{ggtitle}\NormalTok{(}\KeywordTok{paste}\NormalTok{(}\StringTok{"N="}\NormalTok{,N.sample[k]))}
\NormalTok{  plot.list[[k]] <-}\StringTok{ }\NormalTok{plot.test.cheb }
\NormalTok{\}}
\NormalTok{plot.CI <-}\StringTok{ }\KeywordTok{plot_grid}\NormalTok{(plot.list[[}\DecValTok{1}\NormalTok{]],plot.list[[}\DecValTok{2}\NormalTok{]],plot.list[[}\DecValTok{3}\NormalTok{]],plot.list[[}\DecValTok{4}\NormalTok{]],}\DataTypeTok{ncol=}\DecValTok{1}\NormalTok{,}\DataTypeTok{nrow=}\KeywordTok{length}\NormalTok{(N.sample))}
\KeywordTok{print}\NormalTok{(plot.CI)}
\end{Highlighting}
\end{Shaded}

\begin{figure}[htbp]

{\centering \includegraphics[width=0.6\linewidth]{STCI_files/figure-latex/confintervalcheb-1} 

}

\caption{Chebyshev confidence intervals of $\hat{WW}$ for $\delta=$ 0.99 over sample replications for various sample sizes (true confidence intervals in red)}\label{fig:confintervalcheb}
\end{figure}

\subsection{Using the Central Limit Theorem}\label{sec:CLT}

\subsection{Using resampling methods}\label{sec:resamp}

\subsection{Using randomization inference}\label{sec:Fisher}


\end{document}
